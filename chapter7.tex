%%%%%%%%%%%%%%%%%%%%%%%%%%%%%%%%%%%%%
\begin{frame} [plain]
    \frametitle{}
    %\Background[2] 
    \begin{center}
    { {\huge 第七章:自旋与全同粒子 }}
    \end{center}  
    \addtocounter{framenumber}{-1}   
\end{frame}
%%%%%%%%%%%%%%%%%%%%%%%%%%%%%%%%%%

\section{自旋 }
\subsection{电子自旋}
\begin{frame}
  \frametitle{斯特恩和盖拉赫实验}
  1921年,Stern-Gerlach发现处于s态的银原子(类氢原子),经过非均匀磁场后分成两束。 磁矩在z方向的投影为一个玻尔磁子
\begin{figure}[htbp]
  \centering
  \includegraphics[width=0.8\textwidth]{figs/sgexp.png}
  %\caption{}
    %\label{fig:}
\end{figure}
\end{frame} 

\begin{frame}[label=current]
  \frametitle{}
\emf[分析:] 设氢原子磁矩为$\vec{M} $, 外磁场为$\vec{B}$,\\
根据经典理论,磁场中氢原子的势能
\[ U= - \vec{M} \cdot \vec{B} = -MB_z \cos\theta \] 
氢原子受力
\[ F= -\frac{\partial U}{\partial z } = -M \frac{\partial B_z}{\partial z }  \cos\theta \] 
原子磁矩方向不定,$\cos\theta \in (-1, 1)$,不应出现两个峰 

~~\\ 
根据“量子理论”:角动量简并度$2l+1$,去简并产生“2“,有
\[2l+1 =2 \implies l=\frac{1}{2}\]
这是不可能的!s态氢原子有$l=0$
\end{frame} 

\begin{frame}[label=current]
  \frametitle{}
如果有 $ l=\dfrac{1}{2}  $ , 则有磁量子数
\[ m=-l, -(l-1), \cdots, (l-1), +l, \implies m=\pm \frac{1}{2}\]
角动量投影
\[\implies L_z = m\hbar = \pm \frac{1}{2}\hbar\]

~~\\ 
实验证实存在角动量:$\pm \dfrac{1}{2}\hbar$,理论上却没有!

~~\\ 
怎么办? 修改理论!
\end{frame} 


\begin{frame}
  \frametitle{自旋假说}
1925年,乌伦贝克和哥德斯密特假设:电子除了具有轨道角动量$L$外,还有一个新的角动量$S$(自旋角动量)\\

(1)自旋角量子数 
\[s = \frac{1}{2} \quad \text{对应:} l= 0,1,2,\cdots n-1\]

(2)自旋磁量子数 
\[m_s = \pm \frac{1}{2} \quad \text{对应:} m= 0,\pm 1,\pm_2,\cdots \pm l\]

则自旋角动量的本征值为
\[\begin{aligned}
  S^2 &= s(s+1) \hbar^2 = \frac{3}{4} \hbar^2 \\
  S_z &= m_s \hbar = \pm \frac{1}{2} \hbar \\
\end{aligned}\]
\end{frame} 

\begin{frame}[label=current]
  \frametitle{}
(3)自旋磁矩与自旋角动量的关系为
\[ M_S = - \frac{e}{\mu} S\]
磁矩的Z轴的投影大小为一个玻尔磁子
\[ M_{S_z} = - \frac{e}{\mu} S_z = \mp \frac{e}{2\mu } \hbar = \mp M_{B} \]
注意:轨道磁矩与轨道角动量的关系为
\[ M_L = - \frac{e}{2\mu} L\]
\end{frame} 

\begin{frame}[label=current]
  \frametitle{}
\begin{figure}[htbp]
  \centering
  \includegraphics[width=0.99\textwidth]{figs/spin10.png}
  %\caption{}
    %\label{fig:}
\end{figure}
${\color{red}\star}~$ 自旋是一种没有经典对应的纯量子效应,是粒子的固有属性。
\end{frame} 


\begin{frame}
  \frametitle{自旋算符}
这是一个可观测物理量,用厄密算符$\hat{S}$描述,称为自旋算符 

~~\\ 
(1) 对易关系
\[
\begin{aligned}
  &\hat{S}\times \hat{S} = i\hbar \hat{S} \\
  &[S_\alpha, S_\beta ] = \varepsilon_{\alpha\beta\gamma} i\hbar S_\gamma  \\
  &[S_\alpha, S^2] =0
\end{aligned}  
\]
\end{frame} 

\begin{frame}[label=current]
  \frametitle{}
(2)本征方程 \\

由于$[S_z, S^2] =0$, 它们有共同本征矢,记为
$$\left\vert sm_s  \right\rangle \implies \left\{\begin{aligned}
  \left\vert \uparrow \right\rangle = \left\vert +\right\rangle &= \left\vert \frac{1}{2}\frac{1}{2} \right\rangle \\ 
  \left\vert \downarrow \right\rangle = \left\vert - \right\rangle &= \left\vert \frac{1}{2}-\frac{1}{2} \right\rangle \\ 
\end{aligned} \right.$$
本征方程(得解)
$$\left\{\begin{aligned}
  S^2 \left\vert sm_s \right\rangle &= s(s+1)\hbar^2\left\vert sm_s \right\rangle \\ 
  S_z \left\vert sm_s \right\rangle &= m_s\hbar \left\vert sm_s \right\rangle 
\end{aligned} \right.$$
\end{frame} 

\begin{frame}[label=current]
  \frametitle{}
  (3)自旋表象($S^2,S_z$) \\
  本征矢 $\left\vert \uparrow \right\rangle$ 和 $\left\vert \downarrow \right\rangle$ 构成正交归一完备系, 任意态函数可在其上展开
  \[ \left\vert \psi \right\rangle = a  \left\vert \uparrow \right\rangle + b \left\vert \downarrow \right\rangle\]
  因此,有矩阵表示
  \[ \left\vert \psi \right\rangle = \begin{pmatrix}
   a\\
   b
  \end{pmatrix}\]
  本征矢在自身表象的矩阵表示
  \[ \left\vert \uparrow \right\rangle = \begin{pmatrix}
    1\\
    0
   \end{pmatrix}, \qquad \left\vert \downarrow \right\rangle = \begin{pmatrix}
    0\\
    1
   \end{pmatrix}\]
\end{frame} 

\begin{frame}[label=current]
  \frametitle{}
  根据算符理论,算符在其自身表象中为对角矩阵,矩阵的维度是本征函数的数目,对角元是本征值, 得自旋算符的矩阵表示
  \[\hat{S}^2 = \frac{3}{4}\hbar^2\begin{pmatrix}
    1 & 0\\
    0 & 1
   \end{pmatrix}, \qquad \hat{S}_z = \frac{1}{2}\hbar\begin{pmatrix}
    1 & 0\\
    0 & -1
   \end{pmatrix}\]
\end{frame} 

\begin{frame}[label=current]
  \frametitle{}
\例[1]{试证明
\[ \hat{S}_z = \frac{\hbar}{2}\left(\left\vert \uparrow \right\rangle\left\langle  \uparrow\right\vert - \left\vert \downarrow \right\rangle\left\langle  \downarrow\right\vert\right), \quad \hat{S}^2_z = \frac{\hbar^2}{4} I\]
}
\证 (1) 先计算矩阵
\[ \begin{aligned}
  \begin{pmatrix}
    1 & 0\\
    0 & -1
   \end{pmatrix} &=  \begin{pmatrix}
    1 & 0\\
    0 & 0
   \end{pmatrix}  -
   \begin{pmatrix}
    0 & 0\\
    0 & 1
   \end{pmatrix}\\ 
   &= \begin{pmatrix}
    1 \\
    0 
   \end{pmatrix} \begin{pmatrix}
    1 & 0 \\ 
   \end{pmatrix} - \begin{pmatrix}
    0 \\
    1 
   \end{pmatrix} \begin{pmatrix}
    0 & 1 \\ 
   \end{pmatrix} \\
   &= \left\vert \uparrow \right\rangle\left\langle  \uparrow\right\vert - \left\vert \downarrow \right\rangle\left\langle  \downarrow\right\vert
\end{aligned}\]
\[\hat{S}_z = \frac{1}{2}\hbar\begin{pmatrix}
  1 & 0\\
  0 & -1
 \end{pmatrix} = \frac{\hbar}{2}\left(\left\vert \uparrow \right\rangle\left\langle  \uparrow\right\vert - \left\vert \downarrow \right\rangle\left\langle  \downarrow\right\vert\right)\]
\end{frame} 

\begin{frame}[label=current]
  \frametitle{}
(2) 先做本征矢计算
\[ \begin{aligned}
  &\left[\left\vert \uparrow \right\rangle\left\langle  \uparrow\right\vert - \left\vert \downarrow \right\rangle\left\langle  \downarrow\right\vert\right]\cdot \left[\left\vert \uparrow \right\rangle\left\langle  \uparrow\right\vert - \left\vert \downarrow \right\rangle\left\langle \downarrow\right\vert\right] \\
  &= \left\vert \uparrow \right\rangle\left\langle  \uparrow\left\vert \uparrow \right\rangle\left\langle  \uparrow\right\vert\right. 
  - \left\vert \uparrow \right\rangle\left\langle \uparrow \left\vert \downarrow \right\rangle\left\langle  \downarrow\right\vert\right. 
  - \left\vert \downarrow \right\rangle\left\langle \downarrow \left\vert \uparrow \right\rangle\left\langle  \uparrow\right\vert\right.
  + \left\vert \downarrow \right\rangle\left\langle \downarrow \left\vert \downarrow \right\rangle\left\langle \downarrow\right\vert\right. \\
  &=\left\vert \uparrow \right\rangle\left\langle  \uparrow\right\vert + \left\vert \downarrow \right\rangle\left\langle  \downarrow\right\vert  \\
  &= \begin{pmatrix}
    1 & 0\\
    0 & 1
   \end{pmatrix} =I 
\end{aligned}\]
\[ \begin{aligned}\hat{S}^2_z &= \hat{S}_z\cdot\hat{S}_z \\
  &= \frac{\hbar^2}{4} \left[\left\vert \uparrow \right\rangle\left\langle  \uparrow\right\vert - \left\vert \downarrow \right\rangle\left\langle  \downarrow\right\vert\right]\cdot \left[\left\vert \uparrow \right\rangle\left\langle  \uparrow\right\vert - \left\vert \downarrow \right\rangle\left\langle \downarrow\right\vert\right] ~\qquad \qquad \qquad \quad \qquad\\
  &= \frac{\hbar^2}{4} I \end{aligned}\]
\end{frame} 

\begin{frame}[label=current]
  \frametitle{}
  \例[2]{试证明反对易关系
  \[ \left\{\hat{S}_\alpha, \hat{S}_\beta\right\}\equiv \hat{S}_\alpha \hat{S}_\beta + \hat{S}_\beta \hat{S}_\alpha = 0 \]
  }
  \证
  代入对易关系 $ \hat{S}_y \hat{S}_z - \hat{S}_z \hat{S}_y = i\hbar S_x  $, 有 
  \[ \begin{aligned}
    \hat{S}_x \hat{S}_y + \hat{S}_y \hat{S}_x  &=\frac{1}{i\hbar}(\hat{S}_y \hat{S}_z - \hat{S}_z \hat{S}_y)\hat{S}_y + \frac{1}{i\hbar}\hat{S}_y(\hat{S}_y \hat{S}_z - \hat{S}_z \hat{S}_y)  \\ 
    &= \frac{1}{i\hbar}(\hat{S}_y \hat{S}_z \hat{S}_y -   \hat{S}_z \hat{S}_y\hat{S}_y + \hat{S}_y \hat{S}_y \hat{S}_z - \hat{S}_y \hat{S}_z \hat{S}_y)\\
    &= \frac{1}{i\hbar}(\hat{S}_y \hat{S}_z \hat{S}_y -   \hat{S}_z \frac{\hbar^2}{4} I + \frac{\hbar^2}{4} I \hat{S}_z - \hat{S}_y \hat{S}_z \hat{S}_y) \\
    &= 0
  \end{aligned}\]
\end{frame} 

\begin{frame}[label=current]
  \frametitle{}
  \例[3]{试证明算符$S_x, S_y$在$S^2 S_z$表象有如下矩阵表示
  \[\hat{S}_x = \frac{1}{2}\hbar\begin{pmatrix}
    0 & 1\\
    1 & 0
   \end{pmatrix}, \qquad \hat{S}_y = \frac{1}{2}\hbar\begin{pmatrix}
    0 & -i\\
    i & 0
   \end{pmatrix}\]
  }
  \证 $S_x$是厄密矩阵,有 
  \[\hat{S}_x = \frac{1}{2}\hbar\begin{pmatrix}
    a & c\\
    b & d
   \end{pmatrix} = \frac{1}{2}\hbar\begin{pmatrix}
    a^* & b^*\\
    c^* & d^*
   \end{pmatrix} \]
   可知: $a, d$是实数, $c=b^*$,即
   \[\hat{S}_x = \frac{1}{2}\hbar\begin{pmatrix}
    a & b^*\\
    b & d
   \end{pmatrix}\]
\end{frame} 

\begin{frame}[label=current]
  \frametitle{}
代入反对易关系
\[\begin{aligned}
  0&= \hat{S}_x \hat{S}_z + \hat{S}_z \hat{S}_x   \\ 
  &= \frac{1}{4}\hbar^2 \left\{\begin{pmatrix}
    a & b^*\\
    b & d
   \end{pmatrix} \begin{pmatrix}
    1 & 0\\
    0 & -1
   \end{pmatrix} + \begin{pmatrix}
    1 & 0\\
    0 & -1
   \end{pmatrix} \begin{pmatrix}
    a & b^*\\
    b & d
   \end{pmatrix}  \right\} \\
   &= \frac{1}{4}\hbar^2 \begin{pmatrix}
    2a & 0\\
    0 & -2d
   \end{pmatrix}  \\
   \implies & a = d =0
\end{aligned}\]
因此
\[\hat{S}_x = \frac{1}{2}\hbar\begin{pmatrix}
  0 & b^*\\
  b & 0
 \end{pmatrix}\]
\end{frame} 

\begin{frame}[label=current]
  \frametitle{}
代入 $\hat{S}^2_x = \frac{\hbar^2}{4} I $,得
\[\begin{aligned}
  \frac{\hbar^2}{4} I &= \hat{S}^2_x \\
&= \frac{\hbar^2}{4}\hbar\begin{pmatrix}
  0 & b^*\\
  b & 0
 \end{pmatrix} \cdot\begin{pmatrix}
  0 & b^*\\
  b & 0
 \end{pmatrix} \\
 &= \frac{\hbar^2}{4} \begin{pmatrix}
  \left\vert b\right\vert^2 & 0\\
  0 & \left\vert b\right\vert^2
 \end{pmatrix}
\end{aligned}\]
得 $$ \left\vert b\right\vert^2 =1 $$
为简单计,取 $ b =1 $ ,得 
\[\hat{S}_x = \frac{1}{2}\hbar\begin{pmatrix}
  0 & 1\\
  1 & 0
 \end{pmatrix}\]
\end{frame} 

\begin{frame}[label=current]
  \frametitle{}
代入对易关系 $ \hat{S}_z \hat{S}_x - \hat{S}_x \hat{S}_z = i\hbar S_y  $, 有
\[\begin{aligned}
  S_y &= \frac{1}{i\hbar} \left[\hat{S}_z \hat{S}_x - \hat{S}_x \hat{S}_z\right] \\
  &= \frac{1}{i\hbar} \frac{1}{4}\hbar^2 \left\{\begin{pmatrix}
    1 & 0\\
    0 & -1
   \end{pmatrix}\begin{pmatrix}
    0 & 1\\
    1 & 0
   \end{pmatrix}- \begin{pmatrix}
    0 & 1\\
    1 & 0
   \end{pmatrix} \begin{pmatrix}
    1 & 0\\
    0 & -1
   \end{pmatrix}\right\} \\
   &=\frac{1}{4i} \hbar \left\{\begin{pmatrix} 
    0 & 1\\
    -1 & 0
   \end{pmatrix}  - \begin{pmatrix} 
    0 & -1\\
    1 & 0
   \end{pmatrix} \right\} \\
   &= \frac{1}{2}\hbar \begin{pmatrix} 
    0 & -i\\
    i & 0
   \end{pmatrix}
\end{aligned}\]
\end{frame} 

\begin{frame}[label=current]
  \frametitle{泡利矩阵}
  引入泡利矩阵, 自旋算符
  \[\hat{S}^2 = \frac{3}{4}\hbar^2\begin{pmatrix}
    1 & 0\\
    0 & 1
   \end{pmatrix} \quad \hat{S}_x = \frac{1}{2}\hbar\begin{pmatrix}
    0 & 1\\
    1 & 0
   \end{pmatrix}\quad \hat{S}_y = \frac{1}{2}\hbar\begin{pmatrix}
    0 & -i\\
    i & 0
   \end{pmatrix}  \quad \hat{S}_z = \frac{1}{2}\hbar\begin{pmatrix}
    1 & 0\\
    0 & -1
   \end{pmatrix} \]
   可表示为 
   \[\quad \hat{S}^2 = \frac{3}{4}\hbar^2 I \quad \hat{S}_x = \frac{1}{2}\hbar\sigma _x \quad \hat{S}_y = \frac{1}{2}\hbar \sigma _y  \quad \hat{S}_z = \frac{1}{2}\hbar \sigma _z \]
   显然, 存在反对易关系
   \[\sigma _\alpha \sigma _\beta + \sigma _\beta \sigma _\alpha =0 \]
\end{frame} 


\subsection{电子波函数}
\begin{frame}[label=current]
  \frametitle{电子波函数}
  电子既然具有自旋,应增加一个自由度,可用$S_z$表述,构成力学量完全集的力学量数目变为4个,波函数为:
  \[ \Psi(\vec{q},t) = \Psi(\vec{r},S_z, t) = \Psi(x,y,z,S_z, t) \]
  在自旋表象展开
  \[\left\vert \Psi(x,y,z,S_z, t) \right\rangle  = a \left\vert \uparrow \right\rangle + b \left\vert \downarrow \right\rangle \]
  取定本征值
  \[\left\vert \Psi(x,y,z,S_z, t) \right\rangle  = \begin{pmatrix}
    a\\
    b 
   \end{pmatrix} = \begin{pmatrix}
    \left\vert \Psi(x,y,z,~~\frac{\hbar}{2}, t) \right\rangle\\
    \left\vert \Psi(x,y,z,-\frac{\hbar}{2}, t) \right\rangle
   \end{pmatrix} = \begin{pmatrix}
    \left\vert \Psi _{~~\frac{\hbar}{2}}(x,y,z,t) \right\rangle\\
    \left\vert \Psi_{-\frac{\hbar}{2}}(x,y,z,t) \right\rangle
   \end{pmatrix}  \]
\end{frame} 

\begin{frame}[label=current]
  \frametitle{}
自旋向上的概率密度
\[ \omega _{\frac{\hbar}{2}} = \left\vert a\right\vert^2 =\left\vert\Psi _{\frac{\hbar}{2}}(x,y,z,t)\right\vert^2\]
自旋向下的概率密度
\[ \omega _{-\frac{\hbar}{2}} = \left\vert b\right\vert^2 =\left\vert\Psi _{-\frac{\hbar}{2}}(x,y,z,t)\right\vert^2\]
总概率归一
\[ \left\vert\Psi _{\frac{\hbar}{2}}(x,y,z,t)\right\vert^2+ \left\vert\Psi _{-\frac{\hbar}{2}}(x,y,z,t)\right\vert^2 =1\]
表明:电子在时空某点取自旋向上或向下的概率和为1
\end{frame} 

\begin{frame}[label=current]
  \frametitle{}
同理,球坐标系下,有
\[\left\vert \Psi\right\rangle = \sum_k a_k \left\vert nlmm_s\right\rangle \]
量子数组$\{n,l,m,m_s\}$有任何不同,即代表一个不同的量子态$k$
\end{frame} 

\begin{frame}[label=current]
  \frametitle{}
\例[4]{设氢原子处于如下状态, 
\[\left\vert \Psi\right\rangle = \frac{1}{4} \left\vert 100\frac{1}{2}\right\rangle + \frac{\sqrt{5} }{4} \left\vert 100-\frac{1}{2}\right\rangle + \frac{\sqrt{7} }{4} \left\vert 210-\frac{1}{2}\right\rangle + \frac{\sqrt{3}  }{4} \left\vert 210\frac{1}{2}\right\rangle \]
求(1)电子处于自旋向上的概率\\
(2)电子处于自旋向下时,角动量$\hat{L}^2$的可能值及平均值}
\解 波函数已归一化。\\
(1)第一项和第四项电子自旋向上 \\
因此自旋向上的概率为
\[\omega _{\frac{1}{2}} = (\frac{1}{4})^2 + (\frac{\sqrt{3 } }{4})^2 = \frac{1}{4}\]
\end{frame} 

\begin{frame}[label=current]
  \frametitle{}
(2)第二项和第三项电子自旋向下,
\begin{table}[htbp]
  \centering\begin{tabular}{cccc}
    第二项 & $l= 0$ & $\hat{L}^2 = l(l+1)\hbar^2 = 0$ & $\omega _{l=0} = \frac{5}{16}$ \\
    \\
    第三项 & $l= 1$ & $\hat{L}^2 = l(l+1)\hbar^2 = 2\hbar^2$ & $\omega _{l=1} = \frac{7}{16}$ \\
  \end{tabular}
  %\caption{<caption>}
  %\label{<label>}
\end{table}
平均值
$$\overline{\hat{L}^2} =2\hbar^2 \times \frac{7}{16} = \frac{7}{8}\hbar^2$$
\end{frame} 

\begin{frame}[label=current]
  \frametitle{}
  既然波函数增加了一个自由度,算符也一样
  \[ \hat{F} =f (\hat{r},\hat{p}) \implies \hat{F} =f (\hat{r},\hat{p}, \hat{S}_z)  \]
代入矩阵形式$\hat{S}_z = \frac{\hbar}{2} \begin{pmatrix} 1 &  0 \\ 0 & -1 \end{pmatrix}$,得 
\[ \hat{F} = \begin{pmatrix} F_{11} &  F_{12} \\ F_{21} & F_{22} \end{pmatrix}\]
\end{frame} 

\begin{frame}[label=current]
  \frametitle{}
算符的平均值
\[ \begin{aligned}
\overline{F} &= \int \begin{pmatrix} \Psi ^* _\frac{\hbar}{2} &  \Psi^* _{-\frac{\hbar}{2}} \end{pmatrix} \begin{pmatrix} F_{11} &  F_{12} \\ F_{21} & F_{22} \end{pmatrix} \begin{pmatrix} \Psi _\frac{\hbar}{2} \\  \Psi _{-\frac{\hbar}{2}} \end{pmatrix} d \tau \\ 
&= \int \left[ \Psi ^* _\frac{\hbar}{2} F_{11} \Psi _\frac{\hbar}{2}  
+ \Psi ^* _{\frac{\hbar}{2}} F_{12} \Psi _{-\frac{\hbar}{2}} 
+ \Psi ^* _{-\frac{\hbar}{2}} F_{21} \Psi _{\frac{\hbar}{2}} 
+ \Psi ^* _{-\frac{\hbar}{2}} F_{22} \Psi _{-\frac{\hbar}{2}}  \right]d\tau
\end{aligned} 
  \]
第一项是电子只能自旋向上的情况, 第四项是电子只能自旋向下的情况,中间两项源于电子既自旋向上又自旋向下的叠加效应。
\end{frame} 

\begin{frame}[label=current]
  \frametitle{自旋波函数}
电子哈密顿$ H (\vec{r},S_z)  $, 如果轨道与自旋没有耦合,有 
\[ H (\vec{r},S_z) = H (\vec{r}) + H (S_z)\]
此时,电子波函数可以写成
\[\Psi(\vec{r}, S_z, t)  = \Psi(\vec{r}, t)  \chi (S_z, t) \] 
其中 $\chi (S_z, t)$描述电子的自旋状态,称为自旋波函数。
\end{frame} 

\begin{frame}[label=current]
  \frametitle{}
在$S^2 S_z$表象展开,有
\[ \left \vert\chi (S_z) \right\rangle = a \left\vert \uparrow \right\rangle + b \left\vert \downarrow \right\rangle \]
分别取$S_z=\pm \frac{1}{2}\hbar$,有
\[ 
  \left \vert\chi (\frac{1}{2}\hbar) \right\rangle  =\left \vert\chi _{\frac{1}{2}} \right\rangle= \begin{pmatrix}
  1 \\
  0 
 \end{pmatrix} = \left\vert \uparrow \right\rangle = \left\vert + \right\rangle = \left\vert \frac{1}{2} \frac{1}{2} \right\rangle \equiv \alpha \]
 \[ 
  \left \vert\chi (-\frac{1}{2}\hbar) \right\rangle  =\left \vert\chi _{-\frac{1}{2}} \right\rangle= \begin{pmatrix}
  0 \\
  1 
 \end{pmatrix} = \left\vert \downarrow \right\rangle = \left\vert - \right\rangle = \left\vert \frac{1}{2} -\frac{1}{2} \right\rangle  \equiv \beta\]
 氢原子波函数
 \[\Psi _{nlmm_s}  = \psi _{nlm}  \chi _{m_z} = \begin{pmatrix}
  \psi _{nlm} \chi _{+\frac{1}{2}} \\
  \psi _{nlm} \chi _{-\frac{1}{2}} 
 \end{pmatrix}  = \begin{pmatrix}
  \psi _{+} \\
  \psi _{-} 
 \end{pmatrix} = \begin{pmatrix}
  \psi _{1} \\
  \psi _{2} 
 \end{pmatrix}\] 
\end{frame} 

\subsection{磁场中原子的光谱劈裂}

\begin{frame}
  \frametitle{磁场中原子的光谱劈裂}
   \例[5]{不考虑旋-轨耦合,求磁场中类氢原子的能级}
  \解 取磁场方向为z方向,氢原子的附加能
  \[ \begin{aligned}
   U_B &= -\vec{M}\cdot \vec{B} \\ 
   &= -(\vec{M}_L +\vec{M}_S )\cdot \vec{B} \\
   &= \frac{e}{2 \mu } (L_z + 2S_z)B
  \end{aligned} \]
  哈密顿
  \[H= - \frac{\hbar^2}{2\mu} \nabla^2 +V(r) + U_B\]
  本征函数
  \[\Psi _{nlmm_s}  = \psi _{nlm}  \chi _{m_s} = \begin{pmatrix}
    \psi _{+} \\
    \psi _{-} 
   \end{pmatrix} \]
\end{frame} 

\begin{frame}
  \frametitle{}
代入定态薛定谔方程, 有
\[ 
\begin{aligned}
  \begin{pmatrix}
    - \frac{\hbar^2}{2\mu} \nabla^2 +V(r) + \frac{e}{2 \mu } (L_z + 2S_z)B 
   \end{pmatrix}  \psi _{nlm}  \chi _{m_s} = E \psi _{nlm}  \chi _{m_s}  
\end{aligned}
\]
取 $S_z = \pm \frac{1}{2} \hbar, m_s = \pm \frac{1}{2} $,得两个方程
\[ 
\begin{aligned}
  \begin{pmatrix}
    - \frac{\hbar^2}{2\mu} \nabla^2 +V(r) + \frac{e}{2 \mu } (L_z + \hbar)B 
   \end{pmatrix} \psi _{+} = E_{+} \psi _{+} 
\end{aligned}
\]

\[ 
\begin{aligned}
  \begin{pmatrix}
    - \frac{\hbar^2}{2\mu} \nabla^2 +V(r) + \frac{e}{2 \mu } (L_z - \hbar)B 
   \end{pmatrix} \psi _{-} = E_{-} \psi _{-} 
\end{aligned}
\]
\end{frame} 

\begin{frame}[label=current]
  \frametitle{}
统一写成
\[ 
\begin{aligned}
  \begin{pmatrix}
    - \frac{\hbar^2}{2\mu} \nabla^2 +V(r) + \frac{e}{2 \mu } (L_z \pm \hbar)B 
   \end{pmatrix} \psi _{nlm}  = E \psi _{nlm}
\end{aligned}
\]
代入$L_z = m \hbar$
\begin{equation} \label{eq:enl}
\begin{aligned}
  \begin{pmatrix}
    - \frac{\hbar^2}{2\mu} \nabla^2 +V(r)\end{pmatrix} \psi _{nlm} + \frac{eB}{2 \mu } (m \hbar \pm \hbar) 
    \psi _{nlm}  = E \psi _{nlm}
\end{aligned}
\end{equation}
若无磁场($B=0$),则是中心势场问题,能量为
\[ E_n = - \frac{\mu e_s^4}{2 \hbar^2 n^2}\]
对于类氢原子,离子实对原子核有\emf[库仑屏蔽]作用,体系的能级与角量子数相关,记为$E_{nl}$ (见变分法),方程变为
\[
\begin{aligned}
  \begin{pmatrix}
    - \frac{\hbar^2}{2\mu} \nabla^2 +V(r) 
   \end{pmatrix} \psi _{nlm}  = E_{nl} \psi _{nlm}
\end{aligned}
\]
\end{frame} 

\begin{frame}[label=current]
  \frametitle{}
  代回方程[\ref{eq:enl}],有
  \[\begin{aligned}
    E_{nl} \psi _{nlm} + \frac{eB}{2 \mu } (m \hbar \pm \hbar) 
      \psi _{nlm}  = E \psi _{nlm}
  \end{aligned}\]
  得能级
  \[ E = E_{nl} + \frac{eB\hbar}{2 \mu } (m  \pm 1)  \]
  磁场导致类氢原子能级与磁量子相关,记为 $ E = E_{nlm}  $  
\end{frame} 

\begin{frame}[label=current]
  \frametitle{斯特恩-盖拉赫实验}
  \分 (1)当类氢原子处于s态时($l=0, m=0$),能级为 
  \[E_{nlm} = E_{nl} + \frac{eB\hbar}{2 \mu } (m  \pm 1) = E_{n} \pm \frac{eB\hbar}{2 \mu } = E_{n} \pm M_B B\]
  一个能级分裂成二个能级,正是斯特恩-盖拉赫实验结果
\end{frame} 

\begin{frame}[label=current]
  \frametitle{塞曼效应}
(2)当类氢原子处于非s态时($l\ne 0, m=0,\pm 1, \pm+2, \cdots \pm l$),能级为 
\[E_{nlm} = E_{nl} + \frac{eB\hbar}{2 \mu } (m  \pm 1) \]
光谱频率为
\[
  \begin{aligned}
    \omega &= \frac{E_{nlm} - E_{n'l'm'}}{\hbar} \\
    &=  \frac{E_{nl} - E_{n'l'}}{\hbar} + \frac{eB}{2 \mu } (m-m')\\
    &= \omega _0 + \frac{eB}{2 \mu } \Delta m 
  \end{aligned}
  \]
\end{frame} 

\begin{frame}[label=current]
  \frametitle{}
根据选择定则,有
$$ \Delta m  = \begin{cases}
    +1 \\
    0 \\
    -1
\end{cases} \implies   \omega  = \begin{cases}
  \omega_0 +\frac{eB}{2 \mu }  \\
  \omega _0 \\
  -\omega_0 -\frac{eB}{2 \mu }
\end{cases} $$ 
原来的一条谱线在磁场中劈裂成三条,称为\emf[塞曼效应]。

~~\\ 
塞曼效应是要求磁场很强,因为只有这样,自旋磁矩与轨道磁矩都与磁场平行,可以线性求和。当磁场变弱时,自旋磁矩与轨道磁矩方向不同,不能线性求和,因此必须考虑它们之间的耦合,产生复杂的塞曼效应。
\end{frame} 

\begin{frame}
  \frametitle{旋轨耦合(L-S)}
  \例[6]{考虑旋-轨耦合,求磁场中类氢原子的能级}
  \解 相对论量子力学表明,$L-S$耦合能很小,可以看成微扰
  \[ H' = \frac{1}{2\mu ^2 c^2} \frac{1}{r} \frac{\mathrm{d}V}{\mathrm{d} r} \hat{L}\cdot \hat{S} = \xi(r)\hat{L}\cdot \hat{S} \]
  哈密顿为
  \[ H = H^{(0)} + H' = - \frac{\hbar^2}{2\mu} \nabla^2 +V(r) + \xi(r)\hat{L}\cdot \hat{S} \]
$\hat{L}\cdot \hat{S}$导致不能变量分离,即它们不再与$H$对易,不是守恒量,不能构成力学量最小完全集。 
\end{frame} 

\begin{frame}[label=current]
  \frametitle{}
(I) 考虑变量代换,
\[ \begin{aligned}
  H &= H^{(0)} + \frac{\xi(r)}{2}(2\hat{L}\cdot \hat{S}) \\ 
  &= H^{(0)} + \frac{\xi(r)}{2}( \hat{L}^2 + 2\hat{L}\cdot \hat{S} + \hat{S}^2) - \frac{\xi(r)}{2}(\hat{L}^2 + \hat{S}^2) \\
  &=  H^{(0)} + \frac{\xi(r)}{2} ( \hat{L} + \hat{S})^2 - \frac{\xi(r)}{2}(\hat{L}^2 + \hat{S}^2) \\
  &= H^{(0)} + \frac{\xi(r)}{2} (\hat{J}^2 - \hat{L}^2 - \hat{S}^2) \\
  &= H^{(0)} + \frac{\xi(r)}{2} (\hat{J}^2 - \hat{L}^2 - \frac{3}{4}\hbar^2)
\end{aligned} \]
式中 $\hat{J} = \hat{L} + \hat{S} $,称为总角动量\\

很明显,耦合能变成了 $\hat{J}^2$, $\hat{L}^2$, $\hat{S}^2$ 的线性求和项, 可以分离变量。
\end{frame} 

\begin{frame}[label=current]
  \frametitle{}
(II)  问题转化为求$\hat{J}^2$的本征问题
  \[ \boxed{\hat{J}^2 \psi _j = J^2_j \psi _j}\]
  ~~\\ 
  为方便,记
  $$\hat{J} = \hat{L} + \hat{S} = J_1 + J_2$$ 
  由对易关系
  得
  \[ J_1 \times J_1 = i\hbar J_1 \quad J_2 \times J_2 = i\hbar J_2 \quad [J_1, J_2] =0 \]
\end{frame} 

\begin{frame}[label=current]
  \frametitle{}
可以证明,总角动量有如下对易关系
\[
  \begin{aligned}
   &(1) \quad [J_\alpha,J_\beta ] = \varepsilon_{\alpha\beta\gamma}  i \hbar J_\gamma , \qquad (2)\quad [J_z,J^2 ] =0  \\ 
   &(3) \quad [J^2 _i,J^2 ] = 0 , \qquad~\quad \qquad (4)\quad [J_z, J_i^2 ] =0  \\ 
  \end{aligned} 
  \]
例如:
\[
  \begin{aligned}
  [J_x,J_y ] &= [J_{1x} + J_{2x},J_{1y} + J_{2y}] \\
  &= [J_{1x}, J_{1y} ]+ [J_{1x} ,J_{2y}] + [J_{2x},J_{1y}] + [J_{2x},J_{2y}] \\
  &= i \hbar J_{1z} + 0 + 0 + i \hbar J_{2z} \\
  &= i \hbar J_{z}
  \end{aligned} 
  \]
\end{frame}

\begin{frame}[label=current]
  \frametitle{}
有两组相互对易的算符
\begin{itemize}
  \item 耦合表象 : $J_1^2,J_2^2,J^2, J_z \implies \left\vert j_1, j_2, j, m \right\rangle $
  \item 无耦合表象 $J_1^2,J_{1z},J^2 _2, J_{2z} \implies \left\vert j_1, m_1\right\rangle \left\vert j_2, m_2 \right\rangle $
\end{itemize}
\end{frame} 

\begin{frame}[label=current]
  \frametitle{}
  耦合表象本征解
  \[ \begin{aligned}
    &\hat{J}^2  \left\vert j_1, j_2, j, m \right\rangle = j(j+1)\hbar^2 \left\vert j_1, j_2, j, m \right\rangle \\
    & \hat{J}_z  \left\vert j_1, j_2, j, m \right\rangle = m\hbar \left\vert j_1, j_2, j, m \right\rangle  
  \end{aligned} \]
  耦合表象本征函数
  \[ \begin{aligned}
    \left\vert j_1, j_2, j, m \right\rangle &= \left\vert l, s, j, m \right \rangle \\
    &=  \left\vert l, \frac{1}{2}, j, m \right \rangle = \left\vert l, j, m \right \rangle
  \end{aligned}\]
\end{frame}

\begin{frame}[label=current]
  \frametitle{}
 (III) 耦合表象氢原子波函数
  \[\begin{aligned}
    \Psi _{nlm_lm_s} \implies \Psi _{nljm} &=R_{nl}Y_{ljm}(\theta, \varphi , S_z) \\ 
    &= \left\vert n, l\right \rangle  \left\vert l, j, m \right \rangle
  \end{aligned}\]
  ~~\\ 
无耦合表象氢原子波函数
\[\begin{aligned}
  \Psi _{nlm_lm_s} &=R_{nl}Y_{lm_l}(\theta, \varphi) \chi _{m_s}(S_z) \\ 
  &= \left\vert n, l, m_l, m_s \right \rangle
\end{aligned}\]
\end{frame} 

\begin{frame}[label=current]
  \frametitle{}
  耦合表象下:$m$与$m_1$,$m_2$的关系
  \[J_z = J_{1z} + J_{2z} \to m= m_1 + m_2 = m_l + m_s\]
  耦合表象:给定$j_1, j_2$后, $j$的取值范围
  \[j = |j_1 - j_2|, |j_1 - j_2|+1, \cdots, j_1 + j_2 \]
\end{frame} 

\begin{frame}[label=current]
  \frametitle{}
  (IV) 能量一级修正 \\
  哈密顿解耦
  $$
  H= H^{(0)} + H' = H^{(0)} + \frac{\xi(r)}{2} (\hat{J}^2 - \hat{L}^2 - \frac{3}{4}\hbar^2)  
  $$ 
  本征方程 
  $$
  (H^{(0)} + H')\psi = E \psi  
  $$ 
  波函数在耦合表象展开
$$
\psi = \sum C_{ljm}\left\vert l, j, m \right \rangle
$$ 
系数满足方程简并定态微扰矩阵方程

  \begin{equation*}
    \sum_{ljm}\begin{pmatrix}H^{\prime}_{l'j'm',ljm} -E^{(1)}_n \delta _{l'l} \delta _{j'j}\delta _{m'm} \end{pmatrix}C_{ljm} =0
  \end{equation*}
\end{frame} 

\begin{frame}[label=current]
  \frametitle{}
算微扰矩阵元
\begin{equation*}
\begin{aligned}
  H^{\prime}_{l'j'm',ljm} &=\left\langle n, l', j', m' \right\vert  H^{\prime}  \left\vert n, l,  j, m \right \rangle \\
  &=\left\langle n, l', j', m' \right\vert  \frac{\xi(r)}{2} (\hat{J}^2 - \hat{L}^2 - \frac{3}{4}\hbar^2)  \left\vert n, l,  j, m \right \rangle \\
  &= \frac{1}{2} \left\langle n, l' \right\vert  \xi(r) \left\vert n, l\right \rangle \left\langle l', j', m' \right\vert  (\hat{J}^2 - \hat{L}^2 - \frac{3}{4}\hbar^2)  \left\vert l,  j, m \right \rangle \\
  &= \frac{1}{2} \left\langle n, l' \right\vert  \xi(r) \left\vert n, l\right \rangle \left\langle l', j', m' \right\vert  (j(j+1) - l(l+1) - \frac{3}{4})\hbar^2  \left\vert l,  j, m \right \rangle \\
  &= \frac{\hbar^2}{2} (j(j+1) - l(l+1) - \frac{3}{4}) \xi(r)_{nl, nl} \delta _{l'l} \delta _{j'j}\delta _{m'm}  \\
  &= H'_{nlj} \delta _{l'l} \delta _{j'j}\delta _{m'm}
\end{aligned}
\end{equation*}
式中定义了$$ H'_{nlj} = \frac{\hbar^2}{2} (j(j+1) - l(l+1) - \frac{3}{4}) \xi(r)_{nl, nl} $$ 
代回方程
\end{frame} 

\begin{frame}[label=current]
  \frametitle{}
得:
\begin{equation*}
  \begin{aligned}
    & \sum_{ljm}\begin{pmatrix}H'_{nlj} \delta _{l'l} \delta _{j'l}\delta _{m'm} -E^{(1)}_n \delta _{l'l} \delta _{j'j}\delta _{m'm} \end{pmatrix}C_{ljm} =0 \\ 
    & \sum_{ljm}(H'_{nlj} -E^{(1)}_n) \delta _{l'l} \delta _{j'j}\delta _{m'm} C_{ljm} =0 \\
    & (H'_{nlj} -E^{(1)}_n)C_{ljm} =0 \\
    & (H'_{nlj} -E^{(1)}_n) =0
  \end{aligned}
\end{equation*}
解得能量一级修正
\[E^{(1)}_{nlj} = H'_{nlj} = \frac{\hbar^2}{2} (j(j+1) - l(l+1) - \frac{3}{4}) \xi(r)_{nl, nl} \]
根据量子数$lj$的取值组合,得到针对能级$E^{(0)}_n = E_1 \frac{1}{n^2}$的多个一级修正
\end{frame} 

\begin{frame}[label=current]
  \frametitle{}
  (V) 计算 $\xi(r)_{nl, nl} $\\
  $$V(r) = - \frac{Ze^2}{r}  \implies \frac{\mathrm{d}V}{\mathrm{d} r} = \frac{Ze^2}{r^2} $$ 
  $$\xi(r) = \frac{1}{2\mu ^2 c^2} \frac{1}{r} \frac{\mathrm{d}V}{\mathrm{d} r} = \frac{1}{2\mu ^2 c^2} \frac{Ze^2}{r^3}$$
  \begin{equation*}
    \begin{aligned}
      \xi(r)_{nl, nl} = \left\langle n, l \right\vert  \xi(r) \left\vert n, l \right\rangle 
      &= \int_0^\infty R^2 _{nl}  \xi(r) r^2 dr \\
      &=  \int_0^\infty R^2 _{nl} \frac{1}{2\mu ^2 c^2} \frac{Ze^2}{r^3} r^2 dr \\
      &= \frac{Ze^2}{2\mu ^2 c^2} \int_0^\infty \frac{R^2 _{nl}}{r}  dr \\
      &= \frac{e^2}{2\mu ^2 c^2 a^3 _0} \frac{Z^4}{n^3l(l+\frac{1}{2})(l+1)}
    \end{aligned}
  \end{equation*}
\end{frame} 

\begin{frame}[label=current]
  \frametitle{}
  (VI) 一级修正下的能级
  \[ E_{nlj} = E^{(0)}_{n} + \frac{\hbar^2}{2} (j(j+1) - l(l+1) - \frac{3}{4}) \frac{e^2}{2\mu ^2 c^2 a^3 _0} \frac{Z^4}{n^3l(l+\frac{1}{2})(l+1)}  \]
代入 $a_0 = \frac{\hbar^2}{\mu e^2}, j = l \pm \frac{1}{2}$,并取$\alpha = \frac{e^2}{\hbar c} =\frac{1}{137}$, 得
\[ \left\{\begin{aligned}
  E_{nl,j=l+\frac{1}{2}} &= E^{(0)}_{n} + \frac{\mu c^2}{2}( \frac{\alpha Z}{n})^4 \frac{n}{(2l+1)(l+1)} \\ 
  E_{nl,j=l-\frac{1}{2}} &= E^{(0)}_{n} - \frac{\mu c^2}{2}( \frac{\alpha Z}{n})^4 \frac{n}{(2l+1)l} \\
\end{aligned}\right. \] 
由于总角量子数$j$的不同,一条谱线在弱磁场作用下分裂成两条, 称为复杂塞曼效应。
\end{frame} 

\begin{frame}[label=current]
  \frametitle{光谱线精细结构}
  对给定的 $n,l$ 值, 总角量子数$j=l\pm \frac{1}{2}$有两个值,但由于$\xi(r)$能常很小,这两个能级间距非常小, 跃迁产生的谱线靠得非常近,称为光谱线的精细结构
  \begin{figure}[htbp]
    \centering
    \includegraphics[width=0.45\textwidth]{figs/naspr.png}
    %\caption{}
      %\label{fig:}
  \end{figure}
  钠原子光谱中的一条亮黄线$5893$ \AA 在弱磁场出劈裂为 $5890$ \AA 和 $5896$ \AA 两条,只有用高分辨率的光谱仪才能观测得到。 
\end{frame} 


\begin{frame}[label=current]
  \frametitle{课外作业}
  1、求   $\sigma _y  =  \begin{pmatrix}
    0 & -i\\
    i & 0
   \end{pmatrix} $ 的本征问题 \\

  2、设电子处于态 $\psi=\begin{pmatrix}
    a\\
    b
   \end{pmatrix}  $, 求测量得电子自旋向上的概率 \\

  3、对于自旋为$\frac{1}{2}$的粒子,在磁场 $\mathbf{B} = B_x \vec{e}_x + B_y \vec{e}_y + B_z \vec{e}_z$中的哈密顿为
  \[ H = - \frac{2\mu}{\hbar}\mathbf{B}\cdot \mathbf{S}\]
  试求能量本征值 \\
\end{frame} 


\begin{frame}[label=current]
  \frametitle{}
  4、试证明 \[ [\mathbf{J}^2, J_z] =0, \qquad [\mathbf{J}^2, J_{1z}] \ne 0\]
  5、假设轨道角动量允许存在半整数的$l$量子数,比如$l=\frac{1}{2}$\\
    1)试证明函数 $Y_{1/2,1/2}(\theta, \varphi) = c_{\frac{1}{2}}e^{i \varphi/2} \sqrt{sin\theta}$ 满足方程 $$ L_+ Y_{1/2,1/2}(\theta, \varphi) =0 $$
    2)试求由$L_{-}$作用于$Y_{1/2,1/2}(\theta, \varphi)$所产生的$Y_{1/2,-1/2}(\theta, \varphi)$\\
    3)求满足方程$$ L_{-}Y_{1/2,-1/2}(\theta, \varphi) =0 $$ 的函数 $Y_{1/2,-1/2}(\theta, \varphi)$ \\ (这两个函数不可能相同,是轨道量子数不能取半奇整数的一个论据) 
\end{frame} 

\section{全同粒子体系 }
\subsection{全同性原理}
\begin{frame}[label=current]
  \frametitle{全同粒子}
  \alert{定义:} 所有\emf[固有属性] 都相同的粒子称为全同粒子。比如:一个体系中的粒子都是电子, 构成一种全同粒子体系; 所有光子, 固有属性都相同,也构成一种全同粒子体系。 
\begin{figure}[htbp]
  \centering
  \includegraphics[width=0.45\textwidth]{figs/indis.png}
\end{figure}

\alert{特点:不可区分性}

经典力学的全同粒子可通过位置和轨迹等信息进行区分,称为定域体系,量子力学没有轨道的概念, 在全同粒子波函数重叠区, 也没有确定的位置信息, 因此无法区分,称为非定域体系。比如:电子双缝干涉实验 
\end{frame} 

\begin{frame}[label=current]
  \frametitle{全同性原理}
    \begin{atcbox}
      全同粒子体系的状态不会因为任意两全同粒子的交换而发生改变,这种交换不变性(对称性)称为全同性原理。数学上体现为,交换前后体系波函数的概率分布相等
      \[\left\vert\psi(\vec{q}_1,\cdots, \vec{q}_{\color{red}i},\cdots,\vec{q}_{\color{red}j},\cdots,\vec{q}_N,t)\right\vert ^2 = \left\vert\psi(\vec{q}_1,\cdots, \vec{q}_{\color{red}j},\cdots,\vec{q}_{\color{red}i},\cdots,\vec{q}_N,t)\right\vert ^2 \]
    \end{atcbox} 
    全同性原理是量子力学中的基本原理之一,不能推导,只能实验验证。
\end{frame}
 
\subsection{全同粒子体系的波函数}

\begin{frame}[label=current]
  \frametitle{波函数特性}
  \alert{特性-1:} 全同粒子体系波函数要么是交换对称的要么是交换反对称的 \\
\证 设体系含N个全同粒子,哈密顿为
\[ H(\vec{q}_1,\vec{q}_2,\cdots,\vec{q}_N) = \sum_{i=1}^{N}\left[-\frac{\hbar}{2\mu} \nabla^2_i +V(\vec{q}_i , t) \right] + \sum_{i<j}^N U(\vec{q}_i, \vec{q}_j)\]
很明显:两粒子互换,哈密顿量是不变的 \\ 
交换前的薛定谔方程为:
\[i\hbar \frac{\partial }{\partial t} \psi(\vec{q}_1,\cdots, \vec{q}_{\color{red}i},\cdots,\vec{q}_{\color{red}j},\cdots,\vec{q}_N,t)=H(\vec{q}_1,\vec{q}_2,\cdots,\vec{q}_N)\psi(\vec{q}_1,\cdots, \vec{q}_{\color{red}i},\cdots,\vec{q}_{\color{red}j},\cdots,\vec{q}_N,t) \]
\end{frame} 

\begin{frame}[label=current]
  \frametitle{}
交换后,有
\[i\hbar \frac{\partial }{\partial t} \psi(\vec{q}_1,\cdots, \vec{q}_{\color{red}j},\cdots,\vec{q}_{\color{red}i},\cdots,\vec{q}_N,t)=H(\vec{q}_1,\vec{q}_2,\cdots,\vec{q}_N)\psi(\vec{q}_1,\cdots, \vec{q}_{\color{red}j},\cdots,\vec{q}_{\color{red}i},\cdots,\vec{q}_N,t) \]
即交换前后的波函数满足同一个方程,又因交换前后的波函数描述同一个态, 因此它们之间只多相差一个常数因子
\[\psi(\vec{q}_1,\cdots, \vec{q}_{\color{red}i},\cdots,\vec{q}_{\color{red}j},\cdots,\vec{q}_N,t) = \lambda \psi(\vec{q}_1,\cdots, \vec{q}_{\color{red}j},\cdots,\vec{q}_{\color{red}i},\cdots,\vec{q}_N,t)\]
再交换一次,则有
\[\psi(\vec{q}_1,\cdots, \vec{q}_{\color{red}i},\cdots,\vec{q}_{\color{red}j},\cdots,\vec{q}_N,t) = \lambda ^2 \psi(\vec{q}_1,\cdots, \vec{q}_{\color{red}i},\cdots,\vec{q}_{\color{red}j},\cdots,\vec{q}_N,t)\]
\end{frame} 

\begin{frame}[label=current]
  \frametitle{}
得
\[ \lambda ^2 =1 \implies \lambda = \pm 1\]
当$\lambda = 1$时,交换前后的波函数相同,称波函数是交换对称的 \\
当$\lambda = -1$时,交换前后的波函数反号,称波函数是交换反对称的。 

~~\\ 
\textcolor{red}{证毕!}
\end{frame} 

\begin{frame}[label=current]
  \frametitle{}
  \alert{特性-2:} 全同粒子体系波函数的交换对称性不随时间发生改变\\ 
  \证 设$t$时刻波函数是交换对称的
  \[ \psi(t) = \psi _s(t)\]
  代入薛定谔方程
  \[i\hbar \frac{\partial }{\partial t} \psi _s (t ) =H \psi _s (t ) \]
  式左两因子都是交换对称的,所以式右的因子$\frac{\partial }{\partial t} \psi _s (t )$也是交换对称的 \\
  $t+dt$时刻的波函数
  \[ \psi(t+ dt) =\psi (t) + \frac{\partial }{\partial t} \psi (t )= \psi _s(t) + \frac{\partial }{\partial t} \psi _s (t ) \]
  因此,$\psi(t+ dt)$是交换对称的,即它的交换对称性不随时间变化。\\
  同理可证交换反对称波函数满足相同的结论。
\end{frame} 

\begin{frame}[label=current]
  \frametitle{玻色子与费米子}
  \alert{定义:} 波函数交换对称的全同粒子称为玻色子, 波函数交换反对称的全同粒子称为费米子。 记为 :
  \[ \left\{\begin{aligned}
    & P_{ij}\left\vert \text{N个玻色子} \right\rangle = + \left\vert \text{N个玻色子} \right\rangle \\
    & P_{ij}\left\vert \text{N个费米子} \right\rangle = - \left\vert \text{N个费米子} \right\rangle \\ 
  \end{aligned}\right.
  \]
  \alert{特点:} 
  \begin{itemize}
    \item 玻色子的自旋量子数为$\frac{1}{2}$的偶数倍,服从玻色-爱因斯坦统计规律
    \item 费米子的自旋量子数为$\frac{1}{2}$的奇数倍,服从费米-狄拉克统计规律
  \end{itemize}     
\end{frame} 

\begin{frame}[label=current]
  \frametitle{}
  \alert{对易关系:} \\
粒子场基态 
\[ \left\vert 0,0,0,\cdots ,0 \right\rangle = \left\vert \mathbf{0} \right\rangle\]
单粒子态 (占据第$i$本征态)
\[\left\vert 0,0,0,\cdots, n_i =1 ,\cdots, 0 \right\rangle  = \left\vert k_i \right\rangle\]
很明显,有
\[ a_i \left\vert k_j \right\rangle = \delta _{ij} \left\vert \mathbf{0} \right\rangle \]
\end{frame} 

\begin{frame}[label=current]
  \frametitle{}
  双粒子态
  \[\left\vert 0,0,0,\cdots, n_i =1 ,\cdots, n_j =1 ,\cdots, 0 \right\rangle = \left\vert k_{ij} \right\rangle\]
  有
  \[ a_i a_j \left\vert k_{ij} \right\rangle = \left\vert \mathbf{0} \right\rangle \]
  考虑置换
  \[ 
    a_j a_i \left\vert k_{ij} \right\rangle = a_j a_i \pm \left\vert k_{ji} \right\rangle
    = \pm  \left\vert \mathbf{0} \right\rangle  =  \pm  a_i a_j \left\vert k_{ij} \right\rangle
  \] 
  得对易关系
\[\left\{ \begin{aligned}
  &a_i  a_j  - a_j  a_i  = [a_i , a_j ] = 0 \qquad \text{玻色子} \\
  &a_i  a_j  + a_j  a_i  = \{ a_i , a_j \} = 0 \quad \,\,\, \text{费米子} 
\end{aligned}\right.\]
\end{frame} 

\begin{frame}[label=current]
  \frametitle{}
同理,由
\[ a_i^{\dagger} \left\vert \mathbf{0} \right\rangle = \delta _{ij} \left\vert  k_j \right\rangle, \qquad a_i^{\dagger} a_j^{\dagger} \left\vert \mathbf{0} \right\rangle = \left\vert  k_{ij} \right\rangle \]
可以导出
\[\left\{ \begin{aligned}
  &a_i ^{\dagger} a_j ^{\dagger} - a_j ^{\dagger} a_i ^{\dagger} = [a_i ^{\dagger}, a_j ^{\dagger}] = 0 \qquad \text{玻色子} \\
  &a_i ^{\dagger} a_j ^{\dagger} + a_j ^{\dagger} a_i ^{\dagger} = \{ a_i ^{\dagger}, a_j ^{\dagger}\} = 0 \quad \,\,\, \text{费米子} 
\end{aligned}\right.\]
进一步,可得
\[\left\{ \begin{aligned}
  &a_i  a_j ^{\dagger} - a_j ^{\dagger} a_i  = [a_i , a_j ^{\dagger}] = \delta _{ij} \qquad \text{玻色子} \\
  &a_i  a_j ^{\dagger} + a_j ^{\dagger} a_i = \{ a_i , a_j ^{\dagger}\} = \delta _{ij} \quad \,\,\, \text{费米子} 
\end{aligned}\right.\]

\end{frame} 

\begin{frame}[label=current]
  \frametitle{}
\emf[泡利不相容原理:] \\ 
在费米子对易关系中,取$i=j$ (双粒子占据同一本征态)
\[ \begin{aligned}
  a_i  a_j ^{\dagger} + a_j ^{\dagger} a_i &=  \delta _{ij}  \\
  a_i  a_i ^{\dagger} + a_i ^{\dagger} a_i &=  1  \\
\end{aligned}\]
考虑对易关系  
\[ [a_i, a_i ^{\dagger}] = 1\]
得
\[ n =  a_i ^{\dagger} a_i =0 \]
即:不可能出现两全同费米子同时占据同一个态
\end{frame} 

\begin{frame}[label=current]
  \frametitle{波函数$Hartree-Fock$积形式}
1、两粒子体系的波函数 \\
有哈密顿
\[ H(\vec{q}_1,\vec{q}_2) = H_0(\vec{q}_1) + H_0(\vec{q}_2) +  U(\vec{q}_1, \vec{q}_2)\]
定态薛定谔方程
\[ H(\vec{q}_1,\vec{q}_2) \left \vert\psi(\vec{q}_1,\vec{q}_2) \right\rangle = E \left\vert \psi(\vec{q}_1,\vec{q}_2) \right\rangle \]
当相互作用能很小,小到可以忽略不计时,有
\[ H(\vec{q}_1,\vec{q}_2) = H_0(\vec{q}_1) + H_0(\vec{q}_2) \]
称为近独立全同粒子体系,此时,定态薛定谔方程可分离变量,
体系的本征能可表示为单粒子本征能的求和形式
  \[ E= \varepsilon _i + \varepsilon _j\]
\end{frame} 

\begin{frame}[label=current]
  \frametitle{}
体系的本征态可表示为单粒子本征态的$Hartree$积形式
\[\left\vert \psi(\vec{q}_1,\vec{q}_2) \right\rangle  = \left\vert \psi _i(\vec{q}_1)\right\rangle \left\vert \psi _j(\vec{q}_2)\right\rangle \]
交换后的本征态为 
  \[\left\vert \psi(\vec{q}_2,\vec{q}_1) \right\rangle  = \left\vert \psi _i(\vec{q}_2)\right\rangle \left\vert \psi _j(\vec{q}_1)\right\rangle \]
当两粒子处于不同的态时($i \ne j$),上述两个波函数既不是对称的也不是反对称的!违反全同性原理!
\end{frame} 

\begin{frame}[label=current]
  \frametitle{}
$Fock$发现,如果用 $Hartree$积的和差来表示体系的本征矢,则不违反全同性原理!
\begin{itemize}
  \item 对称 (玻色系统)\[\left\vert \psi _S(\vec{q}_1,\vec{q}_2) \right\rangle  \to \left[\left\vert \psi _i(\vec{q}_1)\right\rangle \left\vert \psi _j(\vec{q}_2) \right\rangle  + \left\vert \psi _i(\vec{q}_2)\right\rangle \left\vert \psi _j(\vec{q}_1) \right\rangle\right] \]
  \item 反对称 (费米系统)\[\left\vert \psi _A (\vec{q}_1,\vec{q}_2) \right\rangle  \to \left[\left\vert \psi _i(\vec{q}_1)\right\rangle \left\vert \psi _j(\vec{q}_2) \right\rangle  - \left\vert \psi _i(\vec{q}_2)\right\rangle \left\vert \psi _j(\vec{q}_1) \right\rangle\right] \]
\end{itemize}
这种表示法,称为全同粒子体系波函数的$Hartree-Fock$积形式。
\end{frame}  

\begin{frame}[label=current]
  \frametitle{}
\例[7]{试证明泡利不相容原理}
\证 (1)根据全同性原理,双玻色系统的波函数由如下项构成 
\[\left\vert \psi _S(\vec{q}_1,\vec{q}_2) \right\rangle  \to \left\vert \psi _i(\vec{q}_1)\right\rangle \left\vert \psi _j(\vec{q}_2) \right\rangle  + \left\vert \psi _i(\vec{q}_2)\right\rangle \left\vert \psi _j(\vec{q}_1) \right\rangle \]
设两粒子处于同一个态($i = j$),有
\[
  \begin{aligned}
\left\vert \psi _i(\vec{q}_1)\right\rangle \left\vert \psi _i(\vec{q}_2) \right\rangle  + \left\vert \psi _i(\vec{q}_2)\right\rangle \left\vert \psi _i(\vec{q}_1) \right\rangle= \left\vert \psi _i(\vec{q}_1)\right\rangle \left\vert \psi _i(\vec{q}_2) \right\rangle 
  \end{aligned} \]
  这种项不为零,整个波函数的模方也不会零,即两个玻色子同时处于同一个态是允许的
\end{frame} 

\begin{frame}[label=current]
  \frametitle{}
  (2)根据全同性原理,双费米系统的波函数为 
  \[\left\vert \psi _A(\vec{q}_1,\vec{q}_2) \right\rangle  \to \left[\left\vert \psi _i(\vec{q}_1)\right\rangle \left\vert \psi _j(\vec{q}_2) \right\rangle  - \left\vert \psi _i(\vec{q}_2)\right\rangle \left\vert \psi _j(\vec{q}_1) \right\rangle\right] \]
  设两粒子处于同一个态($i = j$),有
  \[
  \begin{aligned}
    \left\vert \psi _A(\vec{q}_1,\vec{q}_2) \right\rangle  \to \left[\left\vert \psi _i(\vec{q}_1)\right\rangle \left\vert \psi _i(\vec{q}_2) \right\rangle  - \left\vert \psi _i(\vec{q}_2)\right\rangle \left\vert \psi _i(\vec{q}_1) \right\rangle\right]  = 0 
  \end{aligned} \]
  两个全同费米子同时处于同一个态的概率等于零,即是不允许的 \\
  \textcolor{red}{证毕!}
\end{frame} 

\begin{frame}[label=current]
  \frametitle{}
2、N玻色子体系的波函数 \\
~~\\ 
对于N玻色体系来说,多粒子占据同一态是允许的,波函数应是N个粒子占据k个态排列项求和:
\[\begin{aligned}
\left\vert \psi _S(\vec{q}_1,\vec{q}_2,\cdots \vec{q}_N) \right\rangle  
 &= \sum_P P \left\vert n_1 n_2\cdots n_k\right\rangle 
\end{aligned}  \]
式中:$n_1 + n_2 + \cdots + n_k =N $
\end{frame} 

\begin{frame}[label=current]
  \frametitle{}
我们知道,N粒子占据K个态的排列项数目为
$$\frac{N!}{n_1 !n_2 ! \cdots n_k !} = \frac{N!}{\prod\limits_{l=1}^{k}n_l !} $$ 
因此,归一化的N玻色体系波函数为
\[
  \left\vert \psi _S(\vec{q}_1,\vec{q}_2,\cdots \vec{q}_N) \right\rangle  
 = \sqrt{\left.\prod\limits_{l=1}^{k}n_l ! \right/ N!} \sum_P P\left\vert n_1 n_2\cdots n_k\right\rangle  \]
\end{frame} 

\begin{frame}[label=current]
  \frametitle{}
  3、N费米子体系的波函数 \\
  考察双费米体系的波函数的和差项,可以写成行列式形式
  \[\begin{aligned}
    \left\vert \psi _A(\vec{q}_1,\vec{q}_2) \right\rangle  &= \frac{1}{\sqrt{2} }\left[\left\vert \psi _i(\vec{q}_1)\right\rangle \left\vert \psi _j(\vec{q}_2) \right\rangle  - \left\vert \psi _i(\vec{q}_2)\right\rangle \left\vert \psi _j(\vec{q}_1) \right\rangle\right] \\ 
    &= \frac{1}{\sqrt{2}} \begin{vmatrix}\left\vert \psi _i(\vec{q}_1)\right\rangle  &\left\vert \psi _i(\vec{q}_2)\right\rangle \\ 
    \left\vert \psi _j(\vec{q}_1) \right\rangle & \left\vert \psi _j(\vec{q}_2)\right\rangle 
  \end{vmatrix}  
  \end{aligned} \]
  那么,三费米体系应为 
  \[\begin{aligned}
    \left\vert \psi _A(\vec{q}_1,\vec{q}_2,\vec{q}_3) \right\rangle  
    &= \frac{1}{\sqrt{6}} \begin{vmatrix}\left\vert \psi _i(\vec{q}_1)\right\rangle  &\left\vert \psi _i(\vec{q}_2)\right\rangle &\left\vert \psi _i(\vec{q}_3)\right\rangle \\ 
    \left\vert \psi _j(\vec{q}_1) \right\rangle & \left\vert \psi _j(\vec{q}_2)\right\rangle & \left\vert \psi _j(\vec{q}_3)\right\rangle \\
    \left\vert \psi _k(\vec{q}_1) \right\rangle & \left\vert \psi _k(\vec{q}_2)\right\rangle & \left\vert \psi _k(\vec{q}_3)\right\rangle 
  \end{vmatrix}  
  \end{aligned} \]
\end{frame} 

\begin{frame}[label=current]
  \frametitle{}
推广到$N$费米体系
\[\begin{aligned}
  \left\vert \psi _A(\vec{q}_1,\vec{q}_2,\cdots,\vec{q}_N) \right\rangle  
  &= \frac{1}{\sqrt{N!}} \begin{vmatrix}\left\vert \psi _1(\vec{q}_1)\right\rangle  &\left\vert \psi _1(\vec{q}_2)\right\rangle &\cdots &\left\vert \psi _1(\vec{q}_N)\right\rangle \\ 
  \left\vert \psi _2(\vec{q}_1) \right\rangle & \left\vert \psi _2(\vec{q}_2)\right\rangle & \cdots &\left\vert \psi _2(\vec{q}_N)\right\rangle \\
  \cdots & \cdots & \cdots & \cdots \\
  \left\vert \psi _k(\vec{q}_1) \right\rangle & \left\vert \psi _k(\vec{q}_2)\right\rangle & \cdots & \left\vert \psi _k(\vec{q}_N)\right\rangle 
\end{vmatrix}  
\end{aligned} \]
考察:
\begin{itemize}
  \item 交换两粒子,相当于行列式交换两列,行列式变号,满足交换反对称要求
  \item 如果有两粒子处于同一状态,则行列式有两行相等,波函数为零,满足泡利不相容原理要求
\end{itemize}
\end{frame}

\subsection{全同粒子体系的能级}

\begin{frame}[label=current]
  \frametitle{}
\例[8]{一个体系由三个全同费米子构成,粒子间相互作用可忽略,单粒子有三个本征态为$\psi _1, \psi _2,\psi _3 $,对应的能级为1.2 eV, 1.2 eV, 1.5 eV,试求 \\
(1) 体系的波函数,能级及简并度 \\
(2) 若体系只包含两个全同费米子呢\\
(3) 若体系包含的三个全同玻色子呢}
\解 (1)三个费米子分别占据一个态,占据数分布为$\{111\}$,
波函数为 
\[\begin{aligned}
  \left\vert \psi ^{111} _A(\vec{q}_1,\vec{q}_2,\vec{q}_3) \right\rangle  
  &= \frac{1}{\sqrt{3!}} \begin{vmatrix}\left\vert \psi _1(\vec{q}_1)\right\rangle  &\left\vert \psi _1(\vec{q}_2)\right\rangle &\left\vert \psi _1(\vec{q}_3)\right\rangle \\ 
  \left\vert \psi _2(\vec{q}_1) \right\rangle & \left\vert \psi _2(\vec{q}_2)\right\rangle & \left\vert \psi _2(\vec{q}_3)\right\rangle \\
  \left\vert \psi _3(\vec{q}_1) \right\rangle & \left\vert \psi _3(\vec{q}_2)\right\rangle & \left\vert \psi _3(\vec{q}_3)\right\rangle 
\end{vmatrix}  
\end{aligned} \]
能级 
\[ E_1 = E^{111} = 1.2+1.2+1.5 =3.9~ (\text{eV}) \qquad \text{简并度:} ~1\]
\end{frame} 

\begin{frame}[label=current]
  \frametitle{}
(2)两个费米子可占据三个态,可能的占据数分布$\{110, 101, 011\}$\\
波函数:
\[\begin{aligned}
  \left\vert \psi ^{110}_A(\vec{q}_1,\vec{q}_2) \right\rangle 
  &= \frac{1}{\sqrt{2}} \begin{vmatrix}\left\vert \psi _1(\vec{q}_1)\right\rangle  &\left\vert \psi _1(\vec{q}_2)\right\rangle \\ 
  \left\vert \psi _2(\vec{q}_1) \right\rangle & \left\vert \psi _2(\vec{q}_2)\right\rangle 
\end{vmatrix}  \\
\left\vert \psi ^{101}_A(\vec{q}_1,\vec{q}_2) \right\rangle 
  &= \frac{1}{\sqrt{2}} \begin{vmatrix}\left\vert \psi _1(\vec{q}_1)\right\rangle  &\left\vert \psi _1(\vec{q}_2)\right\rangle \\ 
  \left\vert \psi _3(\vec{q}_1) \right\rangle & \left\vert \psi _3(\vec{q}_2)\right\rangle 
\end{vmatrix}  \\
\left\vert \psi ^{011}_A(\vec{q}_1,\vec{q}_2) \right\rangle 
  &= \frac{1}{\sqrt{2}} \begin{vmatrix}\left\vert \psi _2(\vec{q}_1)\right\rangle  &\left\vert \psi _2(\vec{q}_2)\right\rangle \\ 
  \left\vert \psi _3(\vec{q}_1) \right\rangle & \left\vert \psi _3(\vec{q}_2)\right\rangle 
\end{vmatrix}  
\end{aligned} \]
能量 :$$ \begin{aligned} E^{110} &= 1.2+1.2 = 2.4~ (\text{eV}) \\ 
  E^{101} &= 1.2+1.5 = 2.7~ (\text{eV}) \\
  E^{011} &= 1.2+1.5 = 2.7~ (\text{eV}) 
\end{aligned} $$
\end{frame}

\begin{frame}[label=current]
  \frametitle{}
能级:
$$ \begin{aligned} E_1&= 2.4~ (\text{eV}), \qquad \text{简并度:} ~1  \\ 
  E_2&= 2.7~ (\text{eV}), \qquad \text{简并度:} ~2  \\
\end{aligned} $$

~~\\ 
(3)三个玻色子可占据三个态,可能的占据数分布共有10个
$$300, 030, 003; 210, 201, 120, 102, 021,  012, 111 $$
波函数:
\[\begin{aligned}
  \left\vert \psi ^{300}_S(\vec{q}_1,\vec{q}_2,\vec{q}_3) \right\rangle 
  &= \sqrt{\frac{3!0!0!}{3!}} \left\vert \psi _1(\vec{q}_1)\right\rangle \left\vert \psi _1(\vec{q}_2)\right\rangle \left\vert \psi _1(\vec{q}_3)\right\rangle\\ 
  \left\vert \psi ^{030}_S(\vec{q}_1,\vec{q}_2,\vec{q}_3) \right\rangle 
  &= \left\vert \psi _2(\vec{q}_1)\right\rangle \left\vert \psi _2(\vec{q}_2)\right\rangle \left\vert \psi _2(\vec{q}_3)\right\rangle\\ 
  \left\vert \psi ^{003}_S(\vec{q}_1,\vec{q}_2,\vec{q}_3) \right\rangle 
  &=\left\vert \psi _3(\vec{q}_1)\right\rangle \left\vert \psi _3(\vec{q}_2)\right\rangle \left\vert \psi _3(\vec{q}_3)\right\rangle\\ 
\end{aligned} \]
\end{frame} 

\begin{frame}[label=current]
  \frametitle{}
  波函数:
  \[\begin{aligned}
    \left\vert \psi ^{210}_S(\vec{q}_1,\vec{q}_2,\vec{q}_3) \right\rangle 
    &= \sqrt{\frac{2!1!0!}{3!}} \left[\left\vert \psi _1(\vec{q}_1)\right\rangle \left\vert \psi _1(\vec{q}_2)\right\rangle \left\vert \psi _2(\vec{q}_3)\right\rangle 
    + \left\vert \psi _1(\vec{q}_1)\right\rangle \left\vert \psi _1(\vec{q}_3)\right\rangle \left\vert \psi _2(\vec{q}_2)\right\rangle \right. \\
    &  \left.\qquad + \left\vert \psi _1(\vec{q}_2)\right\rangle \left\vert \psi _1(\vec{q}_3)\right\rangle \left\vert \psi _2(\vec{q}_1)\right\rangle
    \right]\\  
    \left\vert \psi ^{120}_S(\vec{q}_1,\vec{q}_2,\vec{q}_3) \right\rangle 
    &= \frac{1}{\sqrt{3}} \left[\left\vert \psi _1(\vec{q}_1)\right\rangle \left\vert \psi _2(\vec{q}_2)\right\rangle \left\vert \psi _2(\vec{q}_3)\right\rangle 
    + \left\vert \psi _1(\vec{q}_2)\right\rangle \left\vert \psi _2(\vec{q}_2)\right\rangle \left\vert \psi _2(\vec{q}_3)\right\rangle \right. \\
    &  \left.\qquad + \left\vert \psi _1(\vec{q}_3)\right\rangle 
    + \left\vert \psi _2(\vec{q}_2)\right\rangle \left\vert \psi _2(\vec{q}_3)\right\rangle 
    \right]\\
    \left\vert \psi ^{012}_S(\vec{q}_1,\vec{q}_2,\vec{q}_3) \right\rangle 
    &= \frac{1}{\sqrt{3}} \left[\left\vert \psi _2(\vec{q}_1)\right\rangle \left\vert \psi _3(\vec{q}_2)\right\rangle \left\vert \psi _3(\vec{q}_3)\right\rangle 
    + \left\vert \psi _2(\vec{q}_2)\right\rangle \left\vert \psi _3(\vec{q}_1)\right\rangle \left\vert \psi _3(\vec{q}_3)\right\rangle \right. \\
    &  \left.\qquad + \left\vert \psi _2(\vec{q}_3)\right\rangle 
    + \left\vert \psi _3(\vec{q}_2)\right\rangle \left\vert \psi _3(\vec{q}_1)\right\rangle 
    \right] \\
    & \cdots 
  \end{aligned} \]
\end{frame} 

\begin{frame}[label=current]
  \frametitle{}
能量:
\[\begin{aligned}
  E^{300} &=  3\times 1.2 = 3.6  ~ (\text{eV}) \\
  E^{030} &=  3\times 1.2 = 3.6  ~ (\text{eV}) \\
  E^{003} &=  3\times 1.5 = 4.5  ~ (\text{eV})  \\
  E^{210} &=  2\times 1.2 +1.2 = 3.6  ~ (\text{eV})  \\
  E^{201} &=  2\times 1.2 +1.5 = 3.9  ~ (\text{eV})  \\
  E^{120} &=  1.2+ 2\times 1.2 = 3.6  ~ (\text{eV})  \\
  E^{021} &=  2\times 1.2 + 1.5= 3.9  ~ (\text{eV})  \\
  E^{102} &=  1.2+2\times 1.5= 4.2  ~ (\text{eV})  \\
  E^{012} &=  1.2+2\times 1.5= 4.2  ~ (\text{eV})  \\
  E^{111} &=  1.2+1.2+1.5= 3.9  ~ (\text{eV})  \\
\end{aligned} \]
\end{frame} 

\begin{frame}[label=current]
  \frametitle{}
  能级:
  $$ \begin{aligned} E_1&= 3.6~ (\text{eV}), \qquad \text{简并度:} ~4  \\ 
    E_2&= 3.9~ (\text{eV}), \qquad \text{简并度:} ~3  \\
    E_3&= 4.2~ (\text{eV}), \qquad \text{简并度:} ~2  \\
    E_4&= 4.5~ (\text{eV}), \qquad \text{简并度:} ~1  \\
  \end{aligned} $$
* 玻色子体系低能级简并度高!

  ~~\\ 
  \textcolor{red}{结束!}
\end{frame} 

\subsection{自旋三重态}

\begin{frame}[label=current]
  \frametitle{双电子系统}
现在考虑这样一个双费子全同粒子体系,比如氢分子或氦原子中的两个电子,由于体系很稳定,可设原子核不动,因此两个电子构成双电子系统
\begin{figure}[htbp]
  \centering
  \includegraphics[width=0.3\textwidth]{figs/he1.png}
  %\caption{}
    %\label{fig:}
\end{figure}
\end{frame} 

\begin{frame}[label=current]
  \frametitle{}
哈密顿: 
  \[
    H=\frac{\mathbf{p}_1^2}{2 m}+\frac{\mathbf{p}_2^2}{2 m}-\frac{2 e^2}{r_1}-\frac{2 e^2}{r_2}+\frac{e^2}{r_{12}}
    \]
电子间库仑相互作用能很大,不可忽略,不能做近独立粒子体系处理。\\
~~\\ 
为简单计,不考虑旋轨耦合,体系的波函数可写成$Hartree$积形式
\[ \Psi(\vec{q}_1, \vec{q}_2) = \Psi(\vec{r}_1, S_{1z}, \vec{r}_2, S_{2z}) = \psi(\vec{r}_1, \vec{r}_2) \chi (S_{1z}, S_{2z})\]
对于全同费米子体系,只有交换反对称的$Hartree$积才是合法的
\end{frame} 

\begin{frame}[label=current]
  \frametitle{}
 分为两类:
  \begin{itemize}
    \item 若 $\psi(\vec{r}_1, \vec{r}_2)$交换对称, 则 $\chi (S_{1z}, S_{2z})  $ 必须交换反对称
    \item 若 $\psi(\vec{r}_1, \vec{r}_2)$交换反对称, 则 $\chi (S_{1z}, S_{2z}) $ 必须交换对称
  \end{itemize}

  ~~\\ 
  ${\color{red}\star}~$ 单就自旋波函数$\chi (S_{1z}, S_{2z}) $来说,它可以是对称的,也可以是反对称的
\end{frame} 

\begin{frame}[label=current]
  \frametitle{自旋波函数}
  电子间的库仑相互作用已体现在位置函数$\psi(\vec{r}_1, \vec{r}_2)$中,若不考虑电子间的自旋相互作用,自旋波函数可再分离变量
  $$\chi (S_{1z}, S_{2z}) = \chi _{\pm \frac{1}{2}} (S_{1z})\chi_{\pm \frac{1}{2}} (S_{2z}) $$
  可构造出四个满足要求的自旋波函数(一个反对称,三个对称)
  \[ 
    \begin{aligned}
      \chi ^{(1)} _S &= \chi _{ \frac{1}{2}} (S_{1z})\chi_{ \frac{1}{2}} (S_{2z})  \qquad (\text{都自旋向上})\\ 
      \chi ^{(2)} _S &= \chi _{ -\frac{1}{2}} (S_{1z})\chi_{ -\frac{1}{2}} (S_{2z})  \qquad (\text{都自旋向下}) \\ 
      \chi ^{(3)} _S &= \frac{1}{\sqrt{2} }\left[\chi _{ \frac{1}{2}} (S_{1z})\chi_{ -\frac{1}{2}} (S_{2z})  + \chi _{ -\frac{1}{2}} (S_{1z})\chi_{\frac{1}{2}} (S_{2z})\right] \\ 
      \chi ^{(4)} _A &= \frac{1}{\sqrt{2} }\left[\chi _{ \frac{1}{2}} (S_{1z})\chi_{ -\frac{1}{2}} (S_{2z})  - \chi _{ -\frac{1}{2}} (S_{1z})\chi_{\frac{1}{2}} (S_{2z})\right] 
    \end{aligned}\]
\end{frame} 

\begin{frame}[label=current]
  \frametitle{自旋算符的值}
  先计算单电子自旋波函数的各种自旋算符的值 \\
  (1) $\chi$是 $S^2$和$S_z$的同共本征态
  \[\begin{aligned}
    S_z \chi_{\frac{1}{2}} &= \frac{1}{2}\hbar \chi_{\frac{1}{2}}  \\
    S^2 \chi_{\frac{1}{2}} &= \frac{3}{4}\hbar^2 \chi_{\frac{1}{2}}\\
    S_z \chi_{-\frac{1}{2}}&=-\frac{1}{2}\hbar \chi_{-\frac{1}{2}}  \\
    S^2 \chi_{-\frac{1}{2}}&= \frac{3}{4}\hbar^2 \chi_{-\frac{1}{2}}
  \end{aligned}\]
\end{frame} 

\begin{frame}[label=current]
  \frametitle{}
  (2) $\chi$ 不是 $S_x$和$S_y$的本征态
  \[\begin{aligned}
    S_x \chi_{\frac{1}{2}} &= \frac{\hbar}{2}\begin{pmatrix}
      0 & 1\\
      1 & 0
     \end{pmatrix} \begin{pmatrix}
      1\\
      0
     \end{pmatrix}  = \frac{\hbar}{2} \begin{pmatrix}
      0\\
      1
    \end{pmatrix} = \frac{\hbar}{2}\chi_{-\frac{1}{2}} \\
    S_x \chi_{-\frac{1}{2}} &= \frac{\hbar}{2}\begin{pmatrix}
      0 & 1\\
      1 & 0
     \end{pmatrix} \begin{pmatrix}
      0\\
      1
     \end{pmatrix}  = \frac{\hbar}{2} \begin{pmatrix}
      1\\
      0
    \end{pmatrix} = \frac{\hbar}{2}\chi_{\frac{1}{2}} \\
    S_y \chi_{\frac{1}{2}} &= \frac{\hbar}{2}\begin{pmatrix}
      0 & -i\\
      i & 0
     \end{pmatrix} \begin{pmatrix}
      1\\
      0
     \end{pmatrix}  = \frac{i\hbar}{2} \begin{pmatrix}
      0\\
      1
    \end{pmatrix} = \frac{i\hbar}{2}\chi_{-\frac{1}{2}} \\ 
    S_y \chi_{-\frac{1}{2}} &= \frac{\hbar}{2}\begin{pmatrix}
      0 & -i\\
      i & 0
     \end{pmatrix} \begin{pmatrix}
      0\\
      1
     \end{pmatrix}  = -\frac{i\hbar}{2} \begin{pmatrix}
      1\\
      0
    \end{pmatrix} = -\frac{i\hbar}{2}\chi_{\frac{1}{2}} \\
  \end{aligned} \]
\end{frame} 

\begin{frame}[label=current]
  \frametitle{}
  再计算双电子自旋波函数的各种自旋算符的值 \\
  (1) 总自旋算符与单电子自旋算符的关系
  \[ \begin{aligned}
    S^2 &= (S_1 + S_2)^2 \\
    &=  S^2_1 + S^2_2 + 2S_1 \cdot S_2 \\
    &=  S^2_1 + S^2_2 + 2(S_{1x}S_{2x} +S_{1y}S_{2y} + S_{1z}S_{2z} ) \\
    S_z &= S_{1z} + S_{2z}
  \end{aligned}
  \]
  (2) 总自旋算符作用于体系的四个波函数 $\chi ^{(1)}_S, \chi ^{(2)}_S, \chi ^{(3)}_S, \chi ^{(4)}_A$
  \[ \begin{aligned}
    S_z \chi ^{(1)}_S &= (S_{1z} + S_{2z})\chi _{ \frac{1}{2}} (S_{1z})\chi_{ \frac{1}{2}} (S_{2z}) \\
    &= \left[S_{1z}\chi _{ \frac{1}{2}} (S_{1z})\right]\chi_{ \frac{1}{2}} (S_{2z}) +  \chi_{ \frac{1}{2}} (S_{1z})\left[S_{2z}\chi _{ \frac{1}{2}} (S_{2z})\right] \\
    &= \frac{\hbar}{2}\chi _{ \frac{1}{2}} (S_{1z})\chi_{ \frac{1}{2}} (S_{2z}) + \frac{\hbar}{2} \chi_{ \frac{1}{2}} (S_{1z}) \chi_{ \frac{1}{2}} (S_{2z}) \\
    &= \hbar \chi ^{(1)}_S
  \end{aligned}
  \]
\end{frame} 

\begin{frame}[label=current]
  \frametitle{}
同理可得其他三个,有
\[ \begin{aligned}
  S_z \chi ^{(1)}_S &= \hbar \chi ^{(1)}_S \\
  S_z \chi ^{(2)}_S &= -\hbar \chi ^{(2)}_S \\
  S_z \chi ^{(3)}_S &= 0 \\
  S_z \chi ^{(4)}_A &= 0 \\
\end{aligned}
\]
说明 $\chi ^{(1)}_S, \chi ^{(2)}_S, \chi ^{(3)}_S, \chi ^{(4)}_A$ 是 $S_z$的本征态 

~~\\ 
接着计算$S^2$
\[ \begin{aligned}
  S^2 \chi ^{(1)}_S &= \left[S^2_1 + S^2_2 + 2(S_{1x}S_{2x} +S_{1y}S_{2y} + S_{1z}S_{2z} )\right] \chi _{ \frac{1}{2}} (S_{1z})\chi_{ \frac{1}{2}} (S_{2z}) \\ 
  &= \left[\frac{3}{4}\hbar^2 + \frac{3}{4}\hbar^2 + 2\frac{\hbar^2}{4}+ 2 (S_{1x}S_{2x} +S_{1y}S_{2y}  )\right] \chi _{ \frac{1}{2}} (S_{1z})\chi_{ \frac{1}{2}} (S_{2z}) 
\end{aligned}
\]
\end{frame} 

\begin{frame}[label=current]
  \frametitle{}
需要计算 
\[ \begin{aligned}
  (S_{1x}S_{2x})  \chi _{ \frac{1}{2}} (S_{1z})\chi_{ \frac{1}{2}} (S_{2z})  &=\left[S_{1x}\chi _{ \frac{1}{2}} (S_{1z})\right]\left[S_{2x}\chi_{ \frac{1}{2}} (S_{2z})\right] \\
  &= \frac{\hbar}{2} \chi _{ -\frac{1}{2}} (S_{1z}) \frac{\hbar}{2} \chi _{ -\frac{1}{2}} (S_{2z}) \\
  &= \frac{\hbar^2}{4} \chi ^{(2)}_S  \\
  (S_{1y}S_{2y})  \chi _{ \frac{1}{2}} (S_{1z})\chi_{ \frac{1}{2}} (S_{2z})  &=\left[S_{1y}\chi _{ \frac{1}{2}} (S_{1z})\right]\left[S_{2y}\chi_{ \frac{1}{2}} (S_{2z})\right] \\
  &= \frac{i\hbar}{2} \chi _{ -\frac{1}{2}} (S_{1z}) \frac{i\hbar}{2} \chi _{ -\frac{1}{2}} (S_{2z}) \\
  &= -\frac{\hbar^2}{4} \chi ^{(2)}_S  \\
\end{aligned}
\]
两式相加,结果为零,代回,有 $S^2 \chi ^{(1)}_S = 2 \hbar^2  \chi ^{(1)}_S$
\end{frame} 

\begin{frame}[label=current]
  \frametitle{}
  同理可得其他三个,有
  \[ \begin{aligned}
    S^2 \chi ^{(1)}_S &= 2 \hbar^2  \chi ^{(1)}_S \\
    S^2 \chi ^{(2)}_S &= 2 \hbar^2  \chi ^{(2)}_S \\
    S^2 \chi ^{(3)}_S &= 2 \hbar^2  \chi ^{(3)}_S \\
    S^2 \chi ^{(4)}_A &= 0 \\
  \end{aligned}
  \]
  说明 $\chi ^{(1)}_S, \chi ^{(2)}_S, \chi ^{(3)}_S, \chi ^{(4)}_A$ 也是 $S^2$的本征态 
\end{frame} 

\begin{frame}[label=current]
  \frametitle{自旋单态和三态}
  \begin{minipage}[b]{0.49\textwidth}
  总自旋角动量大小 
  \[ \begin{aligned}
    S^2 \chi ^{(1)}_S &= 2 \hbar^2  \chi ^{(1)}_S \\
    S^2 \chi ^{(2)}_S &= 2 \hbar^2  \chi ^{(2)}_S \\
    S^2 \chi ^{(3)}_S &= 2 \hbar^2  \chi ^{(3)}_S \\
    S^2 \chi ^{(4)}_A &= 0 \\
  \end{aligned}
  \]
  \end{minipage}
  \begin{minipage}[b]{0.49\textwidth}
    总自旋角动量投影
    \[ \begin{aligned}
      S_z \chi ^{(1)}_S &= \hbar \chi ^{(1)}_S \\
      S_z \chi ^{(2)}_S &= -\hbar \chi ^{(2)}_S \\
      S_z \chi ^{(3)}_S &= 0 \\
      S_z \chi ^{(4)}_A &= 0 \\
    \end{aligned}
    \]  
\end{minipage}
\begin{table}[htbp]
  \centering\begin{tabular}{ccccccc}
  \toprule
   & $\hat{S}$ & $\hat{S}_z$ & $s$ & $m_s$ & $^{2s+1}\chi_{m_s}$ \\
   \midrule
   $\chi ^{(1)}_S$ & $\sqrt{2} \hbar$ & $\hbar$ & $1$ & $1$ & $^{3}\chi_{1}$ & $\text{三态}$ \\ 
   $\chi ^{(2)}_S$ & $\sqrt{2} \hbar$ & $-\hbar$ & $1$ & $-1$ & $^{3}\chi_{-1}$ & $\text{三态}$ \\
   $\chi ^{(3)}_S$ & $\sqrt{2} \hbar$ & $0$ & $1$ & $0$ & $^{3}\chi_{0}$ & $\text{三态}$ \\ 
   $\chi ^{(4)}_A$ & $0 $ & $0$ & $0$ & $0$ & $^{1}\chi_{0}$ & $\text{单态}$\\
   \bottomrule   
  \end{tabular}
  %\caption{<caption>}
  %\label{<label>}
\end{table}
\end{frame} 

\begin{frame}[label=current]
  \frametitle{}
\begin{figure}[htbp]
  \centering
  \includegraphics[width=0.43\textwidth]{figs/spintrp.png}
  %\caption{}
    %\label{fig:}
\end{figure}
物理图像: 交换对称态是自旋平行三重态,交换反对称态是自旋反平行单态
\end{frame} 

\subsection{氦原子}
\begin{frame}[label=current]
  \frametitle{氦原子}
  以电荷为$+2\left\vert e \right\vert$的原子核为坐标原点
  \begin{figure}[htbp]
    \centering
    \includegraphics[width=0.3\textwidth]{figs/he1.png}
    %\caption{}
      %\label{fig:}
  \end{figure}
  氦原子哈密顿
  \[
  H=\frac{\mathbf{p}_1^2}{2 m}+\frac{\mathbf{p}_2^2}{2 m}-\frac{2 e^2}{r_1}-\frac{2 e^2}{r_2}+\frac{e^2}{r_{12}}
  \]
  式中最后一项为电子间的相互作用能, 它的存在导致不能分离变量。\\
\end{frame} 

\begin{frame}[label=current]
  \frametitle{}
  1、先不考虑电子间的相互作用项, 则电子间无相互作用, 问题变为中心势场下的近独立双电子体系。其波函数记为
  \[ \Psi(\vec{q}_1, \vec{q}_2) = \Psi(\vec{r}_1, S_{1z}, \vec{r}_2, S_{2z}) = \psi(\vec{r}_1, \vec{r}_2) \chi (S_{1z}, S_{2z})\]
其中, 自旋波函数已求得(三个对称,一个反对称)
  \[ 
    \begin{aligned}
      \chi ^{(1)} _S &= \chi _{ \frac{1}{2}} (S_{1z})\chi_{ \frac{1}{2}} (S_{2z})  \\ 
      \chi ^{(2)} _S &= \chi _{ -\frac{1}{2}} (S_{1z})\chi_{ -\frac{1}{2}} (S_{2z})  \\ 
      \chi ^{(3)} _S &= \frac{1}{\sqrt{2} }\left[\chi _{ \frac{1}{2}} (S_{1z})\chi_{ -\frac{1}{2}} (S_{2z})  + \chi _{ -\frac{1}{2}} (S_{1z})\chi_{\frac{1}{2}} (S_{2z})\right] \\ 
      \chi ^{(4)} _A &= \frac{1}{\sqrt{2} }\left[\chi _{ \frac{1}{2}} (S_{1z})\chi_{ -\frac{1}{2}} (S_{2z})  - \chi _{ -\frac{1}{2}} (S_{1z})\chi_{\frac{1}{2}} (S_{2z})\right] 
    \end{aligned}\]
\end{frame} 

\begin{frame}[label=current]
  \frametitle{}
由于两粒子间相互作用不计, 位置函数写成单电子形式, 考虑到交换对称性。 与自旋单态$\chi ^{(4)} _A$对应的对称位置函数,记为
$$
\psi _S \left(\vec{r}_1, \vec{r}_2\right)=\frac{1}{\sqrt{2}}\left[\psi_{n' l' m'}\left(\vec{r}_1\right) \psi_{n l m}\left(\vec{r}_2\right) + \psi_{n' l' m'}\left(\vec{r}_2\right) \psi_{n l m}\left(\vec{r}_1\right)\right]
$$
与自旋三态$\chi ^{(1)} _S, \chi ^{(2)} _S, \chi ^{(3)} _S$对应的反对称位置函数,记为
$$
\psi _A \left(\vec{r}_1, \vec{r}_2\right)=\frac{1}{\sqrt{2}}\left[\psi_{n' l' m'}\left(\vec{r}_1\right) \psi_{n l m}\left(\vec{r}_2\right) - \psi_{n' l' m'}\left(\vec{r}_2\right) \psi_{n l m}\left(\vec{r}_1\right)\right]
$$
\end{frame} 

\begin{frame}[label=current]
  \frametitle{}
考虑基态, 只存在对称的 
$$
\begin{aligned}
  \psi _S \left(\vec{r}_1, \vec{r}_2\right)&=\frac{1}{\sqrt{2}}\left[\psi_{100}\left(\vec{r}_1\right) \psi_{100}\left(\vec{r}_2\right) + \psi_{100}\left(\vec{r}_2\right) \psi_{100}\left(\vec{r}_1\right)\right] \\ 
  &= \psi_{100}\left(\vec{r}_1\right) \psi_{100}\left(\vec{r}_2\right) \\
  &= \frac{Z^3}{\pi a_0^3} e^{-Z\left(r_1+r_2\right) / a_0}
\end{aligned}
$$
因此,有 
\[ \Psi _{(1s)^2}(\vec{q}_1, \vec{q}_2) = \psi_{100}\left(\vec{r}_1\right) \psi_{100}\left(\vec{r}_2\right) \chi ^{(4)} _A \]
能量
\[ E^{(0)}_1 = 2 \times \frac{Z^2}{n^2}E_1 = 8 \times (-13.6 \, eV)  = -108. 8 \, eV\]
\end{frame} 

\begin{frame}[label=current]
  \frametitle{}
2、把电子间的相互作用能看成微扰, 则能量一级修正是如下平均值
\[ E^{(1)}_1 = \left\langle \Psi _{(1s)^2} \middle| \frac{e^2}{r_{12}} \middle| \Psi _{(1s)^2} \right\rangle  = \frac{5}{2} \frac{e^2}{2 a_0}  = 34 \,eV\]
因此,一级修正条件的基态能为
\[ E_1 = E^{(0)}_1 + E^{(1)}_1  =-74.8 \, eV\]
实验值为 $78.8 \, eV$
\end{frame} 

\begin{frame}[label=current]
  \frametitle{}
3、变分法 , 把任意在下能量平均值公式在哈密顿本征函数系展开
$$
\begin{aligned}
\bar{H} & =\langle\psi|\hat{H}| \psi\rangle=\sum_n\left\langle\psi|\hat{H}| \phi_n>\left\langle\phi_n \mid \psi\right\rangle\right. \\
& =\sum_n E_n\left\langle\psi\left|\phi_n>\phi_n\right| \psi\right\rangle  \\
& \geq E_0 \sum_n\left\langle\psi \mid \phi_n\right\rangle\left\langle\phi_n \mid \psi\right\rangle \\
& =E_0\langle\psi \mid \psi\rangle=E_0
\end{aligned}
$$ 
${\color{red}\star}~$ 结论:任意态的能量平均值总大于等于基态能量
\end{frame} 

\begin{frame}[label=current]
  \frametitle{}

考虑库仑屏蔽作用, 把质子数作变分, 位置波函数变为
\[ \left\vert \widetilde{0} \right\rangle = \Psi _{(1s)^2}(\vec{q}_1, \vec{q}_2, \mathfrak{Z} ) = \frac{\mathfrak{Z}^3}{\pi a_0^3} e^{-\mathfrak{Z}\left(r_1+r_2\right) / a_0} \]
求能量平均值
\[ \begin{aligned}
  \overline{H} &= \left\langle \Psi _{(1s)^2}(\vec{q}_1, \vec{q}_2, \mathfrak{Z} ) \middle| H \middle| \Psi _{(1s)^2}(\vec{q}_1, \vec{q}_2, \mathfrak{Z} ) \right\rangle \\
  &= \left\langle\widetilde{0}\left|\frac{\mathbf{p}_1^2}{2 m}+\frac{\mathbf{p}_2^2}{2 m}\right| \widetilde{0}\right\rangle-\left\langle\widetilde{0}\left|\frac{Z e^2}{r_1}+\frac{Z e^2}{r_2}\right| \widetilde{0}\right\rangle+\left\langle\widetilde{0}\left|\frac{e^2}{r_{12}}\right| \widetilde{0}\right\rangle \\
  &= \left( 2 \frac{\mathfrak{Z}^2}{2} - 2 Z \mathfrak{Z}+ \frac{5}{8} \mathfrak{Z} \right) \frac{e^2}{a_0}
\end{aligned} \]
\end{frame} 

\begin{frame}[label=current]
  \frametitle{}
极小值条件 
\[ \frac{\partial \overline{H} }{\partial \mathfrak{Z} } = 2 \mathfrak{Z} -2 + \frac{5}{8} = 0 \]
\[ \implies \mathfrak{Z} = 1.6875 \]
代回, 得能量最小值
\[ E_ 1 = \overline{H}_{\text{min}} = -77.5 \, eV \]
与实验测量值已很接近了!
\end{frame} 

\begin{frame}[label=current]
  \frametitle{}
4、数值方法 \\
多电子原子
\[ \begin{aligned}
  \mathrm{H} & =\sum_{i=1}^z\left(-\frac{1}{2} \nabla_i^2-\frac{Z}{r_i}\right)+\frac{1}{2} \sum_{i \neq j} \sum \frac{1}{r_{i j}} \\
  & =\sum_{i=1}^z h_i+\frac{1}{2} \sum_{i \neq j} \sum \frac{1}{r_{i j}}
  \end{aligned}\]
电子无相互作用基态波函数
$$
\psi\left(r_1, r_2, \cdots r_z\right)=\phi_{k_1}\left(r_1\right) \phi_{k_2}\left(r_2\right) \cdots \phi_{k z}\left(r_z\right)
$$
\end{frame} 

\begin{frame}[label=current]
  \frametitle{}
能量平均值
$$
\overline{H}=\sum_{i=1}^z \int \phi_{k_i}^*\left(\mathrm{r}_i\right) h_i \phi_{k_i}\left(\mathrm{r}_i\right) \mathrm{d} \tau_i+\frac{1}{2} \sum_{i \neq j} \sum \iint\left|\phi_{k_i}\left(\mathrm{r}_i\right)\right|^2 \frac{1}{r_{i j}}\left|\phi_{k_j}\left(\mathrm{r}_j\right)\right|^2 \mathrm{~d} \tau_i \mathrm{~d} \tau_j
$$
 
$$ 
\begin{aligned}
  \text{求变分:} \quad \delta \overline{H} & =\sum_i \int\left[\delta \phi_{k_i}^* h_i \phi_{k_i}+\phi_{k_i}^* h_i \delta \phi_{k_i}\right] \mathrm{d} \tau_i \\
& +\frac{1}{2} \sum_{i \neq j} \sum \iint\left[\delta \phi_{k_i}^* \phi_{k_i}+\phi_{k_i}^* \delta \phi_{k_i}\right] \frac{1}{r_{i j}}\left|\phi_{k_j}\left(\boldsymbol{r}_j\right)\right|^2 \frac{1}{r_{i j}} \mathrm{~d} \tau_i \mathrm{~d} \tau_j \\
& +\frac{1}{2} \sum_{i \neq j} \sum \iint\left|\phi_{k_i}\left(\boldsymbol{r}_i\right)\right|^2 \frac{1}{r_{i j}}\left[\delta \phi_{k_j}^* \phi_{k_j}+\phi_{k_j}^* \delta \phi_{k_j}\right] \mathrm{d} \tau_i \mathrm{~d} \tau_j \qquad  \\
& =\sum_i \int\left[\delta \phi_{k_i}^* h_i \phi_{k_i}+\phi_{k_i}^* h_i \delta \phi_{k_i}\right] \mathrm{d} \tau_i \\
& +\sum_{i \neq j} \sum \iint\left[\delta \phi_{k_i}^* \phi_{k_i}+\phi_{k_i}^* \delta \phi_{k_i}\right] \frac{1}{r_{i j}}\left|\phi_{k_j}\left(\boldsymbol{r}_j\right)\right|^2 \frac{1}{r_{i j}} \mathrm{~d} \tau_i \mathrm{~d} \tau_j
\end{aligned}
$$
\end{frame} 

\begin{frame}[label=current]
  \frametitle{}
能量变分用本征值概率变分表示
$$
\delta \overline{H} =\sum_i \varepsilon_i \delta\left|a_i\right|^2=\sum_i \varepsilon_i \delta\left(\int\left|\varphi_{k_i}\left(\mathrm{r}_i\right)\right|^2 \mathrm{~d} \tau_i\right)
$$
联立, 得单电子方程(Hartree方程)
$$
\left[-\frac{1}{2} \nabla_i^2+\left(-\frac{Z}{r_i}+\sum_{j \neq i} \int\left|\varphi_{k_j}\right|\left(\mathrm{r}_j\right)^2 \frac{1}{r_{i j}} \mathrm{~d} \tau_j\right)\right] \varphi_{k_i}=\varepsilon_i \varphi_{k_i}
$$
Hartree 势 由本征函数决定
$$
V^{(0)}\left(r_i\right)=\left(-\frac{Z}{r_i}+\sum_{j \neq i} \int\left|\varphi_{k_i}\right|^2 \frac{1}{r_{i j}} \mathrm{~d} \tau_j \right)
$$
\end{frame} 

\begin{frame}[label=current]
  \frametitle{}
  迭代自洽求解: 
  \begin{figure}[htbp]
    \centering
    \includegraphics[width=0.5\textwidth]{figs/hartree.png}
    %\caption{}
      %\label{fig:}
  \end{figure}
\end{frame} 

\begin{frame}[label=current]
  \frametitle{课外作业}
  \begin{enumerate}
    \item 对于一个自旋为 1 的系统, 哈密顿为
    \[ H = A S_z^2 + B(S_x^2 - S_y^2)\]
    试求体系的能量本征值和归一化的本征态
    \item  在边长为 L 的三维盒子中有四个无相互作用自旋为$\frac{1}{2}$的全同粒子 \\
      1)如果粒子不可区分,试给出系统的前三个最低能级及简并度 \\
      2)如果粒子可区分,试给出系统的前三个最低能级及简并度 \\
      3)如果粒子自旋为1呢?
  \end{enumerate}
\end{frame} 

\begin{frame}[label=current]
  \frametitle{本章要点}
 \begin{enumerate}
  \item 电子自旋的实验根据
  \item 自旋假设的表述
  \item 自旋算符的对易和反对易关系
  \item 自旋算符及其本征态和本征值
  \item 什么是塞曼效应,什么是光谱精细结构
  \item 全同粒子概念
  \item 全同性原理的表述
  \item 全同粒子波函数的特点
  \item 玻色子与费米子体系波函数的形式与能级
  \item 自旋单态与三态的波函数及自旋计算
  \item 原子波函数对称性
 \end{enumerate}
\end{frame} 
%%%%%%%%%%%%%%%%%%%%%%%%%%%%%%%%%%%%%
\begin{frame} [plain]
    \frametitle{}
    %\Background[2] 
    \begin{center}
    { {\bf \huge 第一章:绪论 }}
    \end{center}  
    \addtocounter{framenumber}{-1}   
\end{frame}
%%%%%%%%%%%%%%%%%%%%%%%%%%%%%%%%%%

\section{课程简介}

\begin{frame}
    \frametitle{课程目标}
        \begin{enumerate}
            \Item 量子力学基本理论
            \Item 量子力学描述实际问题
            \Item 量子力学求解
        \end{enumerate}
\end{frame}
\begin{frame} 
    \frametitle{分数构成}
        \begin{enumerate}
            \Item 平时成绩 30\%
            \Item 期中成绩 20\%
            \Item 期末成绩 50\%
        \end{enumerate}
\end{frame}

\begin{frame} 
    \frametitle{教学效果}
    \centering
    \includegraphics[width=1.0\textwidth,height=5.0cm]{figs/QMexam1.png}
\end{frame}

\begin{frame}
    \frametitle{参考书目}
        \begin{itemize}
            \Item 《量子力学》卷I,II, 曾谨言, 科学出版社, 2008           
            \Item Principles of quantum mechanics, shankar
            \Item Modern quantum mechanics, shankar
            \Item Lectures on quantum mechanics, weinberg
            \Item Principles of quantum mechanics, Dirac
        \end{itemize}
\end{frame}

\begin{frame}
    \begin{atcbox}{三条军规}
        \begin{enumerate}
            \Item 具有波粒二象性的物体处于叠加态
            \Item 不去测量,则保持在叠加态
            \Item 测量,则得到的结果是随机的
        \end{enumerate}
    \end{atcbox}
\end{frame}

\section{能量子假说}

\subsection{伟大成就}

\begin{frame}[t]
    \frametitle{经典物理伟大成就}
    \begin{atcbox}
    {经典物理伟大成就}
        \begin{enumerate}
            \Item 牛顿力学
            \Item 电磁学
            \Item 热力学与统计物理学
        \end{enumerate}
    \end{atcbox}  
    \begin{quotation}
        "There is nothing new to be discovered in physics now. All that remains is 
        more and more precise measurements"   \\
        \rightline{$\cdots$ Lord Kelvin (1900)\hspace{3em}}
    \end{quotation}
\end{frame}

\begin{frame}
    \frametitle{}
    \begin{quotation}
        "But the beauty and clearness ... is obscured by two small puzzling clouds \faCloud "  \\
        \rightline{$\cdots$ Lord Kelvin (1900.4)\hspace{3em}}   
    \end{quotation}
    ~~ \vspace{0.3em}
    \begin{atcbox}{两朵乌云}    
        \begin{enumerate}
        \Item 迈克尔逊-莫雷实验
        \Item 黑体辐射实验
        \end{enumerate}
    \end{atcbox} 
\end{frame}

\begin{frame}
    \frametitle{迈克尔逊-莫雷实验}
    \begin{center}
    \includegraphics[width=0.8\textwidth]{figs/michel.png}
    \end{center}
    干涉条纹零没移动. \dots \\
    以太不存在
\end{frame}

\begin{frame}
    建立相对论
    \begin{center}
        \includegraphics[width=0.4\textwidth]{figs/relativity.jpg}
    \end{center}   
    改变人们对时空的看法
\end{frame}

\begin{frame}
    \frametitle{黑体辐射实验}
    \begin{center}
    \includegraphics[width=0.7\textwidth]{figs/2021-12-01-23-47-27.png}
    \end{center}
    经典物理学里没有能精确描述辐射曲线的函数
\end{frame}
\begin{frame}
    量子力学建立
    \begin{center}
        \includegraphics[width=0.45\textwidth]{figs/mqm.jpg}
    \end{center}   
    改变了人们对物质的看法
\end{frame}

\begin{frame}
    \frametitle{现代科学基石}
    \begin{center}
        \includegraphics[width=0.75\textwidth]{figs/stone.png}
    \end{center}   
\end{frame}

%%%%%%%%%%%%%%%%%%%%%%%%%%%%%%%%%%%%
\subsection{普朗克公式}
%%%%%%%%%%%%%%%%%%%%%%%%%%%%%%%%%%%%

\begin{frame}
    \frametitle{黑体辐射实验}
    \emf[黑体:]  在任意温度下都能完全吸收任意波段电磁波的理想物体模型
    \begin{center}
        \includegraphics[width=0.5\textwidth]{figs/blackbody_radn_curves.png}
    \end{center}
    \emf[实验发现:] (1)物体在任何温度下都有热辐射,(2)峰位频率与温度成正比。
\end{frame}

\begin{frame}
    \frametitle{三个经验公式}
    \begin{center}
        \includegraphics[width=0.7\textwidth]{figs/threelaws.png}
    \end{center}
\end{frame}

\begin{frame} [t]
    \frametitle{}
    \emf[维恩公式:] 
    \begin{equation*}
        \rho(\nu) d \nu=c_{1} \nu^{3} e^{-c_{2} \nu / T} d \nu 
    \end{equation*}
    从电磁波理论导出 (1893), 只能正确描述高频辐射.\\ 
    {\color{deepblue} Nobel Prize in physics(1911)}\\
    ~~\\ 
    \emf[瑞-金公式:] 
    \begin{equation*}
        \rho(\nu, T) d \nu=\frac{8 \pi}{c^{3}} \nu^{2} k T d \nu 
    \end{equation*}
    热力学统计物理学导出 (1900), 只能正确描述低频辐射.\\ 
   {高频区存在紫外灾难} 
    \begin{equation*}
         \int_0 ^\infty \frac{8 \pi}{c^{3}} \nu^{2} k T d \nu \to \infty 
    \end{equation*}
    {\color{deepblue} Nobel Prize in physics(1904)}\\ \vspace{0.3em}
\end{frame}

\begin{frame}
    \frametitle{}
    \emf[普朗克公式:]  1900-10-19, Max Planck 为解决紫外灾难提出一个公式 
    \begin{equation}
        \boxed{\rho(\nu, T) d \nu=\frac{8 \pi}{c^{3}} \frac{h \nu^{3}}{e^{h \nu / K T}-1} d \nu}
    \end{equation}
    ${\color{red}\star}~$ 无理论来源, 经验公式,数据内插法,全波段相符! \\
    ~~\\ 
    {\color{deepblue} Nobel Prize in physics(1918)}\\
\end{frame}

\begin{frame}
    \centering
    \begin{atcbox}{问题}
        普朗克公式的理论来源在哪?
    \end{atcbox}
    \begin{atcbox}{方案}
        1900-12-14, 普朗克基于\emf[能量子假说] 提出自己的解决方案
    \end{atcbox}
\end{frame}

%%%%%%%%%%%%%%%%%%%%%%%%%%%%%%%%%%%%
\subsection{能量子假说}
%%%%%%%%%%%%%%%%%%%%%%%%%%%%%%%%%%%%
\begin{frame}{能量子假说}
    \begin{atcbox}{能量子假说}
    黑体是一份一份地辐射能量的(能量是量子化的
    )
    \begin{equation}
        E=n\varepsilon \quad (n=1,2,3)
    \end{equation}
    能量子(单元)  $\varepsilon$ 是由热振动频率决定
    \begin{equation}
        \varepsilon=h\nu = \hbar \omega
    \end{equation}
    ${\color{red}\star}~$ 普朗克常数 $$h=6.6260693(11)\times 10^{-34} J\cdot s,\quad \hbar=\frac{h}{2\pi}= 1.0545\times 10^{-34} J\cdot s$$
    \end{atcbox}
\end{frame}

\begin{frame} {推导公式}
    玻耳兹曼能量分布定律
    \begin{equation*}
        \frac{N_{i}}{N}=\frac{\exp \left(-\frac{E_{i}}{k T}\right)}{\sum_{i} \exp \left(\frac{-E_{i}}{k T}\right)}
    \end{equation*}
    \begin{figure}[htbp]
        \centering
        \includegraphics[width=0.8\textwidth]{figs/bolzdistr.png}
        %\caption{}
        %\label{fig:}
    \end{figure}
\end{frame}

\begin{frame}
    \frametitle{}
    {\Bullet} 若能量连续,$E - E+ dE$ 区的能量子占比
    \begin{equation*}
        \frac{e^{-E / k T}}{\int\limits_{0}^{\infty} e^{-E / k T} d E}
    \end{equation*}  
    平均能量 
    \begin{equation*}
        <E>=\int\limits_{0}^{\infty} E \frac{e^{-E / k T}}{\int\limits_{0}^{\infty} e^{-E / k T} d E} d E = -kT \frac{Ee^{-E / k T}\vert_0 ^\infty-\int\limits_{0}^{\infty} e^{-E / k T} d E } {\int\limits_{0}^{\infty} e^{-E / k T} d E }= kT
    \end{equation*}
\end{frame}

\begin{frame}
    \frametitle{}
    {\Bullet} 若能量分立   
    \begin{equation*}
        \frac{e^{-E / k T}}{\int\limits_{0}^{\infty} e^{-E / k T} d E} 
        \implies \frac{e^{-E / k T}}{\sum\limits_{0}^{\infty} e^{-E / k T}} 
        \implies \frac{e^{-nh\nu / k T}}{\sum\limits_{0}^{\infty} e^{-nhv / k T}} 
    \end{equation*}    
    平均能量
    \begin{equation*}
        \begin{split}
            <E> = \sum\limits_{0}^{\infty} nh\nu\frac{e^{-nh\nu / k T}}{\sum\limits_{0}^{\infty} e^{-nh\nu / k T}} 
            = -h\nu \frac{d}{dx} \frac{n e^{-nx}}{\sum\limits_{0}^{\infty} e^{-nx}} 
            = \frac{h\nu}{e^{h\nu/kT}-1} 
        \end{split} 
    \end{equation*}
    能量分立的影响
    \begin{equation*}
        \text{(连续)} \quad k T \leftarrow \frac{h \nu}{e^{ h \nu / k T}-1} \quad \text{(分立)} 
    \end{equation*}
\end{frame}

\begin{frame}
    瑞-金公式(能量连续)
    \begin{equation*}
        \rho(\nu, T) d \nu=\frac{8 \pi}{c^{3}} \nu^{2} k T d \nu 
    \end{equation*}
    把式中 $kT$ 代换成 $\dfrac{h \nu}{e^{ h \nu / k T}-1}$, 得
    \begin{equation*}
        \rho(\nu, T) d \nu=\frac{8 \pi}{c^{3}} \frac{h \nu^{3}}{e^{h \nu / K T}-1} d \nu
    \end{equation*}
    这正是普朗克公式
\end{frame}

\begin{frame}
    \begin{atcbox}{意义}
        普朗克能量子假说打破了经典物理学所谓连续观念的束缚, 开启了量子力学的大门。 
    \end{atcbox}
\end{frame}

\begin{frame}
    \frametitle{}
    \centering
    \begin{atcbox}{}
        历史上,普朗克,德拜,艾伦菲斯特,劳厄,洛伦兹,庞加莱,泡利,玻色,爱因斯坦等从多角度推导过黑体辐射公式,每一次推导都给物理学带来了新的知识.  黑体辐射是一只会下金蛋的鹅。
    \end{atcbox}
\end{frame}

\begin{frame}
    \frametitle{}
    \begin{atcbox}{学术讨论}
        ~~\\ 
        普朗克黑体辐射公式的理论来源是什么?
        ~~\\ 
    \end{atcbox}
\end{frame}

%%%%%%%%%%%%%%%%%%%%%%%%%%%%%%%%%%%%%%%%%%%%%%%%%%%%%%%%%%%%%%%%%%%
\begin{frame}
    \frametitle{作业}
   \begin{enumerate}
       \item 写出能量子假说的内容
       \item 什么是能量子
       \item 从普朗克黑体辐射公式导出瑞-金公式、维恩公式、维恩位移定律
   \end{enumerate}
\end{frame}
%%%%%%%%%%%%%%%%%%%%%%%%%%%%%%%%%%%%%%%%%%%%%%%%%%%%%%%%%%%%%%%%%%%

\section{波粒二象性} 

\subsection{光的波粒二象性} 

\begin{frame}
    \frametitle{粒子-波的不可调和性}
	\begin{columns}
		\begin{column}[t]{0.49\linewidth}
			\emf[粒子性:] 
			\begin{itemize}
				\Item  确定的位置、能量、动量等
				\Item  两个粒子不能同时占据同一位置
				\Item  同一粒子也不能同时占据多个位置
				\Item  碰撞现象
			\end{itemize}
		\end{column}
		\begin{column}[t]{0.46\linewidth}
			\emf[波动性:] 
			\vspace{1ex}
			\begin{itemize}
				\Item  确定的波长、振幅、相位等
				\Item  可以同时出现在同一位置
				\Item  可以同时占据多个位置
				\Item  衍射、干涉,无碰撞
			\end{itemize}
		\end{column}
	\end{columns}
    ~~\\ 
    物理实验和生活常识表明:一个物体要么是粒子,要那是波!
\end{frame}

\begin{frame} 
    \frametitle{光是粒子还是波?}
    \begin{figure}
        \centering
        \subfigure[]{\includegraphics[width=7.4cm]{figs/2021-12-02-15-26-40.png}}
        \subfigure[]{\includegraphics[width=4.5cm]{figs/2021-12-06-11-44-50.png}}
        %\caption{} %图片标题
        %\label{fig:1}  
    \end{figure} 
    \setcounter{subfigure}{0}  
\end{frame}

\begin{frame} {光的波动说}
	\begin{columns}
		\begin{column}[t]{0.46\linewidth}
            水波
            \begin{center}
                \includegraphics[width=2.5in,height=2.5in]{figs/2021-12-02-15-46-16.png}
            \end{center}
		\end{column}
		\begin{column}[t]{0.46\linewidth}
            光波
            \begin{center}
                \includegraphics[width=2.5in,height=2.5in]{figs/2021-12-02-15-49-36.png}
            \end{center}
		\end{column}
	\end{columns}
\end{frame}

\begin{frame}
    \begin{center}
        \includegraphics[width=0.70\textwidth]{figs/2021-12-02-16-23-16.png}
    \end{center}
    光只是一定波长范围内的电磁波
\end{frame}

\begin{frame} 
    波动说面临的困难:
    \begin{itemize}
        \Item  黑体辐射
        \Item  光电效应
        \Item  康普顿效应
        \Item  氢原子光谱
    \end{itemize}
\end{frame}

\begin{frame} 
    \frametitle{光电效应实验}   
    \begin{center}
       \includegraphics[width=0.53\textwidth]{figs/2021-12-02-16-01-21.png}
   \end{center}  
   {\Bullet} 具有瞬时性 \\
   {\Bullet} 存在临界频率 $\nu_0$ \\
   {\Bullet} 光电子能量由光的频率决定
\end{frame}  

\begin{frame} 
    1905, 爱因斯坦考虑推导普朗克公式  \\
    \begin{itemize}
        \Item  普朗克公式与实验相符但没有理论来源
        \Item  瑞-金公式有理论来源但存在紫外灾难
        \Item  为解决紫外灾难, 爱因斯坦提出光量子假说
        \Item  利用光量子假说, 爱因斯坦解决了光电效应问题
    \end{itemize}
\end{frame}

\begin{frame}
    \frametitle{}
       爱因斯坦认为:黑体辐射的光强分布应该就是光子数分布
        \begin{figure}
            \centering
            \subfigure[光强分布]{\includegraphics[width=0.495\textwidth]{figs/blackbody_radn_curves.png}}
            \subfigure[粒子数分布]{\includegraphics[width=0.495\textwidth]{figs/2022-01-17-14-21-43.png}}
        \end{figure}
    \setcounter{subfigure}{0}
\end{frame}

\begin{frame} 
    \frametitle{光量子假说}
    \begin{atcbox}{光量子假说}
        {\begin{itemize}
            \Item 光不仅具有波动性,也具有粒子性,称为光量子(光子), 
            \Item 光子具体确定的能量
        \[\varepsilon=h\nu = \hbar \omega\]
            \Item $n$ 个光量子的能量是 $$E = n\varepsilon = nh\nu$$
            \Item 光子具有确定的动量 (1918) 
            \[p=\frac{E}{c}=\frac{h\nu}{c}=\frac{h}{\lambda}, \quad \vec{p} = \frac{h}{\lambda} \vec{n} = \hbar \vec{k}\]
        \end{itemize}}
    \end{atcbox}
\end{frame}

\begin{frame} 
    \frametitle{光电效应公式}
    基于光量子假说,提出光电效应公式
    \[
    \frac{1}{2}m_eV_0^2=h\nu-W
    \]
    \begin{itemize}
        \Item  瞬时性:光量子碰上电子时,能量被瞬时吸收
        \Item  临界频率:$\nu_0=\frac{W}{h} $
        \Item  光电子能量与光的频率决定: $E_k=h\nu-W$
    \end{itemize}
    ~~\\ 
    {\color{deepblue} Nobel Prize in physics(1921)}
\end{frame}

\begin{frame} 
    基于光电效应公式:
$$
\frac{1}{2}m_eV_0^2=h\nu-W
$$
1916年,密立根实验上测定普朗克系数,验证光子说\\
~~\\ 
{\color{deepblue}Nobel Prize in physics (1923)} 
\end{frame}

\begin{frame} 
    \begin{atcbox}{光量子假说的意义}
        \begin{itemize}
            \Item  揭示能量子的本质:在于光本身是量子的,具有粒子性
            \Item  揭示光的本质:光既具波动性又具粒子性。
        \end{itemize}
    \end{atcbox}
    ~~\\ 
    \begin{quotation}
        "在近代物理学结出硕果的那些重大问题中,很难找到一个问题是爱因斯坦没有做出过重要贡献的。
        在他的各种推测中,他有时可能也曾经没有中标的,例如他的光量子假设,就有点迷失了方向\dots"  \\
        \rightline{$\cdots$ 普朗克\hspace{3em}}   
    \end{quotation}
\end{frame}

\begin{frame}   
    \frametitle{康普顿效应 (1922)}
    
    \begin{center}
        \begin{overpic}[scale=0.35]{figs/comptonscattering.png}
            \put(30,92){经验公式:$\lambda_{out}-\lambda_{in}=k(1-\cos \theta)$ }
        \end{overpic} 
    \end{center}  
\end{frame}

\begin{frame} 
    ~~\\ 
   \emf[ 推导经验公式:] 电子的能量
    \begin{equation*}
        E_e^2 =m_ec^2=p_e^2c^2 +m_0 ^2 c^4 
    \end{equation*}
    光子能量
    \begin{equation*}
        E =pc 
    \end{equation*}
    设初始电子不动,有能量守恒定律
    \[E_i + m_0 c^2 = E_o + m_ec^2 \]
    移项并平方
    \begin{equation*}
        \begin{split}
        (E_i -E_o + m_0 c^2)^2 &= E_e ^2\\
        (p_i c-p_o c + m_0 c^2) ^2 &= p_e ^2 c^2 +m_0 ^2 c^4 \\
        (p_i-p_o)^2 +2 m_0 (p_i c-p_o c) &= p_e ^2 \qquad \cdots (1)
    \end{split}
    \end{equation*}
\end{frame}

\begin{frame}  
    动量守恒定律
    \[\vec{p}_i -\vec{p}_o = \vec{p}_e\]
    标量化
    \begin{equation*}
        \begin{split}
            (\vec{p}_i -\vec{p}_o)\cdot (\vec{p}_i -\vec{p}_o)  &= \vec{p}_e\cdot \vec{p}_e   \\
            p_i ^2 + p_o ^2 -2p_i p_o \cos \theta &= p_e ^2  
        \end{split}
    \end{equation*}
    代入(1)式, 得
    \begin{equation*}
        \begin{split}
            p_i ^2 + p_o ^2 -2p_i p_o \cos \theta &= (p_i-p_o)^2 +2 m_0 (p_i c-p_o c) \\
            \frac{1}{p_o} -\frac{1}{p_i} &= \frac{1}{m_0 c} (1-\cos \theta) \\
            \lambda_o -\lambda_i &= \lambda _e (1-\cos \theta) 
        \end{split}
    \end{equation*}
    \textcolor{red}{结束!}
\end{frame}

\begin{frame}   
    \begin{atcbox}{意义}
        康普顿效应表明:\\
        {\Bullet}光具有粒子性 \\
        {\Bullet} 波长为($\lambda$) 的光具有量子化动量 \[\vec{p}=\hbar \vec{k}\]
        {\Bullet} 动量守恒定律在原子尺度有效
    \end{atcbox}   
    \color{deepred}{Nobel Prize in physics(1927)}\\
\end{frame}

\begin{frame}  
     \frametitle{氢原子光谱}
     \begin{center}
        \includegraphics[width=0.6\textwidth]{figs/2022-01-17-14-02-45.png}
    \end{center}  
    经验公式:
       $$\dfrac{1}{\lambda}=R_H c (\dfrac{1}{m^2} -\dfrac{1}{n^2})$$ 
\end{frame}

\begin{frame} 
    \frametitle{}  
    卢瑟福有核原子模型:
    \begin{center}
        \includegraphics[width=0.8\textwidth]{figs/utherford_atom.png}
    \end{center}  
\end{frame}

\begin{frame}   \frametitle{}
    \begin{atcbox}{玻尔的氢原子假说}
        \begin{itemize}
            \item 定态假设: 电子在核外作圆周运动,符合量子化条件的态是稳定的,不辐射能量,称为定态 
            \[ L=n \frac{h}{2\pi}= n \hbar,\qquad (\oint p_i dq_i = n_i h)\] 
            \item 跃迁假设:电子从一个定态跃迁到另一定态,会发射或吸收一个光子
            \[ h\nu=E_n -E_m \]
        \end{itemize}
    \end{atcbox}
\end{frame}

\begin{frame}   
    \frametitle{}
    推导光谱经验公式\\
    {\Bullet} 定态轨道半径:
    \begin{equation*}
        \begin{split}
            m\frac{v^2}{r}&=\frac{1}{4\pi\epsilon_0} \frac{e^2}{r^2} \\
            L_n=&n\hbar=mvr_n  \\
            r_n&= n^2 (\frac{\epsilon_0 h^2}{m\pi e^2}) =n^2 r_1   
        \end{split} 
     \end{equation*}
     {\Bullet} 定态轨道能量: 
     \begin{equation*}
        \begin{split}
            E_n &= T + U \\
            &= \frac{1}{2}mv^2- \frac{1}{4\pi\epsilon_0} \frac{e^2}{r_n ^2} \\
            &= \frac{1}{n^2} (-\frac{m e^4}{8 \epsilon_0 ^2 h^2}) = \frac{E_1}{n^2}
        \end{split}  
     \end{equation*}
\end{frame}

\begin{frame} \frametitle{}
    {\Bullet} 光谱: 
    \begin{equation*}
        \begin{split}
         \nu&=\frac{E_n -E_m}{h} \\
         &= \frac{m e^4}{4\pi \hbar ^3} [\frac{1}{m^2} -\frac{1}{n^2}]
        \end{split}  
     \end{equation*}
     {\Bullet} 里德伯常量 : 
     \[R_{theo}= \frac{m e^4}{4\pi \hbar ^3 c} =1.0973731\times 10^7 m^{-1}\]
    \[R_{exp}=1.0974\times10^7 m^{-1} \]  
\end{frame}

\begin{frame}   
    \frametitle{玻尔的原子模型}  
    \begin{center}
        \includegraphics[width=0.6\textwidth]{figs/bohrmodel.png}
    \end{center}  
\end{frame}

\begin{frame}  
  {\Bullet} In 1905, 爱因斯坦提出光量子具有确定的能量 \[ E = h\nu \] 
  {\Bullet} In 1918, 爱因斯坦提出光量子具有确定的动量   \[ p=\frac{h}{\lambda}\]
  至此,人们把无质量光量子称为光子(photon) 
\end{frame}

\begin{frame} 
    \frametitle{}  
    {\Bullet} 1905年爱因斯坦提出的光量子概念,不受名人的重视,普朗克把爱因斯坦的光量子概念说成是“迷失了方向”。\\
    ~~\\ 
    {\Bullet} 1913年,28岁的玻尔,创造性地把光量子概念用到卢瑟福模型上,成功破解氢原子光谱问题 \\
    ~~\\ 
    {\color{deepblue} {\Bullet} Nobel Prize in physics(1922)}\\ 
\end{frame}

\begin{frame} 
    \frametitle{光的波粒二象性}  
  $$\begin{cases}
    \text{光具有波动性}\\
    \text{~~\qquad 光的干涉} \\
    \text{~~\qquad 光的衍射} \\
    \text{光具有粒子性}\\
    \text{~~\qquad 黑体辐射} \\
    \text{~~\qquad 光电效应} \\
    \text{~~\qquad 康普顿效应} \\
   \end{cases}$$
   ~~\\
   光既具波动性又具粒子性,称为光的波-粒二象性 (wave-particle duality)
\end{frame}

\begin{frame}
    \frametitle{}
    \begin{atcbox}{学术讨论}
        ~\\
        How can the light behave like both particles and waves ?
    \end{atcbox}
\end{frame}

%%%%%%%%%%%%%%%%%%%%%%%%%%%
\subsection{物质波假说}
%%%%%%%%%%%%%%%%%%%%%%%%%%%

\begin{frame}   
  \frametitle{物质波假说}
  \begin{atcbox}{物质波假说}
  1923年, 德布罗意(de Broglie)提出:如果作为经典波的光能够具有量子粒子性,那么作为经典粒子的物体也具有量子波动性,波长和频率为
  \[\lambda=\frac{h}{p}, \qquad \nu =\frac{E}{h}\]
  \end{atcbox}
\end{frame}

\begin{frame}  
    \frame{}
    \例[1]{试计算玻尔氢原子轨道上电子的波长}
    \解 根据定态假设,有
    \begin{equation*}
        \begin{split}
            L&=n\hbar \\
            \vec{r} \cdot \vec{p} & =  n\frac{h}{2 \pi} \\
            2\pi r&=  n\frac{h}{p}\\
            2\pi r&=  n\lambda  \\
        \end{split} 
     \end{equation*}
     这正是驻波条件! \\
     代入 $n=1$,即得第一玻尔轨道上电子的波长
     \[\lambda _e = (\frac{2\epsilon_0 h^2}{m e^2})\]
\end{frame}


\begin{frame}   
  \frametitle{实验验证}
  1、电子衍射实验(戴维森和革末, 1927)
  \begin{center}
    \includegraphics[width=0.46\textwidth]{figs/elediffr.jpeg} \\
    \end{center} 
\end{frame}
\begin{frame}   
    \begin{center}
      \includegraphics[width=0.55\textwidth]{figs/scatting.png} \\
    \end{center} 
       由布拉格公式 $$2d\sin \theta=n\lambda $$
       所得的电子波长,正好比电子的理论波长(德布罗意波长)相符\\
       ~~\\ 
  {\color{deepblue} Nobel Prize in physics(1937)}  
  \end{frame}

\begin{frame}{}
    2、电子双缝干涉实验
    \includemedia[
    width=1.0\linewidth,height=0.58\linewidth, 
    activate=pageopen,
    addresource=figs/doubleslite-n.mp4,
    flashvars={
    source=figs/doubleslite-n.mp4
    autoPlay=true
    loop=true
    }
    ]{}{VPlayer.swf}
\end{frame}

\begin{frame}
      \frametitle{}
      3、$C_{60}$分子的双缝干涉实验
    \begin{center}
         \includegraphics[width=0.8\textwidth,height=2.3in]{figs/c60.png}
    \end{center}
\end{frame}

\begin{frame}
    \frametitle{}   
    \begin{atcbox}{结论}
    {波粒二象性是一切物质的本质特性}
    \end{atcbox} 
        \begin{figure}
            \centering
            \subfigure[]{\includegraphics[width=4.5cm]{figs/ds-1.jpeg}}
            \subfigure[]{\includegraphics[width=4.5cm]{figs/ds-2.jpeg}}
            \subfigure[]{\includegraphics[width=4.5cm]{figs/ds-3.jpeg}}
        \end{figure}
    \setcounter{subfigure}{0}
\end{frame}

\begin{frame}
    \frametitle{学术讨论:}
    \begin{atcbox}{Big problem}
        \large  How to interpret the world where waves are particles and particles are waves?
    \end{atcbox}
\end{frame}

%%%%%%%%%%%%%%%%%%%%%%%%%%%%%%%%%%%%%%%%%%%%%%%%%%%%%%%%%%%%%%%%%%%
\begin{frame}
    \frametitle{课外作业}
    \begin{enumerate}
        \item 计算氢原子第一玻尔半径上电子的德布罗意波长
        \item 计算一个静态电子经100 V电势差加速后的德布罗意波长
        \item 一个He-Ne 激光器发射波长为633 nm的激光。若该激光器功率为1 mW,求每秒钟发射多少光子
    \end{enumerate}
\end{frame}
%%%%%%%%%%%%%%%%%%%%%%%%%%%%%%%%%%%%%%%%%%%%%%%%%%%%%%%%%%%%%%%%%%%

%%%%%%%%%%%%%%%%%%%%%%%%%%%%%%%%%%%%%%%%%%%%%%%%%%%%%%%%%%%%%%%%%%%
\begin{frame}
    \frametitle{本章要点}
    \begin{enumerate}
        \item 黑体、紫外灾难、普朗克公式、普朗克能量子假说
        \item 光电效应、光量子假说、光电效应公式
        \item 定态、量子跃迁、玻尔氢原子模型
        \item 物质波假说、电子的德布罗意波长
        \item 康普顿效应实验、戴维森和革末实验、普朗克常数
    \end{enumerate}
\end{frame}
%%%%%%%%%%%%%%%%%%%%%%%%%%%%%%%%%%%%%%%%%%%%%%%%%%%%%%%%%%%%%%%%%%%
%%%%%%%%%%%%%%%%%%%%%%%%%%%%%%%%%%%%%
\begin{frame} [plain]
    \frametitle{}
    %\Background[2] 
    \begin{center}
    { {\bf \huge 第六章:微扰理论 }}
    \end{center}  
    \addtocounter{framenumber}{-1}   
\end{frame}
%%%%%%%%%%%%%%%%%%%%%%%%%%%%%%%%%%

\section{定态微扰理论}

\subsection{非简并定态微扰}
\begin{frame}
  \frametitle{微扰理论在物理学中的地位
  }
  \begin{itemize}
    \Item 单体问题:一个粒子在确定势场中运动。
    \Item 二体问题:两物体之间的相互作用使它们绕质心运动。
    \Item 微扰问题:增加一个物体,构成三体问题,如果对原体系的扰动很小,考虑微扰法等。
  \end{itemize}
  \begin{figure}
    \includegraphics[width=0.4\textwidth]{figs/2023-04-21-21-34-23.png}   
\end{figure}
\end{frame} 

\begin{frame} 
  \frametitle{}
  \begin{itemize}
    \Item 影响很大,则是标准三体问题,不能精确求解,考虑变分法等。
    \Item 多体问题:考虑平均场近似方法
    \Item 大数目问题:体系中物体数目众多,这时呈现出的规律性如:凝聚、超导、超流,体现出统计规律性,考虑统计物理方法。
  \end{itemize}
\end{frame} 

\begin{frame}
  \frametitle{}
  \emf[定态微扰:] 哈密顿不显含时间,微扰导致状态移动——定态问题 \\
  \begin{itemize}
    \item 对非简并能级的扰动
    \item 对简并能级的扰动
  \end{itemize}

  ~~\\ 
  \emf[跃迁问题:] 哈密顿显含时间,微扰导致状态改变——跃迁问题 
\end{frame} 

\begin{frame}
  \frametitle{定态微扰}
  \emf[理想体系:] 不考虑微扰时能精确求解的体系 \\
  \emf[实际体系:] 考虑微扰时不能精确求解的体系 \\

  ~~\\ 
设实际体系的哈密顿(不显含时间),其本征方程为
\[\hat{\boldsymbol{H}}\left|\psi_n\right\rangle =\boldsymbol{E}_n\left|\psi_n\right\rangle \]
这是不能精确求解的。设它可以做某种展开,一阶情况为
$$
\hat{H}=\hat{H}^{(0)}+\lambda\hat{H}^{(1)} =\hat{H}^{(0)}+\hat{H}^{\prime} 
$$
其中,零阶$ \hat{H}^{(0)} $的本征方程是可以精确求解的
$$
\hat{H}^{(0)}\left|\psi_n^{(0)}\right\rangle=E_n^{(0)}\left|\psi_n^{(0)}\right\rangle
$$
\end{frame} 

\begin{frame}
  \frametitle{非简并定态微扰}
若解得的能量本征值$ E_n^{(0)}$ 都是非简并的,则构成非简并定态微扰问题。\\

~~\\ 
实际体系的本征能量和本征函数也写成关于$\lambda$的级数展开式
$$
\begin{aligned}
& E_n=E_n^{(0)}+\lambda E_n^{(1)}+\lambda^2 E_n^{(2)}+\cdots \\
& \left|\psi_n>=\right| \psi_n^{(0)}>+\lambda\left|\psi_n^{(1)}>+\lambda^2\right| \psi_n^{(2)}>+\cdots
\end{aligned}
$$
代回实际体系的本征方程
\small $$ 
\begin{aligned} 
  \left(\hat{\boldsymbol{H}}^{(0)}+\lambda \hat{\boldsymbol{H}}^{(1)}\right)&\left(\left|\psi_n^{(0)}>+\lambda\right| \psi_n^{(1)}>+\lambda^2 \mid \psi_n^{(2)}>+\cdots\right) \\
  & \left.=\left(\boldsymbol{E}_n^{(0)}+\lambda \boldsymbol{E}_n^{(1)}+\lambda^2 \boldsymbol{E}_n^{(2)}+\cdots\right)\left(\left|\psi_n^{(0)}\right\rangle+\lambda\left|\psi_n^{(1)}>+\lambda^2\right| \psi_n^{(2)}\right\rangle+\cdots\right) 
\end{aligned}
$$
\end{frame} 
\begin{frame}
  \frametitle{}
左右两端分别做乘法计算,并按$ \lambda  $的幂整理,得
\[\begin{bmatrix} ~ & \hat{H}^{(0)}\left|\psi_n^{(0)}\right\rangle + \qquad \qquad \qquad \qquad\\ 
  \lambda & \left[\hat{H}^{(0)}\left|\psi_n^{(1)}\right\rangle +\hat{H}^{(1)}\left|\psi_n^{(0)}\right\rangle\right] +  \\
  \lambda ^2  & \left[\hat{H}^{(0)}\left|\psi_n^{(2)}\right\rangle+\hat{H}^{(1)}\left|\psi_n^{(1)}\right\rangle\right]+\\
  \lambda ^3 & \cdots \\
\end{bmatrix} = 
\begin{bmatrix} ~ & E_n^{(0)}\left|\psi_n^{(0)}\right\rangle+ \qquad \qquad \qquad \qquad  \qquad \qquad \qquad \\ 
  \lambda & \left.\left[E_n^{(0)}\left|\psi_n^{(1)}>+E_n^{(1)}\right| \psi_n^{(0)}\right\rangle\right]+ \qquad  \qquad \qquad \quad\\
  \lambda ^2  & \left[E_n^{(0)}\left|\psi_n^{(2)}\right\rangle+E_n^{(1)}\left|\psi_n^{(1)}\right\rangle+E_n^{(2)}\left|\psi_n^{(0)}\right\rangle\right]+\\
  \lambda ^3 & \cdots \\
\end{bmatrix}
\]
等式两边同幂次的系数应相等
\[
  \begin{aligned}
    &\hat{H}^{(0)}\left|\psi_n^{(0)}>=E_n^{(0)}\right| \psi_n^{(0)}> \cdots\cdots (a)\\
    & \hat{H}^{(0)}\left|\psi_n^{(1)}\right\rangle+\hat{H}^{(1)}\left|\psi_n^{(0)}\right\rangle=E_n^{(0)}\left|\psi_n^{(1)}\right\rangle+E_n^{(1)}\left|\psi_n^{(0)}\right\rangle \cdots\cdots (b)\\ 
    & \hat{H}^{(0)}\left|\psi_n^{(2)}\right\rangle+\hat{H}^{(1)}\left|\psi_n^{(1)}\right\rangle=E_n^{(0)}\left|\psi_n^{(2)}\right\rangle+E_n^{(1)}\left|\psi_n^{(1)}\right\rangle+E_n^{(2)}\left|\psi_n^{(0)}\right\rangle \cdots\cdots (c)\\
  \end{aligned} 
  \]
  分别称方程(a),(b),(c)为零阶、一阶和二阶修正方程。
\end{frame}

\begin{frame} 
  \frametitle{}
以上三式左乘$\left\langle  \psi_n^{(0)} \right\vert$,得能量的各级修正
\[
  \begin{aligned}
    & E_n^{(0)} = \left\langle  \psi_n^{(0)} \right\vert H^{(0)} \left\vert \psi_n^{(0)} \right\rangle \cdots\cdots (d)\\
    & E_n^{(1)} = \left\langle  \psi_n^{(0)} \right\vert H' \left\vert \psi_n^{(0)} \right\rangle \cdots\cdots (e)\\
    & E_n^{(2)} = \left\langle  \psi_n^{(0)} \right\vert H' \left\vert \psi_n^{(1)} \right\rangle \cdots\cdots (f)\\
  \end{aligned} 
  \]
  同理,可写出能量的三级和四级修正
  \[
  \begin{aligned}
    & E_n^{(3)} = \left\langle  \psi_n^{(0)} \right\vert H' \left\vert \psi_n^{(2)} \right\rangle \cdots\cdots (g)\\
    & E_n^{(4)} = \left\langle  \psi_n^{(0)} \right\vert H' \left\vert \psi_n^{(3)} \right\rangle \cdots\cdots (h)\\
  \end{aligned} 
  \]
\end{frame} 

\begin{frame}
  \frametitle{一阶修正}
\例[1]{求解一阶修正方程
\[ \hat{H}^{(0)}\left|\psi_n^{(1)}\right\rangle+\hat{H}^{(1)}\left|\psi_n^{(0)}\right\rangle=E_n^{(0)}\left|\psi_n^{(1)}\right\rangle+E_n^{(1)}\left|\psi_n^{(0)}\right\rangle\]
}
\解 整理,得标准形一级修正方程
\begin{equation}\label{eq:yjxz}
  \left[\hat{H}^{(0)}-E_n^{(0)}\right]\left|\psi_n^{(1)}\right\rangle=-\left[\hat{H}^{\prime}-E_n^{(1)}\right]\left| \psi_n^{(0)}\right\rangle
\end{equation}
左乘 $\left\langle\psi_n^{(0)}\right|$
$$
\left\langle\psi_n^{(0)}\left|\left[\hat{H}^{(0)}-E_n^{(0)}\right]\right| \psi_n^{(1)}\right\rangle=-\left\langle\psi_n^{(0)}\left|\left[\hat{H}^{\prime}-E_n^{(1)}\right]\right| \psi_n^{(0)}\right\rangle
$$
\end{frame} 
\begin{frame}
  \frametitle{}
左端
$$
\begin{aligned}
  \left\langle\psi_n^{(0)}\left|\left[\hat{H}^{(0)}-E_n^{(0)}\right]\right| \psi_n^{(1)}\right\rangle & =\left\langle\psi_n^{(0)}\left|\hat{H}^{(0)}\right| \psi_n^{(1)}\right\rangle-\left\langle\psi_n^{(0)}\left|E_n^{(0)}\right| \psi_n^{(1)}\right\rangle \\
  &= E_n^{(0)} \left\langle\psi_n^{(0)}\left|\psi_n^{(1)}\right\rangle\right. - E_n^{(0)} \left\langle\psi_n^{(0)}\left|\psi_n^{(1)}\right\rangle\right. \\
& =0
\end{aligned}
$$
右端
$$
\begin{aligned}
  \left\langle\psi_n^{(0)}\left|\left[E_n^{(1)}-\hat{H}^{\prime}\right]\right| \psi_n^{(0)}\right\rangle &=0 \\
  \left\langle\psi_n^{(0)}\left|E_n^{(1)}\right| \psi_n^{(0)} \right\rangle  -  \left\langle\psi_n^{(0)}\left|\hat{H}^{\prime}\right| \psi_n^{(0)} \right\rangle  &=0 \\ 
  E_n^{(1)} - \left\langle\psi_n^{(0)}\left|\hat{H}^{\prime}\right| \psi_n^{(0)} \right\rangle  &= 0 \\
  E_n^{(1)} = \left\langle n \left|\hat{H}^{\prime}\right| n \right\rangle &= H'_{nn}
\end{aligned}
$$
获得能量一级修正 $ E_n^{(1)}  $ ,下面求波函数的一级修正 $ \psi_n^{(1)}  $ 
\end{frame}

\begin{frame}
  \frametitle{}
本征矢$\{ \left| \psi_n^{(0)} \right\rangle\}$构成完备集, 任意函数都可以在其上展开,现把 $ \left|\psi_n^{(1)}\right\rangle + a \left|\psi_n^{(0)}\right\rangle  $展开 
\[ \left|\psi_n^{(1)}\right\rangle + a \left|\psi_n^{(0)}\right\rangle  =\sum_l a_l^{(1)} \left|\psi_l^{(0)}\right\rangle \]
合理选取$a$,以消除左端的$\left|\psi_n^{(0)}\right\rangle$因子,则有
\begin{equation}\label{eq:yjzk}
  \left|\psi_n^{(1)}\right\rangle =\sum_{l\ne n} a_l^{(1)} \left|\psi_l^{(0)}\right\rangle 
\end{equation}
左乘 $\left\langle \psi_n^{(0)} \right|$
\[ \left\langle \psi_n^{(0)} \left| \psi_n^{(1)}\right\rangle \right.=\sum_{l\ne n} a_l^{(1)} \left\langle \psi_n^{(0)} \left|\psi_l^{(0)}\right\rangle \right.\]
\end{frame} 

\begin{frame}
  \frametitle{}
由正交性,式子右端为零, 得
\[ \left\langle \psi_n^{(0)} \left| \psi_n^{(1)}\right\rangle \right.=0\]
把展开式[\ref{eq:yjzk}]代回一级修正方程[\ref{eq:yjxz}]
\begin{equation*}
 \begin{aligned}
   \left[\hat{H}^{(0)}-E_n^{(0)}\right]\left|\psi_n^{(1)}\right\rangle  &= -\left[\hat{H}^{\prime}-E_n^{(1)}\right]\left| \psi_n^{(0)}\right\rangle \\ 
   \left[\hat{H}^{(0)}-E_n^{(0)}\right]\sum_{l\ne n} a_l^{(1)} \left|\psi_l^{(0)}\right\rangle   &= -\left[\hat{H}^{\prime}-E_n^{(1)}\right]\left| \psi_n^{(0)}\right\rangle \\
   \sum_{l\ne n} a_l^{(1)}\left[\hat{H}^{(0)}-E_n^{(0)}\right] \left|\psi_l^{(0)}\right\rangle   &= -\left[\hat{H}^{\prime}-E_n^{(1)}\right]\left| \psi_n^{(0)}\right\rangle   
 \end{aligned}
\end{equation*}
\end{frame} 

\begin{frame}
  \frametitle{}
左乘$\left\langle \psi_m^{(0)} \right|, \quad (m \ne n)$, 得
\begin{equation*}
  \begin{aligned}
    \sum_{l\ne n} a_l^{(1)}\left\langle \psi_m^{(0)} \right|\left[\hat{H}^{(0)}-E_n^{(0)}\right] \left|\psi_l^{(0)}\right\rangle   &= -\left\langle \psi_m^{(0)} \right|\left[\hat{H}^{\prime}-E_n^{(1)}\right]\left| \psi_n^{(0)}\right\rangle  \\
    \sum_{l\ne n} a_l^{(1)}\left[E_m^{(0)}-E_n^{(0)}\right]\left\langle \psi_m^{(0)} \left|\psi_l^{(0)}\right\rangle \right.   &= -\left\langle \psi_m^{(0)} \right|\hat{H}^{\prime}\left| \psi_n^{(0)}\right\rangle + E_n^{(1)}\left\langle \psi_m^{(0)} \left| \psi_n^{(0)}\right\rangle\right. \\ 
    \sum_{l\ne n} a_l^{(1)}\left[E_m^{(0)}-E_n^{(0)}\right] \delta _{ml}   &= -\left\langle \psi_m^{(0)} \right|\hat{H}^{\prime}\left| \psi_n^{(0)}\right\rangle \\ 
    a_m^{(1)}\left[E_m^{(0)}-E_n^{(0)}\right] &= -\left\langle \psi_m^{(0)} \right|\hat{H}^{\prime}\left| \psi_n^{(0)}\right\rangle \\ 
  \end{aligned}
 \end{equation*}
 解得
 \[ a_m^{(1)} = \frac{\left\langle \psi_m^{(0)} \right|\hat{H}^{\prime}\left| \psi_n^{(0)}\right\rangle}{E_n^{(0)}-E_m^{(0)}} = \frac{{H}^{\prime}_{mn}}{E_n^{(0)}-E_m^{(0)}}, \quad (m \ne n) \]
\end{frame} 

\begin{frame}
  \frametitle{}
得波函数的一级修正
\[ \left|\psi_n^{(1)}\right\rangle =\sum_{m\ne n} a_m^{(1)} \left|\psi_m^{(0)}\right\rangle  
= \sum_{m\ne n} \frac{H^{\prime}_{mn}}{E_n^{(0)}-E_m^{(0)}} \left|\psi_m^{(0)}\right\rangle \]
实际体系(哈密顿$\hat{H} = \hat{H}^{(0)} + \hat{H}^{'} $)的一级修正解
\begin{equation*}
  \boxed{\left\{\begin{aligned}
    E_n &= E_n^{(0)} + E_n^{(1)} = E_n^{(0)} + H'_{nn} \\
    \left|\psi_n \right\rangle &= \left|\psi_n^{(0)}\right\rangle +\left|\psi_n^{(1)}\right\rangle = \left|\psi_n^{(0)}\right\rangle + \sum_{m\ne n} \frac{H^{\prime}_{mn}}{E_n^{(0)}-E_m^{(0)}} \left|\psi_m^{(0)}\right\rangle 
  \end{aligned} \right.}
\end{equation*}    
\end{frame} 

\begin{frame}
  \frametitle{二级修正}
\例[2]{求解能量的二级修正
}
\解 整理,得标准形二级修正方程
\begin{equation}\label{eq:ejxz}
  \left[\hat{H}^{(0)} - E_n^{(0)}\right]\left|\psi_n^{(2)}\right\rangle
  =-  \left[\hat{H}^{\prime} - E_n^{(1)}\right] \left|\psi_n^{(1)}\right\rangle+E_n^{(2)}\left|\psi_n^{(0)}\right\rangle
\end{equation}
代入$\left|\psi_n^{(1)}\right\rangle = \sum_{m\ne n} a_m^{(1)} \left|\psi_m^{(0)} \right\rangle $
\begin{equation*}
  \begin{aligned}
    \left[\hat{H}^{(0)} - E_n^{(0)}\right]\left|\psi_n^{(2)}\right\rangle
    &= \left[E_n^{(1)}-\hat{H}^{\prime}\right] \sum_{m\ne n} a_m^{(1)} \left|\psi_m^{(0)} \right\rangle+E_n^{(2)}\left|\psi_n^{(0)}\right\rangle \\ 
  \end{aligned}  
\end{equation*}
\end{frame} 

\begin{frame}
  \frametitle{}
  左乘$\left\langle \psi_n^{(0)}\right|$
  \begin{equation*}
    \begin{aligned}
      &\left\langle \psi_n^{(0)}\right|\left[\hat{H}^{(0)} - E_n^{(0)}\right]\left|\psi_n^{(2)}\right\rangle
      = \sum_{m\ne n} a_m^{(1)}\left\langle \psi_n^{(0)}\right|\left[E_n^{(1)}-\hat{H}^{\prime}  \right]  \left|\psi_m^{(0)} \right\rangle
      +E_n^{(2)}\left\langle \psi_n^{(0)}\left|\psi_n^{(0)}\right\rangle \right. \\
      &\sum_{m\ne n} a_m^{(1)}\left\langle \psi_n^{(0)}\right|\left[E_n^{(1)}-\hat{H}^{\prime}  \right]  \left|\psi_m^{(0)} \right\rangle
      +E_n^{(2)}\left\langle \psi_n^{(0)}\left|\psi_n^{(0)}\right\rangle \right. =0  \\
      & \sum_{m\ne n} a_m^{(1)}\left\langle \psi_n^{(0)}\right|E_n^{(1)}  \left|\psi_m^{(0)} \right\rangle - \sum_{m\ne n} a_m^{(1)}\left\langle \psi_n^{(0)}\right|\hat{H}^{\prime} \left|\psi_m^{(0)} \right\rangle +E_n^{(2)} =0 \\ 
      &\implies  E_n^{(2)} = \sum_{m\ne n} a_m^{(1)}\left\langle \psi_n^{(0)}\right|\hat{H}^{\prime} \left|\psi_m^{(0)} \right\rangle \\
    \end{aligned}  
  \end{equation*}
\end{frame} 

\begin{frame} 
  \frametitle{}
  得能量二级修正
  \begin{equation*}
    \begin{aligned}
      E_n^{(2)} &= \sum_{m\ne n} a_m^{(1)}\left\langle \psi_n^{(0)}\right|\hat{H}^{\prime} \left|\psi_m^{(0)} \right\rangle \\
      & = \sum_{m\ne n} \frac{H ^{\prime} _{mn}}{E_n^{(0)}-E_m^{(0)}} H^{\prime}_{nm} \\
      &=  \sum_{m\ne n} \frac{H ^{\prime} _{mn}}{E_n^{(0)}-E_m^{(0)}} \left(H^{\prime}_{mn}\right)^* \\
      &= \sum_{m\ne n}  \frac{|H ^{\prime} _{mn}|^2}{E_n^{(0)}-E_m^{(0)}} \quad \text{or} \quad  \sum_{m\ne n}  \frac{|H ^{\prime} _{nm}|^2}{E_n^{(0)}-E_m^{(0)}}
    \end{aligned}  
  \end{equation*}
\end{frame} 

\begin{frame}
  \frametitle{}
二级修正下体系的能量
\[\begin{aligned}
  E_n &= E_n^{(0)} + E_n^{(1)} + E_n^{(2)} \\
   &= E_n^{(0)} + H'_{nn} + \sum_{m\ne n}  \frac{|H ^{\prime} _{mn}|^2}{E_n^{(0)}-E_m^{(0)}} \\ 
\end{aligned} \]
\end{frame} 

\begin{frame}
  \frametitle{小结}
(1)实际体系,哈密顿本征方程不能精确求解
\[ H \left|\psi_n \right\rangle = E_n \left|\psi_n \right\rangle\]
若可以分解成理想体系+微扰
\[ H = H^{(0)} + H', \qquad (H^{(0)} \gg H')\]
且理想体系哈密顿本征方程可精确求解
\[ H^{(0)}\left|\psi ^{(0)} _n \right\rangle = E^{(0)}_n \left|\psi ^{(0)} _n \right\rangle\]
(2)如果理想体系的$ E_n^{(0)} $ 能级非简并,可以近似写出实际体系的本征解
\begin{equation*}
  \boxed{\left\{\begin{aligned}
    E_n &= E_n^{0} + E_n^{(1)} + E_n^{(2)} + \cdots  \\
    \left|\psi_n \right\rangle &= \left|\psi_n^{(0)}\right\rangle +\left|\psi_n^{(1)}\right\rangle + \cdots 
  \end{aligned} \right.}
\end{equation*}    
\end{frame} 

\begin{frame}
  \frametitle{}
  其中能级的一级修正是微扰对本能级的直接影响(平均值)
  $$E_n^{(1)} = \left\langle \psi ^{(0)} _n \right|H' \left|\psi ^{(0)} _n \right\rangle $$
  能级的二级修正是微扰导致临近能级发生耦合(矩阵元)
  $$E_n^{(2)} = \sum_{m\ne n}  \frac{|H ^{\prime} _{mn}|^2}{E_n^{(0)}-E_m^{(0)}} $$
  本征函数的一级修正是微扰导致临近能级发生耦合(矩阵元) 
$$
\left|\psi_n^{(1)}\right\rangle = \sum_{m\ne n} \frac{H^{\prime}_{mn}}{E_n^{(0)}-E_m^{(0)}} \left|\psi_m^{(0)}\right\rangle  
$$ 
(3)如果理想体系的能级简并,请参考简并微扰!
\end{frame} 

\begin{frame}
  \frametitle{}
(4)适用条件:修正公式符合级数收敛条件,
\[\frac{H^{\prime}_{mn}}{E_n^{(0)}-E_m^{(0)}} \ll 1\]
即:能级间隔要大,微扰矩阵元要小。
\end{frame} 

\begin{frame}
  \frametitle{}
  ~~\\ 
  \例[3]{电荷量为$q$的电谐振子,置于微弱的电场$\varepsilon$中,其势函数可以表示为
  \[ V(x) = \frac{1}{2}\omega ^2 x^2 -q \varepsilon x \]
  试采用微扰法求能量本征值和本征函数
  }
  \解 体系的哈密顿为
  \[ H = \left[- \frac{\hbar^2}{2 } \frac{\mathrm{d}^2}{\mathrm{d}x^2} + \frac{1}{2}\omega ^2 x^2\right] -q \varepsilon x = H^{(0)} + H'\]
  其中
  \[H^{(0)} = - \frac{\hbar^2}{2 } \frac{\mathrm{d}^2}{\mathrm{d}x^2} + \frac{1}{2}\omega ^2 x^2 \]
  是可以精确求解的理想谐振子,有
  \[E^{(0)}_n = \left(n+\frac{1}{2}\right)\hbar \omega, \qquad \left|\psi^{(0)}_n (x)\right\rangle = N_n \exp(-\frac{\alpha ^2 x^2}{2}) H_n(\alpha x)  \]
\end{frame} 

\begin{frame}
  \frametitle{}
由于能级非简并,且能级间隔较大($\hbar \omega $),可采用微扰法求解\\
能量一级修正
$$
\begin{aligned}
  E_n^{(1)} &= \left\langle \psi ^{(0)} _n \right|H' \left|\psi ^{(0)} _n \right\rangle \\ 
  &= -q \varepsilon N^2_n \int_{-\infty}^{\infty} H^* _n (\alpha x) x H_n (\alpha x) \exp(-\alpha ^2 x^2) dx \\ 
  &= -q \varepsilon N^2_n \int_{-\infty}^{\infty} x H^2 _n(\alpha x) \exp(-\alpha ^2 x^2) dx \\
 &= 0
\end{aligned} $$
\end{frame} 

\begin{frame}
  \frametitle{}
  能量二级修正得先计算微扰矩阵元
  $$
\begin{aligned}
  H'_{mn} &= \left\langle \psi ^{(0)} _m \right|H' \left|\psi ^{(0)} _n \right\rangle \\ 
  &= -q \varepsilon \left\langle \psi ^{(0)} _m \right| \frac{\alpha x}{\alpha} \left|\psi ^{(0)} _n \right\rangle \\
  &=  - \frac{q\varepsilon}{\alpha} \left\langle \psi ^{(0)} _m \right|  \left[\sqrt{\frac{n}{2}} \left|\psi ^{(0)} _{n-1} \right\rangle + \sqrt{\frac{n+1}{2 }} \left|\psi ^{(0)} _{n+1} \right\rangle\right] \\
  &= - \frac{q\varepsilon}{\alpha}\left(\sqrt{\frac{n}{2}} \delta _{m, n-1} +  \sqrt{\frac{n+1}{2}} \delta _{m, n+1}\right)
\end{aligned} $$
\end{frame} 

\begin{frame}
  \frametitle{}
  能量二级修正
  $$
  \begin{aligned}
    E_n^{(2)} &= \sum_{m\ne n}  \frac{|H ^{\prime} _{mn}|^2}{E_n^{(0)}-E_m^{(0)}} \\ 
    &=\left(\frac{q\varepsilon}{\alpha}\right)^2 \sum_{m\ne n} \frac{1}{E_n^{(0)}-E_m^{(0)}} \left(\frac{n}{2}\delta _{m, n-1} +\frac{n+1}{2}\delta _{m, n+1}  \right) \\
    &=  \left(\frac{q\varepsilon}{\alpha}\right)^2 \left(\frac{n}{2}\frac{1}{E_n^{(0)}-E_{n-1}^{(0)}} +\frac{n+1}{2}\frac{1}{E_n^{(0)}-E_{n+1}^{(0)}}  \right)  \\
    &= \left(\frac{q\varepsilon}{\alpha}\right)^2 \left(\frac{n}{2\hbar \omega } -\frac{n+1}{2 \hbar \omega}  \right) \qquad \Leftarrow \alpha ^2 = \frac{ \omega }{\hbar} \\
    &= - \frac{q^2\varepsilon ^2}{2 \omega ^2}
  \end{aligned} $$
\end{frame} 

\begin{frame}
  \frametitle{}
本征函数的一级修正
$$
\begin{aligned}
  \left|\psi_n^{(1)}\right\rangle &= \sum_{m\ne n} \frac{H^{\prime}_{mn}}{E_n^{(0)}-E_m^{(0)}} \left|\psi_m^{(0)}\right\rangle \\
    &=  - \left(\frac{q\varepsilon}{\alpha}\right)\left(\sqrt{\frac{n}{2}} \frac{1}{E_n^{(0)}-E_{n-1}^{(0)}}\psi ^{(0)} _{n-1} +\sqrt{\frac{n+1}{2}} \frac{1}{E_n^{(0)}-E_{n+1}^{(0)}}\psi ^{(0)} _{n+1}  \right)  \\
    &=  - \left(\frac{q\varepsilon}{\alpha}\right)\left(\sqrt{\frac{n}{2}} \frac{1}{\hbar \omega}\psi ^{(0)} _{n-1} -\sqrt{\frac{n+1}{2}} \frac{1}{\hbar \omega}\psi ^{(0)} _{n+1}  \right)  \\ 
    &= q \varepsilon  \sqrt{\frac{1}{2\hbar \omega ^3 }}\left( \sqrt{n+1}\psi ^{(0)} _{n+1} -  \sqrt{n}\psi ^{(0)} _{n-1}  \right)
\end{aligned}  
$$  
\end{frame} 

\begin{frame}
  \frametitle{}
二级修正下体系的能量
\[ E_n = \left(n+\frac{1}{2}\right) \hbar \omega - \frac{q^2\varepsilon ^2}{2 \omega ^2}  \]
二级修正下体系的本征函数
\[ \psi _n =  \psi ^{(0)} _n + q \varepsilon  \sqrt{\frac{n+1}{2\hbar \omega ^3 }} \psi ^{(0)} _{n+1} - q \varepsilon  \sqrt{\frac{n}{2\hbar \omega ^3 }} \psi ^{(0)} _{n-1} \]
\end{frame} 

\begin{frame}
  \frametitle{}
  ~~\\ 
  \例[4]{电荷量为$q$的电谐振子,置于微弱的电场$\varepsilon$中,其势函数可以表示为
  \[ V(x) = \frac{1}{2}\omega ^2 x^2 -q \varepsilon x \]
  试采用变量代换法求能量本征值和本征函数
  }
  \解 体系的哈密顿为
  \[ 
    \begin{aligned}
      H &= - \frac{\hbar^2}{2 } \frac{\mathrm{d}^2}{\mathrm{d}x^2} + \frac{1}{2}\omega ^2 x^2 -q \varepsilon x \\
      &=  - \frac{\hbar^2}{2 } \frac{\mathrm{d}^2}{\mathrm{d}x^2} + \frac{1}{2}\omega ^2 \left(x-\frac{q\varepsilon}{ \omega ^2} \right)^2 - \frac{q^2\varepsilon ^2}{2 \omega ^2}
    \end{aligned} \]
\end{frame} 

\begin{frame}
  \frametitle{}
  薛定谔方程为
  \[ \left[ - \frac{\hbar^2}{2 } \frac{\mathrm{d}^2}{\mathrm{d}x^2} + \frac{1}{2}\omega ^2 \left(x-\frac{q\varepsilon}{ \omega ^2} \right)^2 \right] \psi(x) =\left(E+\frac{q^2\varepsilon ^2}{2 \omega ^2} \right) \psi(x) \]
  令 
  \[ x' = x-\frac{q\varepsilon}{ \omega ^2}, \qquad E'= E+\frac{q^2\varepsilon ^2}{2 \omega ^2}\]
  有
  \[ \left[ - \frac{\hbar^2}{2 } \frac{\mathrm{d}^2}{\mathrm{d}x'^2} + \frac{1}{2}\omega ^2 x'^2 \right] \psi(x') =E' \psi(x') \]
\end{frame} 

\begin{frame}
  \frametitle{}
能级
\[E'_n = \left(n+\frac{1}{2}\hbar \omega \right) = E_n+\frac{q^2\varepsilon ^2}{2 \omega ^2}\]
所以 
\[E_n = \left(n+\frac{1}{2}\hbar \omega \right)-\frac{q^2\varepsilon ^2}{2 \omega ^2}\]
本征函数
\[\psi _n(x') = N_n \exp(-\frac{\alpha ^2 x'^2}{2}) H_n(\alpha x') \]
所以
\[\psi _n(x) = N_n \exp(-\frac{\alpha ^2 \left(x-\frac{q\varepsilon}{ \omega ^2}\right) ^2}{2}) H_n(\alpha \left(x-\frac{q\varepsilon}{ \omega ^2}\right)) \]
式中:$\alpha ^2 = \frac{ \omega }{\hbar}$
\end{frame} 

\begin{frame}
  \frametitle{}
  \例[5]{一粒子处于如下一维无限深势阱$V(x)$中运动,现在势阱中增加微扰$H'$,试求解能级与波函数的一级修正
  $$ \displaystyle 
 V(x)=\left \{ 
 \begin{array}{cccc}
   0	~~ ~~ 0<x<a \\  
   +\infty ~~x<0, x>a\\
 \end{array}
 \right. \qquad H'(x)=\left \{ 
  \begin{array}{cccc}
    -b~~ ~~ 0<x<\frac{a}{2} \\  
    +b ~~\frac{a}{2}<x<a\\
  \end{array}
  \right.
 $$ }
 \解 理想无限深势阱的能级及波函数为
 \[ E^{(0)} _n = \frac{n^2\pi ^2 \hbar^2}{2\mu a^2}, \qquad \psi ^{(0)} _n (x) = \sqrt{\frac{2}{a}}\sin \frac{n\pi x}{a}  \]
\end{frame} 

\begin{frame}
  \frametitle{}
能级一级修正 
\[
  \begin{aligned}
    E^{(1)} _n &= H'_{nn} \\ 
    &= \int_{-\infty}^{\infty} \psi ^{*(0)} _n (x)H'\psi ^{(0)} _n (x)dx \\
    &=  -\int_{0}^{\frac{a}{2}} \frac{2b}{a}\sin ^2\frac{n\pi x}{a} dx + \int_{\frac{a}{2}}^{a} \frac{2b}{a}\sin ^2\frac{n\pi x}{a} dx \\
    &= 0
  \end{aligned} 
  \]
\end{frame} 

\begin{frame}
  \frametitle{}
微扰矩阵元
\[
\begin{aligned}
  H'_{mn} &= \int_{-\infty}^{\infty} \psi ^{*(0)} _m (x)H'\psi ^{(0)} _n (x)dx  \\
  &=  -\int_{0}^{\frac{a}{2}} \frac{2b}{a}\sin \frac{m\pi x}{a}\sin \frac{n\pi x}{a} dx + \int_{\frac{a}{2}}^{a} \frac{2b}{a}\sin \frac{n\pi x}{a} \sin \frac{m\pi x}{a} dx \\
  &= \frac{2b}{a}\left[ \frac{1}{n+m}\sin \frac{n+m}{2}\pi - \frac{1}{n-m}\sin \frac{n-m}{2}\pi \right]
\end{aligned} 
\]
波函数一级修正
\[ 
\begin{aligned}
  \psi ^{(1)}_n (x) &= \sum_{m\ne n} \frac{H'_{mn}}{E^{(0)}_n - E^{(0)}_m } \psi ^{(0)}_m (x) \\
  &= \frac{2\mu a^2}{\pi ^2\hbar^2} \sqrt{\frac{2}{a}} \sum_{m\ne n} \frac{H'_{mn} }{n^2 -m^2} \sin \frac{k\pi}{a}x   
\end{aligned}  
\]
\end{frame} 

\begin{frame}
  \frametitle{}
\例[6]{设某粒子的哈密顿为
$$ H = \begin{bmatrix}
  1 & c & 0  \\
  c & 3 & 0  \\
  0 & 0 & c-2 \\ 
\end{bmatrix}, \qquad c\ll 1 
$$
试求:(1)粒子的本征能量到二级修正及波函数的一级修正, ~ (2)本征能的精确解}
\解 改写哈密顿
$$ H = H^{(0)} + H' = \begin{bmatrix}
  1 & 0 & 0  \\
  0 & 3 & 0  \\
  0 & 0 & -2 \\ 
\end{bmatrix} + \begin{bmatrix}
  0 & c & 0  \\
  c & 0 & 0  \\
  0 & 0 & c  
\end{bmatrix} 
$$
\end{frame} 

\begin{frame}
  \frametitle{}
1)$ H^{(0)}  $ 的本征值(非简并)为
\[ E^{(0)}_1 = 1, \quad E^{(0)}_2 = 3, \quad E^{(0)}_3 = -2\]  
对应的本征函数为
\[ \left\vert \psi^{(0)}_1\right\rangle  = \begin{bmatrix}
  1  \\
  0  \\
  0  
\end{bmatrix} , \quad \left\vert \psi^{(0)}_2 \right\rangle  = \begin{bmatrix}
  0  \\
  1  \\
  0  
\end{bmatrix} , \quad
\left\vert \psi^{(0)}_3 \right\rangle  = \begin{bmatrix}
  0  \\
  0  \\
  1  
\end{bmatrix} \] 
2)能级的一级修正
\[ H'_{nn} = \left\langle\psi^{(0)}_n \right\vert H' \left\vert \psi^{(0)}_n \right\rangle\]
\end{frame} 

\begin{frame}
  \frametitle{}
\[ 
\begin{aligned}
  E^{(1)}_1 = H'_{11} &= \left\langle \psi^{(0)}_1 \right\vert H' \left\vert \psi^{(0)}_1 \right\rangle \\ 
  &= \begin{pmatrix}
    1 & 0 & 0  \\ 
  \end{pmatrix} 
  \begin{pmatrix}
    0 & c & 0  \\
    c & 0 & 0  \\
    0 & 0 & c  
  \end{pmatrix} 
  \begin{pmatrix}
    1  \\
    0  \\
    0  
  \end{pmatrix} \\
  &=0 
\end{aligned}  
\]

\[ 
\begin{aligned}
  E^{(1)}_2 = H'_{22} &= \left\langle \psi^{(0)}_2 \right\vert H' \left\vert \psi^{(0)}_2 \right\rangle \\ 
  &= \begin{pmatrix}
    0 & 1 & 0  \\ 
  \end{pmatrix} 
  \begin{pmatrix}
    0 & c & 0  \\
    c & 0 & 0  \\
    0 & 0 & c  
  \end{pmatrix} 
  \begin{pmatrix}
    0  \\
    1  \\
    0  
  \end{pmatrix} \\
  &=0 
\end{aligned}  
\]
\end{frame} 

\begin{frame}
  \frametitle{}
  \[ 
    \begin{aligned}
      E^{(1)}_3 = H'_{33} &= \left\langle \psi^{(0)}_3 \right\vert H' \left\vert \psi^{(0)}_3 \right\rangle \\ 
      &= \begin{pmatrix}
        0 & 0 & 1  \\ 
      \end{pmatrix} 
      \begin{pmatrix}
        0 & c & 0  \\
        c & 0 & 0  \\
        0 & 0 & c  
      \end{pmatrix} 
      \begin{pmatrix}
        0  \\
        0  \\
        1  
      \end{pmatrix} \\
      &=c 
    \end{aligned}  
    \]
\end{frame} 

\begin{frame}
  \frametitle{}
  3)能级的二级修正,
  \[ E^{(2)}_n = \sum_{m\ne n}  \frac{|H ^{\prime} _{mn}|^2}{E_n^{(0)}-E_m^{(0)}} \]
  由于各微扰矩阵元$H ^{\prime} _{mn}$已知,可进行直接计算
  \[ \begin{aligned}
    E^{(2)}_1 &= \frac{|H ^{\prime} _{21}|^2}{E_1^{(0)}-E_2^{(0)}} + \frac{|H ^{\prime} _{31}|^2}{E_1^{(0)}-E_3^{(0)}} \\ 
    &= \frac{c^2}{1-3} + \frac{0}{1-(-2)} \\
    &= - \frac{1}{2}c^2
  \end{aligned}  \]
\end{frame} 

\begin{frame}
  \frametitle{}
  \[ \begin{aligned}
    E^{(2)}_2 &= \frac{|H ^{\prime} _{12}|^2}{E_2^{(0)}-E_1^{(0)}} + \frac{|H ^{\prime} _{32}|^2}{E_2^{(0)}-E_3^{(0)}} \\ 
    &= \frac{c^2}{3-1} + \frac{0}{3-(-2)} \\
    &= \frac{1}{2}c^2
  \end{aligned}  \]
  \[ \begin{aligned}
    E^{(2)}_3 &= \frac{|H ^{\prime} _{13}|^2}{E_3^{(0)}-E_1^{(0)}} + \frac{|H ^{\prime} _{23}|^2}{E_3^{(0)}-E_2^{(0)}} \\ 
    &= \frac{0}{-2-1} + \frac{0}{-2-3} \\
    &= 0
  \end{aligned}  \]
\end{frame} 

\begin{frame}
  \frametitle{}
因此,二级修正下各能级为 
\[ E_1 = E^{(0)}_1 + E^{(1)}_1 + E^{(2)}_1 = 1 - \frac{1}{2}c^2 \]
\[ E_2 = E^{(0)}_2 + E^{(1)}_2 + E^{(2)}_2 = 3 + \frac{1}{2}c^2 \]
\[ E_3 = E^{(0)}_3 + E^{(1)}_3 + E^{(2)}_3 = -2 + c \]
\end{frame} 

\begin{frame}
  \frametitle{}
 4)波函数的一级修正
 $$
\left|\psi_n^{(1)}\right\rangle = \sum_{m\ne n} \frac{H^{\prime}_{mn}}{E_n^{(0)}-E_m^{(0)}} \left|\psi_m^{(0)}\right\rangle  
$$ 
由于各微扰矩阵元$H ^{\prime} _{mn}$已知,可进行直接计算
$$
\left|\psi_1^{(1)}\right\rangle = \frac{H^{\prime}_{21}}{E_1^{(0)}-E_2^{(0)}} \left|\psi_2^{(0)}\right\rangle + \frac{H^{\prime}_{31}}{E_1^{(0)}-E_3^{(0)}} \left|\psi_3^{(0)}\right\rangle  = -\frac{c}{2} \left|\psi_2^{(0)}\right\rangle  
$$ 
$$
\left|\psi_2^{(1)}\right\rangle = \frac{H^{\prime}_{12}}{E_2^{(0)}-E_1^{(0)}} \left|\psi_1^{(0)}\right\rangle + \frac{H^{\prime}_{32}}{E_2^{(0)}-E_3^{(0)}} \left|\psi_3^{(0)}\right\rangle  = \frac{c}{2} \left|\psi_1^{(0)}  \right\rangle  \quad 
$$ 
$$
\left|\psi_3^{(1)}\right\rangle = \frac{H^{\prime}_{13}}{E_3^{(0)}-E_1^{(0)}} \left|\psi_1^{(0)}\right\rangle + \frac{H^{\prime}_{23}}{E_3^{(0)}-E_2^{(0)}} \left|\psi_2^{(0)}\right\rangle  = 0 \qquad \qquad 
$$ 
\end{frame} 

\begin{frame}
  \frametitle{}
  一级修正下的波函数为
  $$ \begin{aligned}
    &\left|\psi_1 \right\rangle = \left|\psi_1^{(0)}\right\rangle + \left|\psi_1^{(1)}\right\rangle  =  \begin{pmatrix}
      1  \\
      0  \\
      0  
    \end{pmatrix}  -\frac{c}{2}  \begin{pmatrix}
      0  \\
      1  \\
      0  
    \end{pmatrix} \\
    &\left|\psi_2 \right\rangle = \left|\psi_2^{(0)}\right\rangle + \left|\psi_2^{(1)}\right\rangle  =  \begin{pmatrix}
      1  \\
      0  \\
      0  
    \end{pmatrix}  + \frac{c}{2}  \begin{pmatrix}
      0  \\
      1  \\
      0  
    \end{pmatrix} \\
    &\left|\psi_2 \right\rangle = \left|\psi_2^{(0)}\right\rangle + \left|\psi_2^{(1)}\right\rangle  =  \begin{pmatrix}
      0  \\
      0  \\
      1  
    \end{pmatrix}
  \end{aligned}$$
\end{frame} 

\begin{frame}
  \frametitle{}
5)精确求解 \\
设$H$的本征方程为
$$ \begin{bmatrix}
  1 & c & 0  \\
  c & 3 & 0  \\
  0 & 0 & c-2 \\ 
\end{bmatrix}  \begin{bmatrix}
  a_1 \\
  a_2 \\
  a_3 \\ 
\end{bmatrix} = E \begin{bmatrix}
  a_1 \\
  a_2 \\
  a_3 \\ 
\end{bmatrix}
$$ 
解九期方程
$$ \begin{bmatrix}
  1-E & c & 0  \\
  c & 3-E & 0  \\
  0 & 0 & c-2-E \\ 
\end{bmatrix} =0 $$
$$ \implies (c-2-E)(E^2-4E+3-c^2) =0 $$
\end{frame} 

\begin{frame}
  \frametitle{}
解得本征能量的精确解
$$ E_1 = 2- \sqrt{1+c^2}, \quad  E_2 = 2 + \sqrt{1+c^2}, \quad  E_3 = -2 + c $$
由于$c$是小量,做级数展开,有
$$ E_1 = 1-\frac{1}{2}c^2 + \frac{1}{8} c^4 + \cdots $$
$$ E_1 = 3+\frac{1}{2}c^2 - \frac{1}{8} c^4 + \cdots $$
微扰的二级修正解与精确解不计$c^4$及更高阶项时的结果相同!
\end{frame} 

\subsection{简并定态微扰}

\begin{frame}
  \frametitle{简并定态微扰}
  ~~\\ 
若$H^{(0)}$的能量本征值$ E_n^{(0)}$ 是简并的,则构成简并定态微扰问题。\\

~~\\ 
设简并度为$f$, 简并本征函数记为
$$ \left\vert n_1 \right\rangle, \left\vert n_2 \right\rangle,\cdots \left\vert n_f \right\rangle $$
它们都满足本征方程
\[ H^{(0)} \left\vert n_\alpha \right\rangle = E_n^{(0)} \left\vert n_\alpha \right\rangle, \qquad (\alpha =1, 2, \cdots, f) \]
\end{frame} 

\begin{frame}
  \frametitle{}
这$f$简并本征函数构成的线性叠加态
\begin{equation}\label{eq:jbdj}
\left\vert \psi ^{(0)}_n \right\rangle = \sum_{\alpha=1}^f
c_\alpha \left\vert n_\alpha \right\rangle, \qquad \sum |c_\alpha|^2 =1 
\end{equation}
依然满足本征方程
\[ H^{(0)} \left\vert \psi ^{(0)}_n \right\rangle = E_n^{(0)} \left\vert \psi ^{(0)}_n \right\rangle \]
即
\[ \sum_{\alpha=1}^f
c_\alpha H^{(0)} \left\vert n_\alpha \right\rangle =  \sum_{\alpha=1}^f
c_\alpha E_n^{(0)}\left\vert n_\alpha \right\rangle\]
\end{frame} 

\begin{frame}
  \frametitle{}
把叠加态代入一级修正方程[\ref{eq:yjxz}]
\begin{equation*}
\begin{aligned}
  \left[\hat{H}^{(0)}-E_n^{(0)}\right]\left|\psi_n^{(1)}\right\rangle &= -\left[\hat{H}^{\prime}-E_n^{(1)}\right]\left| \psi_n^{(0)}\right\rangle \\ 
  &= -\left[\hat{H}^{\prime}-E_n^{(1)}\right]\sum_{\alpha=1}^f
  c_\alpha \left\vert n_\alpha \right\rangle \\
  &=\sum_{\alpha=1}^f
  c_\alpha E_n^{(1)}\left\vert n_\alpha \right\rangle  -\sum_{\alpha=1}^f
  c_\alpha \hat{H}^{\prime}\left\vert n_\alpha \right\rangle 
\end{aligned}
\end{equation*}
左乘$\left\langle n_\beta \right\vert$
\[\left\langle n_\beta \right\vert \left[\hat{H}^{(0)}-E_n^{(0)}\right]\left|\psi_n^{(1)}\right\rangle = \sum_{\alpha=1}^f
c_\alpha E_n^{(1)}\left\langle n_\beta \left\vert n_\alpha \right\rangle\right. -\sum_{\alpha=1}^f
c_\alpha \left\langle n_\beta \right\vert \hat{H}^{\prime}\left\vert n_\alpha \right\rangle \]
\end{frame} 

\begin{frame}
  \frametitle{}
计算
\[
  \begin{aligned}
    c_\alpha \sum_{\alpha=1}^f E_n^{(1)} \delta _{\alpha \beta} - c_\alpha \sum_{\alpha=1}^f H^{\prime}_{\beta\alpha } &=0 
  \end{aligned} 
   \]
  整理,得矩阵方程
  \begin{equation}\label{eq:jqfc}
    \begin{pmatrix}H^{\prime}_{11} -E^{(1)}_n &H^{\prime}_{12}&\cdots&H^{\prime}_{1f}\\
      H^{\prime}_{21}&H^{\prime}_{22}-E^{(1)}_n&\cdots&H^{\prime}_{2f}\\\vdots&\vdots&\ddots&\vdots\\H^{\prime}_{f 1}&H^{\prime}_{f 2}&\cdots&H^{\prime}_{ff}-E^{(1)}_n\end{pmatrix}
      \begin{pmatrix}c_{\alpha 1} \\c_{\alpha 2} \\ \cdots \\c_{\alpha f} \end{pmatrix} =0
  \end{equation}
\end{frame} 

\begin{frame}
  \frametitle{}
  系数行列式为零
\[\begin{bmatrix}H^{\prime}_{11} -E^{(1)}_n &H^{\prime}_{12}&\cdots&H^{\prime}_{1f}\\
  H^{\prime}_{21}&H^{\prime}_{22}-E^{(1)}_n&\cdots&H^{\prime}_{2f}\\\vdots&\vdots&\ddots&\vdots\\H^{\prime}_{f 1}&H^{\prime}_{f 2}&\cdots&H^{\prime}_{ff}-E^{(1)}_n\end{bmatrix} =0\]
  解久期方程,得$E_n^{(1)}$的$f$个根(能量一级修正)
  \[ E_{n1} ^{(1)}, \quad  E_{n2} ^{(1)}, \quad\cdots, E_{nk} ^{(1)}, \cdots\quad E_{nf} ^{(1)}\]
  若无重根,则简并完全消除。若有部分重根则简并部分消除,有必要考虑二级或更高级的修正才能实现简并完全消除。

  ~~\\ 
  把根依次代回矩阵方程[\ref{eq:jqfc}],得$f$个$c_\alpha$, 记为 
  \[ c_{\alpha 1} ^{(1)}, \quad  c_{\alpha 2} ^{(1)}, \quad\cdots, c_{\alpha k} ^{(1)}, \cdots\quad c_{\alpha f} ^{(1)}\]
\end{frame} 

\begin{frame}
  \frametitle{}
代回式[\ref{eq:jbdj}],得$f$个零级近似波函数
\[ \left\vert \psi ^{(0)}_{nk} \right\rangle = \sum_{\alpha=1}^f
c_{\alpha k} \left\vert n_\alpha \right\rangle  , \quad k =1,2,\cdots, f\]

~~\\ 
$f$个简并的零级波函数
$$ \left\vert n_1 \right\rangle, \left\vert n_2 \right\rangle,\cdots \left\vert n_f \right\rangle $$
变成了$f$个非简并零级近似波函数
$$ \left\vert \psi ^{(0)}_{n 1} \right\rangle, \left\vert \psi ^{(0)}_{n2} \right\rangle,\cdots \left\vert \psi ^{(0)}_{nf} \right\rangle  $$
如果微扰致使简并完全消除,则这$f$个新的零级本征函数都是非简并的!代入非简并公式,可进一步求能量的二级修正和波函数的一级修正。
\end{frame} 

\begin{frame}
  \frametitle{斯塔克效应}
  \emf[定义:]氢原子在外电场作用下产生谱线分裂的现象称为$Stark$效应。\\
~~\\ 
\emf[定性分析:]电子受球对称库仑场作用,造成第$n$能级$n^2$度简并。外电场的存在破坏了原有势场的对称性,简并消除,导致谱线分裂。

~~\\ 
\emf[定量计算:]\\
库仑场 $U=-\frac{e^2_s}{r}$,$\quad $外场势 $H'=e\vec{\varepsilon}\cdot \vec{r} = e \varepsilon r \cos\theta$

$r$很小,内场远大于外场,$n$不大,能级间隔大,满足微扰法条件 
$$E_n^{(0)} = - \frac{\mu Z^2 e_s^4}{2\hbar^2}\frac{1}{n^2}$$
\end{frame} 

\begin{frame}
  \frametitle{}
(1) 写出简并态:考虑谱线$E_2^{(0)}\to E_1^{(0)}$, $E_2^{(0)}$四度简并,记为
\begin{itemize}
  \item $ \left\vert 2_1 \right\rangle = \psi _{200} =\mathscr{R}_{20}(r) Y_{00}$ 
  \item $ \left\vert 2_2 \right\rangle = \psi _{210} =\mathscr{R}_{21}(r) Y_{10} $ 
  \item $ \left\vert 2_3 \right\rangle = \psi _{211}=\mathscr{R}_{21}(r) Y_{11}  $ 
  \item $ \left\vert 2_4 \right\rangle = \psi _{21-1} =\mathscr{R}_{21}(r) Y_{1-1} $
\end{itemize}
\end{frame} 

\begin{frame}
  \frametitle{}
  (2) 求微扰矩阵$\{H'_{nm}\}$
  $$ \begin{pmatrix}
    \left\langle 2_1 \right\vert H' \left\vert 2_1 \right\rangle & \left\langle 2_1 \right\vert H' \left\vert 2_2 \right\rangle & \left\langle 2_1 \right\vert H' \left\vert 2_3 \right\rangle & \left\langle 2_1 \right\vert H' \left\vert 2_4 \right\rangle \\
    \left\langle 2_2 \right\vert H' \left\vert 2_1 \right\rangle & \left\langle 2_2 \right\vert H' \left\vert 2_2 \right\rangle & \left\langle 2_2 \right\vert H' \left\vert 2_3 \right\rangle & \left\langle 2_2 \right\vert H' \left\vert 2_4 \right\rangle \\
    \left\langle 2_3 \right\vert H' \left\vert 2_1 \right\rangle & \left\langle 2_3 \right\vert H' \left\vert 2_2 \right\rangle & \left\langle 2_3 \right\vert H' \left\vert 2_3 \right\rangle & \left\langle 2_3 \right\vert H' \left\vert 2_4 \right\rangle \\
    \left\langle 2_4 \right\vert H' \left\vert 2_1 \right\rangle & \left\langle 2_4 \right\vert H' \left\vert 2_2 \right\rangle & \left\langle 2_4 \right\vert H' \left\vert 2_3 \right\rangle & \left\langle 2_4 \right\vert H' \left\vert 2_4 \right\rangle 
  \end{pmatrix} $$
  细节
  $$
  \begin{aligned}
    H'_{21} &= \left\langle 2_2 \right\vert H' \left\vert 2_1 \right\rangle   
     \\ 
     &= \left\langle \mathscr{R}_{21}Y_{10} \right\vert e \varepsilon r \cos\theta \left\vert \mathscr{R}_{20} Y_{00} \right\rangle   
     \\ 
     &= e \varepsilon\left\langle \mathscr{R}_{21}  \right\vert  r \left\vert \mathscr{R}_{20} \right\rangle  \left\langle  Y_{10} \right\vert \cos\theta \left\vert Y_{00} \right\rangle   
     \\ 
  \end{aligned}
  $$ 
\end{frame} 

\begin{frame}
  \frametitle{}
  先求一般式 $ \left\langle  Y_{l'm'} \right\vert \cos\theta \left\vert Y_{lm} \right\rangle $  \\
  代入递推式
  $$
   \cos\theta \left\vert Y_{lm} \right\rangle  = \sqrt{\frac{(l+1)^2-m^2}{(2l+1)(2l+3)}}\left\vert Y_{l+1,m} \right\rangle  + \sqrt{\frac{l^2-m^2}{(2l-1)(2l+1)}}\left\vert Y_{l-1,m} \right\rangle  
  $$ 
  有:
  $$
  \begin{aligned}
   & \left\langle  Y_{l'm'} \right\vert \cos\theta \left\vert Y_{lm} \right\rangle \\
    & \qquad = \sqrt{\frac{(l+1)^2-m^2}{(2l+1)(2l+3)}}\left\langle Y_{l'm'} \left\vert Y_{l+1,m} \right\rangle \right. + \sqrt{\frac{l^2-m^2}{(2l-1)(2l+1)}} \left\langle Y_{l'm'} \left\vert Y_{l-1,m} \right\rangle \right. \\
    &\qquad = \sqrt{\frac{(l+1)^2-m^2}{(2l+1)(2l+3)}} \delta _{l'l+1} \delta _{m'm} + \sqrt{\frac{l^2-m^2}{(2l-1)(2l+1)}} \delta _{l'l-1}\delta _{m'm} 
  \end{aligned}
  $$ 
  上式不为零,要求$l'=l\pm 1$且$m'=m$,发现只有$H'_{12}, H'_{21}$满足条件
\end{frame} 

\begin{frame}
  \frametitle{}
  $$ 
  \begin{aligned}
   \left\langle  Y_{10} \right\vert \cos\theta \left\vert Y_{00} \right\rangle 
   &= \int_{0}^{\pi}\sqrt{\frac{3}{4\pi} }\cos\theta \cdot \cos\theta \cdot \frac{1}{\sqrt{4\pi} } \cdot\sin\theta d \theta \cdot \int_{0}^{2\pi}  d\varphi \\
   &= \frac{\sqrt{3}}{2}\int_{0}^{\pi} \cos^2\theta  \sin \theta d \theta \\
   &= \frac{1}{\sqrt{3} }
  \end{aligned}
  $$   
  $$ 
  \begin{aligned}
    H'_{12} = H'_{21} &= e \varepsilon\left\langle \mathscr{R}_{21}  \right\vert  r \left\vert \mathscr{R}_{20} \right\rangle  \left\langle  Y_{10} \right\vert \cos\theta \left\vert Y_{00} \right\rangle   
     \\ 
     &= \frac{e \varepsilon }{\sqrt{3}} \left\langle \mathscr{R}_{21}  \right\vert  r \left\vert \mathscr{R}_{20} \right\rangle  \\
     &= \frac{e \varepsilon }{\sqrt{3}} \int_{0}^{\infty} \left(\frac{1}{2 a_0}\right)^{3 / 2} \frac{r}{a_0 \sqrt{3}} \mathrm{e}^{-\frac{r}{2 a_0}} \cdot r \cdot \left(\frac{1}{2 a_0}\right)^{3 / 2}\left(2-\frac{r}{a_0}\right) \mathrm{e}^{-\frac{r}{2 a_0}}  \cdot  r^2 dr \\
     &=  \frac{e \varepsilon }{24} (\frac{1}{a_0 })^4  \int_{0}^{\infty} \left(2-\frac{r}{a_0}\right) \mathrm{e}^{-\frac{r}{a_0}} r^4 dr = -3e \varepsilon a_0 \\ 
  \end{aligned}
  $$ 
\end{frame} 

\begin{frame}
  \frametitle{}
(3) 解轨期方程
$$ \begin{bmatrix}
  -E^{(1)}_2 & -3e\varepsilon a_0 & 0 & 0 \\
  -3e\varepsilon a_0 & -E^{(1)}_2 & 0 & 0 \\
  0 & 0 & -E^{(1)}_2 & 0 \\
  0 & 0 & 0 & -E^{(1)}_2 \\  
\end{bmatrix} = 0 $$
得简并能级$E^{(1)}_2$的四个一级修正 
$$E^{(1)}_{21} = 3e\varepsilon a_0, \quad E^{(1)}_{22} = - 3e\varepsilon a_0, \quad E^{(1)}_{23} = 0, \quad E^{(1)}_{24} = 0 $$
\end{frame} 

\begin{frame}
  \frametitle{}
(5)求$E^{(0)}_1 $的能量一级修正 \\
这是非简并能级,一级修正是平均值
$$ 
\begin{aligned}
  H'_{11} &= \left\langle R^{(0)}_{11}Y_{00} 
 \right\vert e \varepsilon r \cos\theta \left\vert R^{(0)}_{11}Y_{00}  \right\rangle  \\ 
 &= \left\langle R^{(0)}_{11} 
 \right\vert e \varepsilon r \left\vert R^{(0)}_{11} \right\rangle   \left\langle Y_{00} 
 \right\vert \cos\theta \left\vert Y_{00}  \right\rangle \\
 &= 0 
\end{aligned}
$$ 
\end{frame} 

\begin{frame}
  \frametitle{}
$\star$ 解释$stark$ 效应
\begin{figure}[htbp]
  \centering
  \includegraphics[width=0.9\textwidth]{figs/stocke.png}
  %\caption{}
    %\label{fig:}
\end{figure}
基态能量不变,第一激发态4度简并,电场导致部分去简并,谱线由一变三!
\end{frame} 

\begin{frame}
  \frametitle{}
(6) 求零级近似波函数\\
把$E^{(1)}_2$的四个一级修正值 
$$E^{(1)}_{21} = 3e\varepsilon a_0, \quad E^{(1)}_{22} = - 3e\varepsilon a_0, \quad E^{(1)}_{23} = 0, \quad E^{(1)}_{24} = 0 $$
依次代入式[\ref{eq:jqfc}],即下式 
\begin{equation*}
  \begin{pmatrix}-E^{(1)}_2 & -3e\varepsilon a_0 & 0 & 0 \\
    -3e\varepsilon a_0 & -E^{(1)}_2 & 0 & 0 \\
    0 & 0 & -E^{(1)}_2 & 0 \\
    0 & 0 & 0 & -E^{(1)}_2 \end{pmatrix}
    \begin{pmatrix}c_{\alpha 1} \\c_{\alpha 2} \\ c_{\alpha 3} \\c_{\alpha 4} \end{pmatrix} =0
\end{equation*}
可得四全零级近似波函数,
\end{frame} 

\begin{frame}
  \frametitle{}
  *取$E^{(1)}_2=E^{(1)}_{21} = 3e\varepsilon a_0 $,得
  \begin{equation*}
    \begin{pmatrix}-3e\varepsilon a_0 & -3e\varepsilon a_0 & 0 & 0 \\
      -3e\varepsilon a_0 & -3e\varepsilon a_0 & 0 & 0 \\
      0 & 0 & -3e\varepsilon a_0 & 0 \\
      0 & 0 & 0 & -3e\varepsilon a_0 \end{pmatrix}
      \begin{pmatrix}c_{2 1} \\c_{2 2} \\ c_{2 3} \\c_{2 4} \end{pmatrix} =0
  \end{equation*}
  解得 
  \[ c_{2 1} = - c_{2 2} =c, \qquad c_{2 3} = c_{2 4} =0 \]
  代入式[\ref{eq:jbdj}],有
  \begin{equation*}
    \left\vert \psi ^{(0)}_{21} \right\rangle = \sum_{\alpha=1}^4
    c_\alpha \left\vert n_\alpha \right\rangle = c \left\vert 2_1 \right\rangle
    -c \left\vert 2_2 \right\rangle \end{equation*}
    归一化
    \begin{equation*}
      \left\vert \psi ^{(0)}_{21} \right\rangle = \frac{1}{\sqrt{2} } \left\vert \psi _{200} \right\rangle
      -\frac{1}{\sqrt{2} } \left\vert\psi _{210} \right\rangle \end{equation*}
\end{frame} 
 
\begin{frame}
  \frametitle{}
  *取$E^{(1)}_2=E^{(1)}_{22} = -3e\varepsilon a_0 $,得
  \begin{equation*}
    \begin{pmatrix}3e\varepsilon a_0 & -3e\varepsilon a_0 & 0 & 0 \\
      -3e\varepsilon a_0 & 3e\varepsilon a_0 & 0 & 0 \\
      0 & 0 & 3e\varepsilon a_0 & 0 \\
      0 & 0 & 0 & 3e\varepsilon a_0 \end{pmatrix}
      \begin{pmatrix}c_{2 1} \\c_{2 2} \\ c_{2 3} \\c_{2 4} \end{pmatrix} =0
  \end{equation*}
  解得 
  \[ c_{2 1} = c_{2 2} =c, \qquad c_{2 3} = c_{2 4} =0 \]
  代入式[\ref{eq:jbdj}],有
  \begin{equation*}
    \left\vert \psi ^{(0)}_{22} \right\rangle = \sum_{\alpha=1}^4
    c_\alpha \left\vert n_\alpha \right\rangle = c \left\vert 2_1 \right\rangle
    +c \left\vert 2_2 \right\rangle \end{equation*}
    归一化
    \begin{equation*}
      \left\vert \psi ^{(0)}_{21} \right\rangle = \frac{1}{\sqrt{2} } \left\vert \psi _{200} \right\rangle
      +\frac{1}{\sqrt{2} } \left\vert\psi _{210} \right\rangle \end{equation*}
\end{frame}  

\begin{frame}
  \frametitle{}
  *取$E^{(1)}_2=E^{(1)}_{23} = E^{(1)}_{24}  = 0 $,得
  \begin{equation*}
    \begin{pmatrix}0 & -3e\varepsilon a_0 & 0 & 0 \\
      -3e\varepsilon a_0 & 0 & 0 & 0 \\
      0 & 0 & 0 & 0 \\
      0 & 0 & 0 & 0 \end{pmatrix}
      \begin{pmatrix}c_{2 1} \\c_{2 2} \\ c_{2 3} \\c_{2 4} \end{pmatrix} =0
  \end{equation*}
  解得 
  \[ c_{2 1} = c_{2 2} =0, \qquad c_{2 3} =a, c_{2 4} =b \]
  代入式[\ref{eq:jbdj}],有
  \begin{equation*}
    \left\vert \psi ^{(0)}_{23}/\psi ^{(0)}_{24} \right\rangle = \sum_{\alpha=1}^4
    c_\alpha \left\vert n_\alpha \right\rangle = a \left\vert 2_3 \right\rangle
    +b \left\vert 2_4 \right\rangle \end{equation*}
\end{frame} 

\begin{frame}
  \frametitle{}
不防取 
\begin{equation*}
  \left\vert \psi ^{(0)}_{23} \right\rangle = \left\vert 2_3 \right\rangle = \left\vert\psi _{211} \right\rangle \end{equation*}
\begin{equation*}
  \left\vert \psi ^{(0)}_{24} \right\rangle = \left\vert 2_4 \right\rangle = \left\vert\psi _{21-1} \right\rangle \end{equation*}
得四个零级近似波函数
\begin{equation*}
  \begin{aligned}
    \left\vert \psi ^{(0)}_{21} \right\rangle &= \frac{1}{\sqrt{2} } \left\vert \psi _{200} \right\rangle
    -\frac{1}{\sqrt{2} } \left\vert\psi _{210} \right\rangle \\ 
    \left\vert \psi ^{(0)}_{22} \right\rangle &= \frac{1}{\sqrt{2} } \left\vert \psi _{200} \right\rangle
    +\frac{1}{\sqrt{2} } \left\vert\psi _{210} \right\rangle \\ 
    \left\vert \psi ^{(0)}_{23} \right\rangle &= \left\vert\psi _{211} \right\rangle \\ 
    \left\vert \psi ^{(0)}_{24} \right\rangle &= \left\vert\psi _{21-1} \right\rangle \\
  \end{aligned} 
\end{equation*}
\end{frame} 

\begin{frame}
  \frametitle{}
  (7) 讨论\\
  氢原子基态,正负电荷中心重合,电场无其无影响。氢原子第一激发态, 正负电荷中心不重合, 这是电偶极矩($\vec{D}$), 在外电场($\vec{\varepsilon}$)中有四种可能取向 \\
  \begin{itemize}
    \Item $\vec{D}$与$\vec{\varepsilon}$反向, $H' = - \vec{D} \cdot \vec{\varepsilon} = -D\varepsilon \cos\pi = 3 e \varepsilon a_0$,零级近似波函数为 
    $$
    \left\vert \psi ^{(0)}_{21} \right\rangle = \frac{1}{\sqrt{2} } \left\vert \psi _{200} \right\rangle -\frac{1}{\sqrt{2} } \left\vert\psi _{210} \right\rangle  
    $$ 
    \Item $\vec{D}$与$\vec{\varepsilon}$ 同向, $H' = -D\varepsilon \cos 0 = - 3 e \varepsilon a_0$,零级近似波函数
    $$
    \left\vert \psi ^{(0)}_{21} \right\rangle = \frac{1}{\sqrt{2} } \left\vert \psi _{200} \right\rangle + \frac{1}{\sqrt{2} } \left\vert\psi _{210} \right\rangle 
    $$ 
    \Item $\vec{D}$与$\vec{\varepsilon}$ 相互垂直, $H' = -D\varepsilon \cos \frac{\pi}{2} = 0$, 电场对氢原子状态无影响 \\
    \begin{itemize}
      \item $\vec{D}$处于$x$轴向 $\left\vert \psi ^{(0)}_{23} \right\rangle = \left\vert\psi _{211} \right\rangle$
      \item $\vec{D}$处于$y$轴向 $\left\vert \psi ^{(0)}_{24} \right\rangle = \left\vert\psi _{21-1} \right\rangle$
    \end{itemize}
  \end{itemize}
\end{frame} 

\begin{frame}
  \frametitle{}
\例[7]{设某粒子的哈密顿为
$$ H = \begin{bmatrix}
  2 & 0 & c  \\
  0 & 2 & 0  \\
  c & 0 & 2 \\ 
\end{bmatrix}, \qquad c\ll 1 
$$
试求:(1)粒子能级到一级修正和波函数的零级修正}
\解 改写哈密顿
$$ H = H^{(0)} + H' = \begin{bmatrix}
  2 & 0 & 0  \\
  0 & 2 & 0  \\
  0 & 0 & 2 \\ 
\end{bmatrix} + \begin{bmatrix}
  0 & 0 & c  \\
  0 & 0 & 0  \\
  c & 0 & 0  
\end{bmatrix} 
$$
\end{frame}

\begin{frame}
  \frametitle{}
  1)$ H^{(0)}  $ 的本征值(三重简并)为
  \[ E = 2\]  
  对应的本征函数为
  \[ \left\vert 1\right\rangle  = \begin{bmatrix}
    1  \\
    0  \\
    0  
  \end{bmatrix} , \quad \left\vert 2 \right\rangle  = \begin{bmatrix}
    0  \\
    1  \\
    0  
  \end{bmatrix} , \quad
  \left\vert 3 \right\rangle  = \begin{bmatrix}
    0  \\
    0  \\
    1  
  \end{bmatrix} \]
  构造零级近似波函数
  $$\left\vert \psi ^{(0)} \right\rangle = \sum_{\alpha=1}^3 a_\alpha \left\vert \alpha \right\rangle $$
\end{frame} 

\begin{frame}
  \frametitle{}
系数$a_\alpha$满足方程
\begin{equation}\label{eq:alpha}
  \begin{pmatrix}
    -E^{(1)} & 0 & c \\
    0 & -E^{(1)} & 0 \\
    c & 0 & -E^{(1)} \end{pmatrix}
    \begin{pmatrix}a_{1} \\a_{2} \\ a_{3} \end{pmatrix} =0
\end{equation}
2)由系数行列式为零,解得三个根
\[E_{1}^{(1)} = -c, \qquad E_{2}^{(1)} = 0, \qquad E_{3}^{(1)} = c\]
它们是能量一级修正,因此简并完全消除!
\end{frame} 

\begin{frame}
  \frametitle{}
3)零级近似波函数 \\
把$E_{1}^{(1)} = -c$代回方程[\ref{eq:alpha}],有 
\begin{equation*}
  \begin{pmatrix}
    c & 0 & c \\
    0 & c & 0 \\
    c & 0 & c \end{pmatrix}
    \begin{pmatrix}a_{1} \\a_{2} \\ a_{3 } \end{pmatrix} =0
\end{equation*}
解得 
\[ a_{1} = - a_{3} =a, \quad a_{2} =0 \]
得第一个零级近似波函数
$$\left\vert \psi ^{(0)}_{1} \right\rangle = \sum_{\alpha=1}^3 a_\alpha \left\vert \alpha \right\rangle  = a \left\vert 1 \right\rangle -a \left\vert 3 \right\rangle = \frac{1}{\sqrt{2}} \left\vert 1 \right\rangle - \frac{1}{\sqrt{2}} \left\vert 3 \right\rangle = \frac{1}{\sqrt{2}} \begin{pmatrix}
  1 \\
  0  \\
  -1 \end{pmatrix}$$
\end{frame} 

\begin{frame}
  \frametitle{}
  把$E_{2}^{(1)} = 0 $代回方程[\ref{eq:alpha}],有 
  \begin{equation*}
    \begin{pmatrix}
      0 & 0 & c \\
      0 & 0 & 0 \\
      c & 0 & 0 \end{pmatrix}
      \begin{pmatrix}a_{1} \\a_{2} \\ a_{3 } \end{pmatrix} =0
  \end{equation*}
  解得 
  \[ a_{1} =  a_{3} =0, \quad a_{2} =a \]
  得第二个零级近似波函数
  $$\left\vert \psi ^{(0)}_{2} \right\rangle = a\left\vert 2 \right\rangle = \begin{pmatrix}
    0 \\
    1  \\
    0 \end{pmatrix} $$
\end{frame} 

\begin{frame}
  \frametitle{}
  把$E_{3}^{(1)} = c$代回方程[\ref{eq:alpha}],有 
  \begin{equation*}
    \begin{pmatrix}
      -c & 0 & c \\
      0 & -c & 0 \\
      c & 0 & -c \end{pmatrix}
      \begin{pmatrix}a_{1} \\a_{2} \\ a_{3 } \end{pmatrix} =0
  \end{equation*}
  解得 
  \[ a_{1} =  a_{3} =a, \quad a_{2} =0 \]
  得第三个零级近似波函数
  $$\left\vert \psi ^{(0)}_{3} \right\rangle = \frac{1}{\sqrt{2}} \left\vert 1 \right\rangle + \frac{1}{\sqrt{2}} \left\vert 3 \right\rangle = \frac{1}{\sqrt{2}} \begin{pmatrix}
    1 \\
    0  \\
    1 \end{pmatrix} $$
\end{frame} 

\begin{frame}
  \frametitle{}
因此, 一级修正下的能级为
\[E_{1} = 2-c, \qquad E_{2} = 2, \qquad E_{3} = 2+ c\]
零级近似波函数为
$$
\begin{aligned}
  \left\vert \psi ^{(0)}_{1} \right\rangle  = \frac{1}{\sqrt{2}} \begin{pmatrix}
    1 \\
    0  \\
    -1 \end{pmatrix}, \quad  \left\vert \psi ^{(0)}_{2} \right\rangle  = \begin{pmatrix}
      0 \\
      1  \\
      0 \end{pmatrix},  \quad \left\vert \psi ^{(0)}_{3} \right\rangle  =\frac{1}{\sqrt{2}} \begin{pmatrix}
        1 \\
        0  \\
        1 \end{pmatrix}  
\end{aligned}
$$
\end{frame} 

\section{含时微扰}
\subsection{$H^{(0)}$表象薛定谔方程}
\begin{frame} 
  \frametitle{跃迁问题}
实际体系的哈密顿不显含时间,若可写成
\[H = H^{(0)} + H'\]
有$H^{(0)} \gg H'$, 且理想体系的能级间隔较远时,采用定态微扰法可使方程
\[ H\psi _n = E_n \psi _n\]
得解。因此有 
\[ \Psi(\vec{r},t_0) = \sum_n a_n(t_0) \psi _n\]
\[ 
\begin{aligned}
  \Psi(\vec{r},t)
  &= \sum_n a_n(t_0)e^{-\frac{i}{\hbar}E_n t} \psi _n (\vec{r}) \\
  &= \sum_n U(t,t_0) \psi _n (\vec{r}) 
\end{aligned} \]
\end{frame} 

\begin{frame} 
  \frametitle{}
即:若实际体系哈密顿不显含时间则是封闭体系, 其波函数随时间的演化只是一种幺正变换。

~~\\ 
若实际体系哈密顿显含时间, 则不再是封闭体系。也就是说体系与环境之间存在能量交换 
\begin{itemize}
  \Item 获得能量(吸收光子), 体系被激发到高能态
  \Item 失去能量(放出光子), 体系回到低能态
\end{itemize}
因此,含时微扰是微扰导致理想体系发生跃迁的问题。
\end{frame} 

\begin{frame} 
  \frametitle{ $H^{(0)}$表象}
设实际体系的哈密顿可写成 ($H'(t)\ll H^{(0)} $)
\[H(t) = H^{(0)} + H'(t)\]
若理想体系已得解
\[H^{(0)}\phi _n = \varepsilon _n \phi _n \]
则$\{\phi _n\}$构成 $H^{(0)}$表象的基。 \\
~~\\ 
本征函数演化到$t$时刻为
\[ \Phi _n(t) =  \phi _n e^{-\frac{i}{\hbar}\varepsilon _n t}\]
\end{frame} 

\begin{frame} 
  \frametitle{}
实际体系$t$时刻的波函数可在$H^{(0)}$表象展开
\begin{equation}\label{eq:twave}
  \Psi(\vec{r},t) =\sum_n a_n(t) \Phi _n(t) 
\end{equation}
它服从薛定谔方程
\[ i \hbar \frac{\partial }{\partial t}\Psi =(H^{(0)} +H' )\Psi\]
代入展开式
\[ i \hbar \frac{\partial }{\partial t}\sum_n a_n(t) \Phi _n  =(H^{(0)} +H' )\sum_n a_n(t) \Phi _n \]
\[ i \hbar \sum_n \frac{d a_n(t)}{d t} \Phi _n + \sum_n a_n(t) \left[i \hbar \frac{\partial }{\partial t} \Phi _n\right]  =\sum_n H^{(0)}a_n(t) \Phi _n +  \sum_n H' a_n(t) \Phi _n \]
\end{frame} 

\begin{frame} 
  \frametitle{}
代入 
\[i \hbar \frac{\partial }{\partial t} \Phi _n  = H^{(0)}\Phi _n  \]
得
\[ i \hbar \sum_n \frac{d a_n(t)}{d t} \left\vert \Phi _n  \right\rangle  = \sum_n H' a_n(t) \left\vert \Phi _n  \right\rangle \]
左乘$\left\langle \Phi _m  \right\vert$,得
\[ i \hbar \sum_n \frac{d a_n(t)}{d t} \left\langle \Phi _m \left\vert \Phi _n  \right\rangle\right.  = \sum_n a_n(t)\left\langle \Phi _m  \right\vert H'  \left\vert \Phi _n  \right\rangle \]
\[ \begin{aligned}
  i \hbar \sum_n \frac{d a_n(t)}{d t} \delta _{mn}  &= \sum_n a_n(t)\left\langle \phi _m  \right\vert H'  \left\vert \phi _n   \right\rangle e^{\frac{i}{\hbar}(\varepsilon _m -\varepsilon _n)t} \\ 
  i \hbar \frac{d a_m(t)}{d t}  &= \sum_n a_n(t)\left\langle \phi _m  \right\vert H'  \left\vert \phi _n   \right\rangle e^{i\omega _{mn}{\hbar}t} 
\end{aligned}\]
\end{frame} 

\begin{frame} 
  \frametitle{}
交换$n,m$, 有:
\begin{equation}\label{eq:sdw-h0}
\boxed{i \hbar \frac{d a_n(t)}{d t} = \sum_m a_m(t) H'_{nm}  e^{i \omega _{nm} t} }  
\end{equation}
式中微扰矩阵元定义于$H^{(0)}$表象
\[ H'_{nm} = \left\langle \phi _n  \right\vert H'  \left\vert \phi _m   \right\rangle \]
求解上方程,可得$a_n(t)$,代回式[\ref{eq:twave}], 则实际体系的波函数$\Psi(\vec{r},t)$得解。因此, 称上述方程为$H^{(0)}$表象中的薛定谔方程。
\end{frame} 

\begin{frame} 
  \frametitle{近似求解}
设$t=0$时刻,体系处于某定态$\Phi _k$, 
\begin{equation*}
  \Psi(\vec{r},t)|_{t=0} = \Phi _k = \sum_n \delta _{nk} \Phi _n(t)|_{t=0} 
\end{equation*}
它的展开式为
\begin{equation*}
  \Psi(\vec{r},t)|_{t=0} = \sum_n a_n(t)\Phi _n(t)|_{t=0} 
\end{equation*}
联立两式,有 
\[ a_n(t)|_{t=0} =a_n(0) = \delta _{nk} \]
\end{frame} 

\begin{frame} 
  \frametitle{}
$\bullet$ 零级近似 :\\
取$H'(t)=0$, 则有微扰矩阵 $H'_{nm} =0$, 代入$H^{(0)}$表象中的薛定谔方程[\ref{eq:sdw-h0}], 得
\begin{equation*}
  i \hbar \frac{d a_n(t)}{d t} = \sum_m a_m(t) H'_{nm}  e^{i \omega _{nm} t} =0
  \end{equation*}
解得零级近似 
\[ a^{(0)}_n(t) = c = a_n(0) = \delta _{nk}\]
代入展开式
\begin{equation*}
  \Psi(\vec{r},t) = \sum_n a_n(t)\Phi _n(t) = \sum_n \delta _{nk}\phi _n(\vec{r})e^{-\frac{i}{\hbar}\varepsilon _n t} 
\end{equation*}
\end{frame} 

\begin{frame} 
  \frametitle{}
  $\bullet$ 一级近似 :\\
  把零级近似写成
  \[ a^{(0)}_m(t) = \delta _{mk}\]
  代入$H^{(0)}$表象中的薛定谔方程[\ref{eq:sdw-h0}], 得 
  \begin{equation*}
    \begin{aligned}
      i \hbar \frac{d a_n(t)}{d t} &= \sum_m \delta _{mk} H'_{nm}  e^{i \omega _{nm} t}  \\ 
      &= H'_{nk}  e^{i \omega _{nk} t}
    \end{aligned}
    \end{equation*}
  得一级近似公式
  \[ a^{(1)}_n(t) = \frac{1}{i\hbar}\int_{0}^{t} H'_{nk}  e^{i \omega _{nk} \tau} d\tau\]
\end{frame} 

\begin{frame} 
  \frametitle{}
  代入展开式
  \begin{equation*}
    \begin{aligned}
      \Psi(\vec{r},t) &= \sum_n a_n(t)\Phi _n(t) \\
      &=  \frac{1}{i\hbar} \sum_n \int_{0}^{t} H'_{nk}  e^{i \omega _{nk} \tau} d\tau \phi _n(\vec{r})e^{-\frac{i}{\hbar}\varepsilon _n t} 
    \end{aligned}
  \end{equation*}
  如果把一级近似结果再代回方程$H^{(0)}$表象中的薛定谔方程[\ref{eq:sdw-h0}],可得到二级近似公式,逐级进行,可得更高级近似解。
\end{frame} 

\begin{frame} 
  \frametitle{跃迁概率}
设$t=0$时刻,粒子处于$\Phi _k$态, $t$时刻,粒子处于叠加态
\begin{equation*}
  \Psi(\vec{r},t) = \sum_m a_m(t)\Phi _m(t)
\end{equation*}
则$\left\vert a_m(t)\right\vert ^2$是微扰导致粒子从初态$\Phi _k$态激发到末态$\Phi _m $的概率,取一级近似,有
\begin{equation*}
  \begin{aligned}
    W_{k\to m} &= \left\vert a^{(1)}_m(t)\right\vert ^2 \\
    &= \frac{1}{\hbar^2} \left\vert \int_{0}^{t} H'_{mk}  e^{i \omega _{mk} \tau} d\tau \right\vert ^2 
  \end{aligned}
\end{equation*}
跃迁概率与微扰矩阵元的大小,微扰作用时长及初末态频率差同共决定
\end{frame} 

\subsection{常微扰}

\begin{frame} 
  \frametitle{常微扰}
如果含时微扰可以表示成如下分段函数形式
\[ H'(t) = \left\{
  \begin{aligned}
   &0, \qquad (t<0)\\
   &H'(\vec{r}), \quad (0\le t \le t_1) \\
   &0, \qquad (t > t_1) 
  \end{aligned}\right. \]
称为常微扰

~~\\ 
设$t<t_1$, 一级近似为

\[ a^{(1)}_n(t) = \frac{1}{i\hbar}\int_{0}^{t} H'_{mk}(r)  e^{i \omega _{mk} \tau} d\tau = \frac{H'_{mk}(r)}{i\hbar}\int_{0}^{t} e^{i \omega _{mk} \tau} d\tau \]
\end{frame} 

\begin{frame} 
  \frametitle{}
\[ 
\begin{aligned}
  a^{(1)}_n(t) &= \frac{H'_{mk}(r)}{i\hbar} \frac{1}{i \omega _{mk}} \left[ e^{i \omega _{mk}\tau }\right]^t _0 \\ 
  &= - \frac{H'_{mk}(r) }{ \hbar \omega _{mk}}\left[ e^{i \omega _{mk}t} -1\right] \\
  &= - \frac{H'_{mk}(r) }{ \hbar \omega _{mk}} e^{i \omega _{mk}t/2} \left[e^{i \omega _{mk}t/2} -e^{-i \omega _{mk}t/2}\right] \\
  &= - \frac{H'_{mk}(r) }{ \hbar \omega _{mk}}  2i e^{i \omega _{mk}t/2} \sin(\frac{1}{2} \omega _{mk} t )
\end{aligned}  
\]
跃迁概率 
\begin{equation*}
  \begin{aligned}
    W_{k\to m} &= \left\vert a^{(1)}_m(t)\right\vert ^2 \\
    &= \frac{4 \left\vert H'_{mk}(r)\right\vert ^2 \sin^2(\frac{1}{2} \omega _{mk} t ) }{\hbar^2 \omega ^2 _{mk}} 
  \end{aligned}
\end{equation*}
\end{frame} 

\begin{frame} 
  \frametitle{}
  考虑$t\to \infty$, 由数学公式
  \[ \lim_{\alpha \to \infty} \frac{\sin^2 (\alpha x)}{\pi \alpha x^2} = \delta(x)\]
  得
  \[ \lim_{t \to \infty} \frac{\sin^2 (t \frac{1}{2} \omega _{mk} )}{ t (\frac{1}{2} \omega _{mk} )^2} = \pi\delta(\frac{1}{2} \omega _{mk} ) = 2 \pi \hbar \delta(\varepsilon _m - \varepsilon _k)\]
代入,有跃迁概率 
\begin{equation*}
  \begin{aligned}
    W_{k\to m} =  \frac{2\pi t}{\hbar} \left\vert H'_{mk}(r) \right\vert ^2 \delta(\varepsilon _m - \varepsilon _k)
  \end{aligned}
\end{equation*}
跃迁速率(概率密度)
\begin{equation*}
  \begin{aligned}
    w_{k\to m} = \frac{W_{k\to m} }{t} = \frac{2\pi }{\hbar} \left\vert H'_{mk}(r) \right\vert ^2 \delta(\varepsilon _m - \varepsilon _k)
  \end{aligned}
\end{equation*}
常微扰只能导致简并态或极其临近态之间的跃迁。
\end{frame} 

\begin{frame} 
  \frametitle{黄金定则}
设末态$\varepsilon _m$附近$d\varepsilon _m$范围内态的数目为
$\rho(\varepsilon _m)d\varepsilon _m$, 则从初态 $d\varepsilon _k$ 跃迁到这些末态的总跃迁速率为
\[ 
\begin{aligned}
  w &= \int \rho(\varepsilon _m) w_{k\to m} d \varepsilon _m  \\ 
  &= \int \rho(\varepsilon _m) \frac{2\pi }{\hbar} \left\vert H'_{mk}(r) \right\vert ^2 \delta(\varepsilon _m - \varepsilon _k) d \varepsilon _m 
\end{aligned}  
\] 
在$\rho$和 $H'_{mk}(r)$都是平滑变化的情况下,有
\[ 
\begin{aligned}
  \boxed{w = \frac{2\pi }{\hbar} \left\vert H'_{mk}(r) \right\vert ^2 \rho(\varepsilon _m) } 
\end{aligned}  
\] 
跃迁速率与末态附近的态密度成比例,  上述公式称为\emf[黄金定则]。
\end{frame} 

\begin{frame} 
  \frametitle{}
\例[8]{求全空间自由粒子的态密度}
\解 考虑三维解的箱归一化形式 
\[ \psi _{\vec{p}}= \frac{1}{(2\pi \hbar)^{3/2}}e^{\frac{i}{\hbar} \vec{p}\cdot\vec{r}} = \frac{1}{L^{3/2}}e^{\frac{i}{\hbar} \vec{p}\cdot\vec{r}}
\]
其动量本征值为
\[p_x = \frac{2\pi \hbar}{L} n_x, \quad p_y = \frac{2\pi \hbar}{L} n_y, \quad p_z = \frac{2\pi \hbar}{L} n_z\]
由于第一组量子数$(n_x, n_y, n_z)$对应一个态, 所以在动量区域$\vec{p}\to \vec{p}+ d \vec{p}$内的总态数目为
\[ dn = dn_x dn_y dn_z =(\frac{L}{2\pi \hbar})^3 dp_x dp_y dp_z =  (\frac{L}{2\pi \hbar})^3 p^2 dp d  \Omega\] 
\end{frame} 

\begin{frame} 
  \frametitle{}
由能量与动量的关系,得
\[E = \frac{p^2}{2\mu} \implies dE = \frac{p}{\mu} d p \]
代入态密度定义式,得
\[ \rho(E) = \frac{dn}{dE} = (\frac{L}{2\pi \hbar})^3 \mu p d\Omega = (\frac{L}{2\pi \hbar})^3 \mu p \sin\theta d\theta d\varphi \]
\end{frame} 

\subsection{简谐微扰}

\begin{frame} 
  \frametitle{简谐微扰}
  1)微扰:如果含时微扰可以表示成如下分段函数形式
  \[ H'(t) = \left\{
    \begin{aligned}
     &0, \qquad (t<0)\\
     &A\cos \omega t , \quad (t \ge 0) \\
    \end{aligned}\right. \]
  称为简谐微扰。 利用欧拉公式, 有
  \[ A\cos \omega t = F(e^ {i \omega t} + e^ {-i \omega t}) \]
  式中$F$为振幅
  \end{frame} 

  \begin{frame} 
    \frametitle{}
  2)微扰矩阵 
  \[
    \begin{aligned}
      H'_{mk} &=  \left\langle \phi _m  \right\vert H'  \left\vert \phi _k   \right\rangle \\ 
      &= \left\langle \phi _m  \right\vert F(e^ {i \omega t} + e^ {-i \omega t})   \left\vert \phi _k   \right\rangle  \\
      &= (e^ {i \omega t} + e^ {-i \omega t}) \left\langle \phi _m  \right\vert F \left\vert \phi _k   \right\rangle  \\
      &= (e^ {i \omega t} + e^ {-i \omega t}) F_{mk}
    \end{aligned}
    \]
  \end{frame} 

  \begin{frame} 
    \frametitle{}
   3) 一级近似
   \[   \begin{aligned}
    a^{(1)}_m(t) 
    &= \frac{1}{i\hbar}\int_{0}^{t} H'_{mk}  e^{i \omega _{mk} \tau} d\tau \\
    &= \frac{1}{i\hbar}\int_{0}^{t} (e^ {i \omega \tau} + e^ {-i \omega \tau}) F_{mk}  e^{i \omega _{mk} \tau} d\tau \\
    &= \frac{F_{mk} }{i\hbar} \int_{0}^{t} (e^ { i \omega _{mk} \tau+i \omega \tau} + e^ {i \omega _{mk} \tau -i \omega \tau})   d\tau \\
    &= \frac{F_{mk} }{i\hbar}  \left[\frac{e^ { i \omega _{mk} \tau+i \omega \tau}}{i(\omega _{mk}+ \omega)} + \frac{e^ { i \omega _{mk} \tau - i \omega \tau}}{(i\omega _{mk}- \omega)} \right]^t _{0}\\
    &= - \frac{F_{mk} }{\hbar} \left[\frac{e^ { i (\omega _{mk} + \omega) t} -1 }{(\omega _{mk}+ \omega)} + \frac{e^ { i(\omega _{mk}  - \omega) t} -1}{(\omega _{mk}-  \omega)} \right]
 \end{aligned}
 \]
  \end{frame} 

  \begin{frame} 
    \frametitle{}
  4)设简谐微扰是光照,可见光频率很大,分子的值很小。
  \begin{itemize}
    \Item 当 $ \omega \approx \omega _{mk} $ 主要贡献源于第二项,有
    \[ \varepsilon _m = \varepsilon _k + \hbar \omega ,\qquad \text{共振吸收}\]
    \Item 当 $ \omega \approx - \omega _{mk} $ 主要贡献源于第一项,有
    \[ \varepsilon _m = \varepsilon _k - \hbar \omega ,\qquad \text{共振发射}\]
    \Item 当 $ \omega \ne \pm \omega _{mk} $  两项贡献都很小,没有显著跃迁发生
  \end{itemize}
  \end{frame} 

  \begin{frame} 
    \frametitle{}
  5)跃迁概率 \\
  共振跃迁: $ \omega \approx \omega _{mk}$ 
\[ a^{(1)}_m(t)  = - \frac{F_{mk} }{\hbar} \frac{e^ { i(\omega _{mk}  - \omega) t} -1}{(\omega _{mk} \pm  \omega)} \]
跃迁概率 
\begin{equation*}
  \begin{aligned}
    W_{k\to m} =  \frac{2\pi t}{\hbar} \left\vert F_{mk}\right\vert ^2 \delta(\varepsilon _m - \varepsilon _k \pm \hbar \omega)
  \end{aligned}
\end{equation*}
跃迁速率(概率密度)
\begin{equation*}
  \begin{aligned}
    w_{k\to m} = \frac{2\pi }{\hbar} \left\vert F_{mk} \right\vert ^2 \delta(\varepsilon _m - \varepsilon _k \pm \hbar \omega)
  \end{aligned}
\end{equation*}
  \end{frame} 

\begin{frame} 
  \frametitle{}
  5)细致平衡 \\
  由于$F$描述振幅,是厄密算符,有 $F_{mk} = F^* _{km} $,设 $\varepsilon _m > \varepsilon _ k$\\
  \begin{equation*}
    \begin{aligned}
    \text{吸收光子} \quad  w_{k\to m} &= \frac{2\pi }{\hbar} \left\vert F_{mk}\right\vert ^2 \delta(\varepsilon _m - \varepsilon _k - \hbar \omega) \\
    \text{发射光子} \quad w_{m\to k} &= \frac{2\pi }{\hbar} \left\vert F_{km}\right\vert ^2 \delta(\varepsilon _k - \varepsilon _m + \hbar \omega) \\
    \end{aligned}
  \end{equation*}
  即:两能级间的发射跃迁速率与吸收跃迁速率相等,称为细致平衡原理
\end{frame} 

\section{ 光的吸收和发射}

\subsection{吸收与发射系数}

\begin{frame} 
  \frametitle{三种过程}
  \begin{minipage}[b]{0.99\textwidth}
    \vspace{0.3em}
    光与原子相互作用表现为三种基本过程  
    \begin{itemize}
      \Item 共振吸收: 系数$B_{km}$,每个低能$k$态原子受激跃迁到$m$态的概率\emf[$B_{km}I$]
      \Item 自发发射: 系数$A_{mk}$,每个高能$m$态原子自发跃迁到$k$态的概率为\emf[$A_{mk}$](非相干光)
      \Item 受激发射: 系数$B_{mk}$,每个高能$m$态原子受激跃迁到$k$态的概率\emf[$B_{mk}I$] (相干光)
    \end{itemize}
    \vspace{2em}
  \end{minipage}
  \begin{minipage}[b]{0.99\textwidth}
 \begin{figure}[h]
    \centering
    \includegraphics[width=0.99\textwidth]{figs/bab.png}
    %\caption{}
    %\label{fig:}
\end{figure} 
\end{minipage}
\end{frame} 

\begin{frame} 
  \frametitle{两种框架}
  处理光与原子相互作用的两种框架
  \begin{itemize}
    \Item 半经典框架:量子化原子+经典光场 (初等量子力学)\\
    \Item 全量子框架:量子化原子+量子化光场 (量子光学与量子电动力学)
  \end{itemize}
  基于半经典框架,爱因斯坦成功获得自发发射系数、吸收系数和受汽发射系数之间的关系。
\end{frame} 

\begin{frame} 
  \frametitle{偶极矩近似}
  (I)忽略磁场分量的作用 \\
  光的电场与磁场分量对原子核外电子的作用能
  \[ U_E = e \vec{E}\cdot \vec{r} \approx e E a, \quad a = \frac{\hbar^2}{\mu e^2} \]
  \[ U_B = - \vec{M}\cdot \vec{B} =  - \frac{-e}{2\mu c } L_z B\approx \frac{e}{2\mu c } \hbar E \]
  二者的比值
  \[ \frac{U_B}{U_E} = \frac{e^2}{\hbar c } = \frac{1}{137} =\alpha \quad (\text{精细结构常数})\]
\end{frame} 

\begin{frame} 
  \frametitle{}
(II)均匀电场近似 \\ 
考虑沿$z$轴传播的单色偏振光,电场部分表示为
\[ \left\{
  \begin{aligned}
    &E_x =E_0 \cos(\frac{2\pi}{\lambda}z - \omega t) \\ 
    &E_y = E_z =0
  \end{aligned} \right.\]
  光子在原子核附近运动,原子的大小约为$10^{-10} m$, 而光波长约在$10^{-6} m$,因此在原子尺度,电场可作均匀场处理,有
  \[ E_x =E_0 \cos(\omega t)\]
\end{frame} 

\begin{frame} 
  \frametitle{计算过程}
  1)微扰  \\
  \[ 
    \begin{aligned}
      H' &=  exE_x \\ 
      &= exE_0\cos \omega t \\
      &= \frac{1}{2}exE_0[e^{i \omega t} + e^{-i \omega t}] \\
      &= F [e^{i \omega t} + e^{-i \omega t}]
    \end{aligned} \]
    振幅为 $ F = \frac{1}{2}exE_0$,计算 $E_0$
    \[ \overline{E^2} = \frac{1}{T} \int_0^T E^2_0 \cos^2 \omega t =\frac{1}{2} E^2_0\]
    \[I = \frac{1}{8\pi} (\overline{E^2 + B^2}) =  \frac{1}{8\pi} E^2_0 \quad \implies \quad  E^2_0 = 8\pi I\] 
\end{frame} 

\begin{frame} 
  \frametitle{}
  2)微扰矩阵 \\
  \[
  \begin{aligned}
    F_{mk} &=  \left\langle \phi _m  \right\vert F  \left\vert \phi _k   \right\rangle \\ 
    &= \left\langle \phi _m  \right\vert (\frac{1}{2}ex (8\pi I)^\frac{1}{2})  \left\vert \phi _k   \right\rangle\\
    &= \frac{1}{2}e (8\pi I)^\frac{1}{2} \left\langle \phi _m  \right\vert x \left\vert \phi _k   \right\rangle \\
    &= \frac{1}{2}e (8\pi I)^\frac{1}{2} x_{mk}
  \end{aligned}
  \]

\end{frame} 

\begin{frame} 
  \frametitle{}
  3)跃迁速率  \\
\begin{equation*}
  \begin{aligned}
    w_{k\to m} &= \frac{2\pi }{\hbar} \left\vert F_{mk} \right\vert ^2 \delta(\varepsilon _m - \varepsilon _k \pm \hbar \omega) \\
    &= \frac{2\pi }{\hbar} \left\vert \frac{1}{2}e (8\pi I)^\frac{1}{2} x_{mk} \right\vert ^2 \delta(\varepsilon _m - \varepsilon _k \pm \hbar \omega) \\
    &= \frac{4\pi ^2 e^2 }{\hbar^2} I \left\vert x_{mk} \right\vert ^2 \delta( \omega _{mk} - \omega)
  \end{aligned}
\end{equation*}
\end{frame} 

\begin{frame} 
  \frametitle{}
  4)回到自然光, (I) 去单色条件, 能量密度应是一个针对$\omega$的分布
  \[ d w_{k\to m} = \frac{4\pi ^2 e^2 }{\hbar^2} I(\omega) \left\vert x_{mk} \right\vert ^2 \delta( \omega _{mk} - \omega)\]
  \[ w_{k\to m} = \frac{4\pi ^2 e^2 }{\hbar^2} \left\vert x_{mk} \right\vert ^2 \int I(\omega)  \delta( \omega _{mk} - \omega) d \omega  = \frac{4\pi ^2 e^2 }{\hbar^2} \left\vert x_{mk} \right\vert ^2 I(\omega _{mk})\]
  (II) 去偏振条件 (设各向同性)
  \[ 
   \begin{aligned}
     w_{k\to m} &= \frac{4\pi ^2 e^2 }{\hbar^2} \frac{1}{3}[\left\vert x_{mk} \right\vert ^2 + \left\vert y_{mk} \right\vert ^2 + \left\vert z_{mk} \right\vert ^2 ]I(\omega _{mk}) \\ 
     &= \frac{4\pi ^2 e^2 }{3\hbar^2} \left\vert \vec{r}_{mk} \right\vert ^2 I(\omega _{mk}) \\
   \end{aligned}
    \]
\end{frame} 

\begin{frame} 
  \frametitle{}
代入电偶极矩定义 $\overline{D}_{mk} = e \overline{r}$,得
\[ w_{k\to m} = \frac{4\pi ^2 }{3\hbar^2} \left\vert \vec{D}_{mk} \right\vert ^2 I(\omega _{mk}) \qquad \text{吸收跃迁} \]
\[ w_{m\to k} = \frac{4\pi ^2 }{3\hbar^2} \left\vert \vec{D}_{km} \right\vert ^2 I(\omega _{mk}) \qquad \text{受激发射跃迁} \]
\end{frame} 

\begin{frame} 
  \frametitle{吸收系数与受激发射系数}
  吸收系数$B_{km}$: 单位时间光子被吸收的几率(光强描述总光子数)\\
  \[ B_{km} = \frac{w_{k\to m}}{I(\omega _{mk})} = \frac{4\pi ^2 }{3\hbar^2} \left\vert \vec{D}_{km} \right\vert ^2 \]
  受激发射系数$B_{mk}$:单位时间激发态原子受激发出光子的几率
  \[ B_{mk} = \frac{w_{m\to k}}{I(\omega _{mk})} = \frac{4\pi ^2 }{3\hbar^2} \left\vert \vec{D}_{mk} \right\vert ^2 \]
  由于$\overline{r}$是厄密算符,有 
  \[ B_{km} =  B_{mk} \]
\end{frame}

\begin{frame} 
  \frametitle{自发发射系数}
  自发发射系数$A_{mk}$: 在没有光照的情况下,单位时间激发态原子自发发射光子的几率 \\
设处于初态的原子数目为$N_k$,处于激发态的原子数目为$N_m$ \\
\begin{itemize}
  \Item 单位时间吸收的光子数等于从初态$k$跃迁到激发态$m$的原子数目
         \[N_k[B_{km}I(\omega _{mk})] \]
  \Item 单位时间发射的光子数等于从激发态$m$跃迁到初态$m$的原子数目 
        \[N_m[A_{mk} + B_{mk}I(\omega _{mk})] \]
\end{itemize}
\end{frame} 

\begin{frame} 
  \frametitle{}
    最终会达到电磁辐射平衡
    \[ N_k[B_{km}I(\omega _{mk})] = N_m[A_{mk} + B_{mk}I(\omega _{mk})]\]
    得能量密度:
    \[ I(\omega _{mk}) =   \frac{N_mA_{mk}}{N_kB_{km} -N_m B_{mk}} 
    = \frac{A_{mk}}{B_{mk}\left( \frac{N_k}{N_m}-1\right)}
    \]
    由玻尔兹曼分布律,有
    \[\frac{N_k}{N_m} = e^{\frac{\varepsilon _m - \varepsilon _k }{k_b T}} = e^{\hbar\omega _{mk} / k_b T} \]
    代回,得 
    \[ I(\omega _{mk}) 
    = \frac{A_{mk}}{B_{mk}} \left[ \frac{1}{e^{\hbar\omega _{mk} / k_b T} -1}\right]
    \]
\end{frame} 

\begin{frame} 
  \frametitle{}
  在 $\omega \to \omega + d \omega$ 区域, 光场的能量为
  \[ I(\omega _{mk}) d \omega = \frac{A_{mk}}{B_{mk}} \left[ \frac{1}{e^{\hbar\omega _{mk} / k_b T} -1}\right] d \omega \]
  黑体辐射平衡时,在$\nu \to \nu + d \nu$ 区域, 光场的能量为
  \[ \rho(\nu)d \nu =  \frac{8\pi h  \nu ^3}{c^3} \frac{1}{e^{h \nu /k_b T} -1} d \nu\]
  取 $d \omega = 2\pi d \nu$, 有
  \[ \frac{A_{mk}}{B_{mk}} \left[ \frac{1}{e^{\frac{\hbar\omega _{mk}}{k_b T}} -1}\right] 2\pi d \nu = \frac{8\pi h  \nu ^3}{c^3} \frac{1}{e^{h \nu /k_b T} -1} d \nu \]
\end{frame} 

\begin{frame} 
  \frametitle{}
解得 
\[ A_{mk} = \frac{4h \nu ^3 _{mk}}{c^3}  B_{mk}\]
或者
\[ A_{mk} = \frac{\hbar \omega ^3_{mk} }{\pi ^2 c^3}  B_{mk} =  \frac{4\omega ^3_{mk}}{3 \hbar c^3 } \left\vert \vec{D}_{km} \right\vert ^2  \]
自发辐射的能量
\[ J_{mk} = N_m A_{mk} \hbar \omega _{mk} = N_m \frac{4\omega ^4_{mk}}{3  c^3 } \left\vert \vec{D}_{km} \right\vert ^2 \]
\end{frame} 

\begin{frame} 
  \frametitle{激发态寿命}
  为求原子数目关于时间的函数,写出 $dt$时间内减少的原子数目
  \[
    \begin{aligned}
      d N_m &= - A_{mk} N_m dt  \\ 
      \frac{d N_m}{N_m} &= - A_{mk} dt \\
      \ln N_m &= - A_{mk} t +c. \\
      N_m(t) & = Ce^{-A_{mk} t}
    \end{aligned} 
    \]
    取$t=0$,
    \[ N_m (0) = C \]
    因此
    \[N_m (t)  = N_m (0)e^{-A_{mk} t} \]
\end{frame} 

\begin{frame} 
  \frametitle{}
  设$k$态下面还有大量的低能量,则有
  \[
  \begin{aligned}
    N_m (t)  &= N_m (0)\prod_{n=1}^k e^{-A_{mn} t} \\ 
     &=  N_m (0) e^{-\sum_{n=1}^k A_{mn} t}
     &= N_m (0) e^{-t / \tau}
  \end{aligned}  
   \]
   式中 $\tau = \frac{1}{\sum_{n=1}^k A_{mn}} $,称为平均寿命 

   ~~\\ 
   自发发射系数就是衰减系数,数值越大,衰减得越快,寿命越短。
\end{frame} 

\subsection{单电子选择定则}

\begin{frame} 
  \frametitle{禁戒跃迁}
  在偶极矩近似下,有跃迁速率公式
  \[ 
   \begin{aligned}
     w_{k\to m} &= \frac{4\pi ^2 e^2 }{\hbar^2} \frac{1}{3}\left[\left\vert x_{mk} \right\vert ^2 + \left\vert y_{mk} \right\vert ^2 + \left\vert z_{mk} \right\vert ^2 \right]I(\omega _{mk}) \\ 
     &= \frac{4\pi ^2 e^2 }{3\hbar^2} \left\vert \vec{r}_{mk} \right\vert ^2 I(\omega _{mk}) \\
   \end{aligned}
    \]
  表明:当 $\left\vert \vec{r}_{mk} \right\vert ^2 =0$时,跃迁速率为零,称为禁戒跃迁。
  
  ~~\\ 
  要实现有效跃迁,必有 
  \[\left\vert x_{mk} \right\vert ^2, \quad \left\vert y_{mk} \right\vert ^2 ,\quad \left\vert z_{mk} \right\vert ^2 \]
三者不同时为零。
\end{frame} 

\begin{frame} 
  \frametitle{选择定则}
  原子的波函数在球坐标系
  \[\psi_{nml} = R_{nl}(r)Y_{lm}(\theta,\varphi) = \left\vert nl \right\rangle \left\vert lm \right\rangle\]
  算符也用球坐标系
  $$\left\{\begin{aligned}
	  &x= r\sin \theta \cos \varphi = \frac{r}{2 }\sin \theta [e^{i\varphi} + e^{-i\varphi}]\\
	  &y= r\sin \theta \sin \varphi  = \frac{r}{2i }\sin \theta [e^{i\varphi} - e^{-i\varphi}]\\
	  &z= r\cos \theta
  \end{aligned} \right.$$ 
\end{frame} 

\begin{frame} 
  \frametitle{}
1)微扰矩阵元:
$$\left\{\begin{aligned}
  x_{mk}=  \frac{1}{2 }\left\langle n'l' \left\vert r \right\vert nl \right\rangle \left\langle l'm' \left\vert \sin \theta [e^{i\varphi} + e^{-i\varphi}] \right\vert lm \right\rangle  \\
  y_{mk}=  \frac{1}{2i }\left\langle n'l' \left\vert r \right\vert nl \right\rangle \left\langle l'm' \left\vert \sin \theta [e^{i\varphi} - e^{-i\varphi}] \right\vert lm \right\rangle  \\
  z_{mk}=  \left\langle n'l' \left\vert r \right\vert nl \right\rangle \left\langle l'm' \left\vert \cos \theta \right\vert lm \right\rangle \\
\end{aligned} \right. $$ 
(I)以前有计算 $\left\langle  Y_{l'm'} \right\vert \cos\theta \left\vert Y_{lm} \right\rangle$
$$
\begin{aligned}
 & \left\langle l'm' \left\vert \cos \theta \right\vert nlm \right\rangle = \left\langle  Y_{l'm'} \right\vert \cos\theta \left\vert Y_{lm} \right\rangle  \\
  & \qquad = \sqrt{\frac{(l+1)^2-m^2}{(2l+1)(2l+3)}}\left\langle Y_{l'm'} \left\vert Y_{l+1,m} \right\rangle \right. + \sqrt{\frac{l^2-m^2}{(2l-1)(2l+1)}} \left\langle Y_{l'm'} \left\vert Y_{l-1,m} \right\rangle \right. \\
  &\qquad = \sqrt{\frac{(l+1)^2-m^2}{(2l+1)(2l+3)}} \delta _{l'l+1} \delta _{m'm} + \sqrt{\frac{l^2-m^2}{(2l-1)(2l+1)}} \delta _{l'l-1}\delta _{m'm} 
\end{aligned}
$$ 
上式不为零,要求$l'=l\pm 1$且$m'=m$
\end{frame} 

\begin{frame} 
  \frametitle{}
  (II)计算 $\left\langle  l'm' \right\vert \sin\theta e^{\pm i\varphi}\left\vert lm \right\rangle$ \\
代入递推式
\[ \begin{aligned} \sin\theta e^{\pm i\varphi}\left\vert lm \right\rangle
&= \pm \sqrt{\frac{(l\pm m+1)(l\pm m+2)}{(2l+1)(2l+3)}} \left\vert l+1,m \pm 1 \right\rangle  \\
& \hspace{2em} \mp \sqrt{\frac{(l\mp m)(l\mp m -1)}{(2l-1)(2l+1)}} \left \vert  l-1,m\pm 1 \right\rangle 
\end{aligned}
\]
得
\[ \begin{aligned} \left\langle  Y_{l'm'} \right\vert \sin\theta e^{\pm i\varphi}\left\vert Y_{lm} \right\rangle 
  &= \pm \sqrt{\frac{(l\pm m+1)(l\pm m+2)}{(2l+1)(2l+3)}} \delta _{l'l+1} \delta _{m'm\pm 1}  \\
  & \hspace{2em} \mp \sqrt{\frac{(l\mp m)(l\mp m -1)}{(2l-1)(2l+1)}} \delta _{l'l-1} \delta _{m'm\pm 1} 
  \end{aligned}
  \]
  上式不为零,要求$l'=l\pm 1$且$m'=m\pm 1$
\end{frame} 

\begin{frame} 
  \frametitle{}
  (III)计算 $\left\langle  n'l' \right\vert r \left\vert nl \right\rangle$ \\
  \[ \begin{aligned} \left\langle  n'l' \right\vert r \left\vert nl \right\rangle 
    &=  \int_0^{\infty} R_{n'l'} r^3 R_{nl} dr  \\
    &\ne 0 
    \end{aligned}
    \]

  ~~\\
  综合(I)、(II) 、(III),得偶极矩近似下的\emf[选择定则] 
  \[
  \left\{ \begin{aligned}
     & \Delta l = \pm 1\\
     &  \Delta m = 0,\quad \pm 1
  \end{aligned}\right. 
  \]
\end{frame} 

\begin{frame} 
  \frametitle{}
\begin{figure}[htbp]
  \centering
  \includegraphics[width=0.9\textwidth]{figs/jump.png}
  %\caption{}
    %\label{fig:}
\end{figure}
试指出图中哪些跃迁是禁戒跃迁
\end{frame} 

\begin{frame} 
  \frametitle{严格禁戒跃迁}
\begin{itemize}
  \Item 不满足偶极矩近似选择定则的跃迁,称为\emf[禁戒跃迁]。
  ~~\\  
  \Item 四极矩近似、八极矩近似等更高级别近似有新的选择定则。不满足任意近似选择定则的跃迁,称为\emf[严格禁戒跃迁]。
\end{itemize}
\end{frame} 

\begin{frame} 
  \frametitle{1s与2p之间的跃迁}
  
  \begin{figure}[htbp]
    \centering
    \includegraphics[width=0.1\textwidth]{figs/1s2p.png}
    %\caption{}
      %\label{fig:}
  \[\Delta l = \pm 1, \quad \Delta m = 0 \]
  \end{figure}
  \begin{center}
    \animategraphics[height=2in,loop]{30}{figs/gif/sp}{1}{123} 
  \end{center} 
\end{frame} 

\begin{frame} 
  \frametitle{本章要点}
 \begin{enumerate}
  \item 非简并定态微扰求能量一二级修正及波函数的一级修正
  \item 简并定态微扰求能量的一级修正及及零级近似波函数
  \item 电偶极近似的内容
  \item 微扰公式适用条件
  \item $H^{(0)}$表象薛定谔方程
  \item 黄金定则的表述
  \item 选择定则的内容,禁戒跃迁的定义
  \item 什么是斯塔克效应及产生原因
  \item 态密度的概念及计算
  \item 光的发射与吸收三种基本过程的系数表达式
 \end{enumerate}
\end{frame} 